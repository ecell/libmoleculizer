\chapter{Useful Concepts in libmoleculizer}
\label{chap:conceptualOverviewChapter}

In the introduction, you should have learned what the libmoleculizer
program is: a computer system that models complicated biochemical
pathways (such as signal transduction pathways), by representing chemical
species as being composed of one or more indivisible molecule types
bound together and reactions between species as being special cases of
interactions between interations between molecules.  

This chapter gives an overview of the key concepts needed to
understand and use the libmoleculizer system.  

\section{Biochemical networks consist of chemical species and sets of
  chemical  reactions between them}
Libmoleculizer takes in models that represent an abstraction of a
biochemical network and explicitly generates that network.  What this
means is that libmoleculizer generates a complete list of species
along with all reactions those species can participate in.  

This list of species and reaction can be sent to another simulator, as
a list of variable names and reaction equations between them, and can
be used to simulate the entire network.  Nevertheless, this is always
what is mean when we talk about libmoleculizer expanding all or a part
of the network: using the rules to generate lists of species
identities and of reactions between species.

\section{Species consist of one or more indivisible units in complex}

Species in libmoleculizer are constructed by combining together one or
more indivisible reaction units, called molecules.  Molecules have
binding sites, and are joined together between their binding sites.
By taking one or more molecules and joining them together, a
structurally unique complex species is formed, which is a unique
member of the biochemical species in that reaction network.

\section{Molecules represent indivisible units of reaction}
As discussed previously, molecules are the indivisible components of
species.  No submolecular reactions can take place in a moleculizer
biochemical network.  The smallest level of interaction is at the
molecule level.  

\subsection{Molecules have a structure consisting of binding sites and
  modification sites}

Molecules themselves have internal structure.  Moleculizer recognizes
two types of molecules: modifiable-molecules, which may have any
number of unique binding sites that may participate to bind the
molecule into a larger complex , as well as any number of modification
sites that can accept different post-translational modifications.

There are also small-molecules, which are molecules that can
participate in reactions but which do not have differentiable binding
sites, or modification sites.  These molecules can be a part of a
complex species, but only in a simple way: they can only participate
in a single binding event at one time.  This level of resolution is
appropriate for modeling biological small molecules, which can be
considered a modular molecular component of species in which that
occur, but for which it would be innapropriate to model at higher
degrees of detail.


\section{Association and dissociation reactions between complex
  species are caused by the associations and dissociations on the
  binding sites belonging to the mols that make up that complex}

In libmoleculizer all dimerization (assocition) reactions, and the
reverse decomposition (disassociation) reactions are caused as a
result of interactions between different molecule-types.  In the
network model, an association interaction is defined between two
specific binding sites on two types of molecules.  When libmoleculizer
becomes aware of species that posess those molecules and which have
those binding sites free, an association reaction is created between
them.  
\section{Species transformation reactions are made possible by the
  species possessing a ``transformation-enabling'' subcomplex}
Association reactions are generated when libmoleculizer finds two
species, each of which possesses on of the two molecules involved in
the association with the relevant binding site free.  Transformation
reactions, such as those involved in kinase-reactions, in which a
protein kinase bound to a downstream reporter will phosphorylate a
target on that reporter, are expressed by defining a target
sub-complex: some bunch of molecules in a particular configuration
(there may be any number of molecules in the sub-complex, and the
sub-complex may have its state further specified to any specific or
general level of specification).  

With transformation interactions, libmoleculizer generates 
reactions in a fashion similar to association interactions.  Except
this time, libmoleculizer searches species that it has become aware of
to see if they possess the specified sub-structure within them.  When
it does, it creates the reaction by taking the found reactant species,
performing the specified transformation on it (possibly creating a new
species in the process), and adding the new reaction to the
libmoleculizer instance.  

\section{Allosteric reaction rates are differing rates of association
  and dissassociation amongst molecules, conditional on the states of
  the complexes the reaction molecules are found in}  

One feature observed in real biochemical networks is allosteric
binding and unbinding.  What this means is that a particular
association reaction will be observed to happen with a particular rate
(controlled for quantities of substance and other chemical
considerations).  However, when the species in which the associating
molecules live are in certain specific forms, rate of interaction
changes dramatically.  An example may be the behavior of a G-protein
coupled receptor, which typically has a very strong afflinity for its
Gbc subunit.  However, when the receptor binds ligand, the affinity
drops substantially, typically causing release of the Gbc subunit.
This is allostery.  

In libmoleculizer, allosteric interactions are modeled first by
allowing binding sites to be defined with multiple states,
representing different kinetic behavioral profiles.  Next, in another
section of the model where allosteric conditions are listed, any
number of conditions that would cause the binding site to go from its
default shape to the non-default allosteric state can be listed.

Finally, when entering the association interactions that can be
affected by allosteric rates, a default rate is given, followed by any
number of non-default, allosteric rates.  These non-default conditions
are specified by listing one or more pairs of shapes of participating
binding sites, followed by additional kinetic infomation which is used
when the allosteric conditions match.


\section{There are many names that can be given to a particular
  complex species, but only one UniqueID}  

Libmoleculizer generates explicit species and reactions listings of
biochemical reaction networks, which can be read out with the
libmoleculizer API to use in other programs for other purposes.  

Some words, however, should be said about the facilities for naming
species that exist in libmoleculizer, because otherwise difficulties
arise.  

The problem with naming species is that they are just a sets of
connected parts.  Because the sets of parts (molecule types and
bindings) have no ordering to them, there isn't any one name for the
species -- in fact there are as many names are there are lists can be
ordered.  

First, what IS NOT the problem.  For any species, we can get a name.
This is never a problem.  We can always take a name we've been given
and get back the same species every single time.  

What is the problem?  The problem is that because species have so many
names that can describe them, we might look at two species, and get
names for the two of them, and, because the names are different, 
conclude the two species must be different.  Even though the two
species are really the same, they could have different names. 

Because of this, libmoleculizer has two solutions.  The first is a
unique, temporary name, which is called a ``Tag'' in other parts of
the manual.  The ``Tag'' or ``Tagged Name'' as it is known is a name
that is unique over a particular run of libmoleculizer.  As long as
tags are only compared during a single program run (don't save a tag
and try to use it later for instance) they are your best option.

The second solution is a permanent name, which we call a ``Unique
ID''.  The unique id is a name that persists beyond program runs.  An
identical species will always generate the same unique id, no matter
what.  Accordingly, they are good for activities where biochemical
networks are saved to disk, and need to be compared between simulation
runs.  The downside to using unique ids is that they are slower, when
compared with tags.  Tags are preferred in all situations in which
they can be used.
