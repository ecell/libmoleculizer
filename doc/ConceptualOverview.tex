\chapter{Useful Concepts in libmoleculizer}
\label{chap:conceptualOverviewChapter}

The objective of libmoleculizer is to take in a description of
proteins and other molecular entities along with a list of rules
describing various biochemical interactions, and then use this
information to generate information about the species and reactions in
the reaction network that are implicitly described by those rules.

In order to understand in detail how libmoleculizer generates species
and reactions, including kinetics, that makes up the implicit reaction
networks, there are several key background concepts about what
libmoleculizer represents, and how it understands and interprets rules
of interaction.

\section{Combinatorial Reaction Networks}
TODO Write this section.

This section will discuss how protein-protein interactions can lead
under the right conditions to a combinatorial explosion, where the
system size grows combinatorially to the number of rules.  Discuss
what this means, in terms of not being able to explicitly write the
system down.

\subsection{Signal Transduction Pathways}
TODO Write this section.

This section will discuss very quickly what signal transduction
pathways are and why they are important, and how they usually fall
into the category of systems which display combinatorial complexity.  

\section{Rule-Based Reaction Network Specifications}
\label{concRuleBasedSpec}
\index{rule-based modeling}

The fundamental principle underlying libmoleculizer is that reaction
networks can be specified by describing the basic constituents of
chemical species, such as types of proteins and other biochemical molecules
(including small-molecules, but also including other larger molecules
such as DNA) that exist in the network, along with a set of
rules, each of with describes an individual biochemical interaction
amongst those types.  We call this type of modeling rule-based
modeling.  


\subsection{Species are made from structural bindings between
  structural molecules}

In libmoleculizer, the physical objects that participate in reactions
are called species. Species are different structural orientations of
objects called mols. Each mol represents a variety of indivisible
biochemical entity; typically the types of mols will be various kinds
of proteins, but also can mean different small molecules, DNA, and
potentially even objects such as organelles as well.

Each mol has its own name and is defined with a set of modular binding
sites.  Each of these binding sites has its own uniquely identifying
name, as well as a collection of shapes, which represent the different
conformational states that binding site may be, thus affecting binding
kinetics.  

Mols bind together between their binding sites in order to form
species.  This includes the special case where a complex species
consists of only a single mol that is bound to nothing else.

Mols also may be defined with modification sites, which are different
locations on the mol are associated with different modifications,
post-translational modifications such as phosphorylation groups, that
are relevant to the state of the mol, but which do involve a binding
between two mols.  

\subsection{Species have structural features based on their
 constituent mols}

The relevance of species is that they have structural features based
on the structural features of the mols that make them up.  For
instance, if a species is composed of an A protein and a B protein,
and the A protein has a modification site that is phosphorylated, then
when we bind that A protein together with the B protein, in a
molecular binding between two of their binding sites, the complex
species, the A-B dimer, will also posess a modification site that is
phosphorylated. Likewise, the dimer will posess all the same binding
sites as did the individual mols before they were bound.

Features in libmoleculizer refer to specific structural features,
defined in terms of either mols or sets of mols, that individiual
species may or may not possess. For instance, a feature might be the
condition that a particular modification site on a specific type of
mol has a certain modification state -- e.g. that a particular
modification site on a Ste11 protein kinase must be
phosphorylated. Any complex species containing one or more a Ste11
mols with that modification site phosphorylated will match that
feature.

Another structural feature, this time concerning two mols, would be
the presence of a particular binding between two mols at two binding
sites - e.g. the condition that there is a binding that joins a Ste5
to a Ste11. Any complex that contains a Ste11 bound to a Ste5 between
the correct binding sites will match that feature.

There is no limit in principle to the complexity that features in
libmoleculizer can have. It is possible to talk about a feature which
is the condition of possessing a complicated sub-complex (called an
omni-plex in libmoleculizer), with a particular pattern of
modification states on it.  

\subsection{Rules are amongst features}
As stated before, rules in libmoleculizer specify the different types
of biochemical interactions in a MZR model.  These rules are then used
to generate new species and new reactions between them based upon an
initial set of initial species. The underlying metaphor in how
libmoleculizer specifies interaction rules is to say that all
interactions can be specified in terms of features and the features
react together.  Reaction rules specify features, and reactions
between the features.  Libmoleculizer applies the feature reaction
transformations specified in the rules to any sets of species
possessing the relevent features.  This process is applied
iteratively and thus, a network of species and reactions is generated.  

The features that are required by each type of rule is rule-specific
to the three kinds of rules.  The first type of rule, called a
dimerization-gen, which specifies a particular binding interaction -
it looks for two features, one matching a free binding site on a
particular mol type and the other with another free binding site on
another type of mol, and when two species are found that match these
features, the two free sites are bound together.  The second type of
rule is an omni-gen, which looks for a particular omni-plex, a mol
sub-complex of species, with any portion of its modification state
specified.  When it finds this omni-plex, any number of small molecule
exchanges or modification exchanges may be performed on that
omni-plex.  Finally, the uni-mol-gen, which looks for a specific type
of mol with a potentially partially specified modification state; on
this mol any number of small molecule exchanges or modification
exchanges may be specified.  Note that this is a special case of the
omni-gen, and will likely be depreciated in the next version.

Each of these three types of rule, dimerization-gen by looking for two
free site features, omni-gen looking for an omni-plex feature, and
uni-mol, looking for a single mol matching feature behaves similarly:
it looks for features, and when finding them, has a specified
transformation it applies to the species.  The dimerization-gen joins
the two free binding sites together, and both the omni-gen and the
uni-mol-gen perform some kind of state change - a small mol exchange
or a modification exchanage - on the matching structural feature.

\subsection{Reaction rules create species and reactions from an
  initial list of species}

Either in the set of rules or by using the libmoleculizer API, the
user must specify a list of initial species that exist within the
model. For sets of species containing the features needed for the
different rules, libmoleculizer will apply the rule's transformation
to the matching substrate species, in order to create a set of product
species, which may or may not be new.  This mapping of substrate
species to product species is added as a reaction to the list of
reactions.  In addition, kinetics are generated based on the reaction
rule and added to the reaction, a topic to be discussed later in this
chapter. At this point, the reaction rules have been applied to the
initial set of species, which, when applied, create a new reaction
with potentially new product species.  If any of the product species
are new, they are checked for features, which may set off the
application of rules, causing more new reactions with potentially new
product species to be formed.  The whole reaction network relative to
a set of initial species includes all species and all reactions that
might be created on any iterative step of the expansion process.  

\section{Specifying Reaction Kinetics} 

For any reaction rule, one or more reaction rates must be specified
that indicate the kinetics of the reaction.  Uni-mol and omni-gen
reaction rules take a single reaction rate.  A dimerization-gen
reaction rule takes two: dimerization reactions are assumed to be
reversible by default.  Irreversible reaction rules are expressed as a
reversible reaction with an off rate of 0.  

There are two ways this kinetic parameter can be interpreted.  If no
reaction rate extrapolation is used, then the on-rates supplied in the
rule are passed into the generated reaction directly.  Using this
method, users can use whatever units they would like.

Libmoleculizer also has the potential to extrapolate a reaction rate
from the kinetics provided based on the masses of the substrate
species involved. 

%% TODO figure this out and get a reference in here.
This assumption is used because of a particular equation suggested in
Gillespie.

%% Figure out these units here.
When reaction rate extrapolation is used, the rate associated with a
binary reaction is multiplied by the reduced-mass of the the substrate
species, in order to better approximate the way that the masses of the
species affect the collisions due to collisions that underly the
reaction.  In this interpretation, units must be in either X or Y.

\section{Allosteric Kinetics are Based on Binding Shape}
\label{sec:allosterickinetics}

Biochemical reaction networks, particularly signal transduction
pathways, often use allosteric proteins to regulate their
behavior. Allosteric proteins are proteins whose shape, and
consequently behavior, with regards to different interactions can
change into one of multiple different forms depending upon the state
of certain modifications on that protein as well as what entities that
protein is bound to.  One example would be a model in which a protein
Fic10 can bind to another Sca3, after it has been activated by being
doubly phosphorylated on two of its modification sites.  

How does libmoleculizer treat an allosteric reaction?  Say we would
like to express the above reaction: Sca3 binds with Fic10, say from
the ``to-Fic10'' and ``to-Sca3'' sites respectively, but only when
Sca3 has been doubly phosphorylated, say at it's ``phos-site-one'' and
``phos-site-two'' sites; how can this be done?  Libmoleculizer
decomposes a reaction like that into defining the charectoristics that
cause the behavior of that molecule binding site (in this case the
``to-Fic10'' site of Sca3) to change.  These charectoristics, another
feature that can be tested for in new species, are associated with
binding shape changes.  In this case, the binding site ``to-Fic10''
may have been defined with the mandatory default-binding-site-shape
parameter equal to ``half-activated''; specifying the feature that
a Sca3 has both of its phos-sites phosphorylated would be associated
with the binding-site-shape change that when this feature is matched
the matching Sca3 mol's ``to-Fic10'' site is updated to the shape
``fully-activated''.  Second, in the applicable reaction rule the
allostery applies to, describe the allosteric kinetic rates in terms
of the binding site shapes that supply the allostery. 

For instance, to represent above reaction describing the doubly
phosphorylated Sca3-Fic10 dimerization, we start by providing
information that the Sca3 binding site has default shape ``inactive''
and define that when it is doubly phosphorylated at its
``phos-site-one'' and ``-two'' sites, that shape becomes ``active''.
Next, we define the dimerization rule representing Sca3 and Fic10
dimerization with a dimerization-gen.  The default kinetics described
in that reaction rule applywhen the two binding sites are have their
default shapes.  In this case, we would want to express that that
reaction does not occur, so we give the on-rate as 0, and the off-rate
as a very high number to indicate that does not happen.  Inside that
rule, we would also specify the allostery, with an shape-based
'exception' to the default kinetics: i.e. if the ``to-Fic10'' site has
shape ``activated'', then in that case the on- and off-rates should be
given as the activated kinetic rates.  

\section{Design rationale for using binding site shapes to describe allostery}
TODO Write this section

This section will discuss why the designers of this software chose to
use binding site shapes to describe allostery.  The alternative is to
use something like the BioNetGen schema for specifying rules.
Effectively, rather than defining omniplexes that transform certain
binding sites within them and then defining allostery based on which
binding sites are participating in a reaction, platforms like
BioNetGen specify rules in terms of the omniplexes themselves.  This
is a relatively big difference between the two: this section makes the
case as to why we believe our method is preferable.  

\section{The Naming of Complex Species}
One problem that libmoleculizer has had to solve regarding complex
species is that of automatically generating names.  From names, we
hope we can get out several features.  These are discussed below.  

The purpose of a name is to uniquely identify a species.  That is, a
naming process is a function that takes complex species and generates
corresponding names, which are strings of charectors.  While many
processes can generate strings from species, what makes naming
processes naming processes is that the strings have two properties.
The first is that the process of naming is injective.  That is, two
species have the same name if and only if they are the same species.  
The second property is that the naming process is in some sense
invertible: if you have the name, you can recover the species that
belongs with that name.  

From this, libmoleculizer has two kinds of names that it can
generate, tagged names and unique ids.  The difference is the scope
under which the two conditions listed above work.  Tagged names
meet the two conditions of being names listed above within a single
run of the program; unique ids meet the two conditions between
seperate runs of the program -- we say they are serializable.  


\section{Tagged names for naming complex species within a single session
  of libmoleculizer}

The first method of naming species within libmoleculizer is with
tagged names.  These names are short, typically 8 charecters long.
Within a single run of libmoleculizer, tags can be used to compare two
species: two complex species produced \it{within the same run of
  libmoleculizer} are structurally the same if and only if their tags
are identical.  Additionally, given that a tagged name has been
produced, that tag can be used to recover the species, as long as this
all occurs within the same run.  However, none of these naming
guarentees carry over between seperate runs.  Between seperate runs of
libmoleculizer, identical complex species may generate different tags,
non-equal species may generate identical tags, and from one run to
another, it is not possible to take a tag and recover the correct
species.  

However, within a single run, the advantage of tagged names is that
they are both much shorter that unique ids and they are also much
faster to generate.  The tagged name is very quick; basically a
constant time process. \footnote{When species are generated in
  libmoleculizer, they are checked to see if they are new.  If they
  are, they are registered.  If they are not new, they are associated
  with the first isomorphic complex species that was registered, which
  stands in as the representative for that isomorphism class of
  complex.  The tagged name is the ``stringification'' of the pointer
  to the isomorphism class for the complex species, and as such, is
  a constant time process.}  As such, if names of species are needed
for uses within single runs, it is \textbf{always preferable to use
  tagged names}.  


\section{Unique IDs for Naming Complex Species}
Many uses of names involve serialization: that is, uses which
compare names between one instance of moleculizer and another.  These
may be for things like comparing the results of different simulation
runs, \footnote{Of course, libmoleculizer does not simulate anything,
  and no absolute guarantees can be made about the architectures of
  programs which use libmoleclizer.  However, we expect that most
  simulators which use libmoleculizer will create a new libmoleculizer
  for each simulation run.  Whenever this is the case, unique ids
  should be used to compare the results of the different runs.},
putting information into databases, recording names with the desire of
finding out information about the corresponding species at some future
date, etc.  For these uses, we need a stronger guarantee for what
names can give us: we need names that uniquely identify species
between instances of libmoleculizer; we also need names for which we
can recover the complex species given a name at any time.  Unique ids
fulfil these requirements.  

While it is outside the scope of this document to discuss in deep
detail how these names are produced, we will quickly cover this
information here.  Any complex species consists of a set of molecules
that are bound together in some fashion.  Any linear description, such
as a string, then which describes it, will consist of a list of
species, and then a list of bindings.  Therefore, for each list of
molecules in a species (for a complex species consisting of $n$
molecules, there are $n!$ lists that describe those molecules
(however, if some of the molecules are identical, some of the lists
will be the same).  For each of these lists, we can create a set of
bindings, where each binding represents something like ``the second mol
in the list binds at its first binding site to the third mol in the
list at its third binding site''.  For a particular list of species,
we can say that the correct way to write the binding is to have the
mol with the smaller mol index listed first and the larger indexed mol
listed second; then the correct way to write the set of bindings is as
a list of bindings, where the list of bindings is ordered
lexicographically. \footnote{ Lexicographically is a fancy way of
  saying alphabetically.  As discussed before, each binding is a set
  of four numbers $(a,b,c,d)$, which basically means ``the $a^{th}$ mol in
  the list binds at its $b^{th}$ site to the $c^{th}$ mol at its
  $d^{th}$ site''.  To say that a binding $(a_1, b_1, c_1, d_1) <
  (a_2, b_2, c_2, d_2)$ lexicographically, we mean that $a_1 < a_2$ or
  $a_1 = a_2$ and $b_1 < b2$ or that $a_1 = a_2$, $b_1 = b_2$, and
  $c_1 < c_2$, and so on.} Finally, for any list of mols, we can
create a canonical list of modification states in the same way as we
created the canonical binding list for a given list of mols.  So for
any list of the mols in a species, we can use this method to define a
``correct'' name for the species given that list.  We then can define
the unique id for that species to be the lexicographically least name
over all the ``correct'' names for that species.  This unique id is
``canonical'', which is to say that its identity depends only on the
structure of the species and nothing else.  

Because of the way in which unique ids are generated, two species will
have the same unique id if and only if they are structurally
identical.  Furthermore, because the name is a structural description
of the species, libmoleculizer can use a name to regenerate and then
produce the corresponding complex species at any time.  This is the
advantage of the unique id.  However, the disadvantages are that the
production of these names takes a relatively long time.  While the
novel algorithm used to generate the unique id is relatively fast, it
is much slower than the constant time performance of the tagged
names.  Additionally, because the unique id contains the structural
information contained in the species (in order to recreate it), it may
be quite long.  Arbitrarily large species will have arbitrarily long
names.  \footnote{In a model run by the authors, a complex of 16 mols
  and with a name of 658 charecters was produced.  However, if a model
  file has rules enabling larger complexes, there is the potential for
  names which are much longer as well.}  

In conclusion, unique ids have the powerful capability that they are
true names.  Unique ids can always be compared amongst one another,
and may always be used to regenerate the corresponding complex
species.  However, because of their size and the time taken to
generate them, they should not be used unless they are needed, instead
prefering the use of the tagged names.  
