\chapter{Useful Concepts in libmoleculizer}
\label{chap:conceptualOverviewChapter}

In the introduction, you should have learned what libmoleculizer is
exactly: a computer system that models complicated biochemical
pathways such as signal transduction pathways by representing chemical
species as being composed of one or more indivisible molecule types
bound together and reactions between species as being special cases of
local interactions between the component molecules that make up a
species.  

This chapter gives an overview of these key concepts.  

\section{Biochemical networks consist of chemical species and sets of
  chemical  reactions between them}
Libmoleculizer takes in models that represent an abstraction of a
biochemical network and explicitly generates that network.  What this
means is that libmoleculizer generates a complete list of species
along with all reactions those species can participate in.  This list
of species and reaction can be sent to another simulator, as a list of
variable names and reaction equations between them, and can be used to
simulate the entire network.  

\section{Species consist of one or more indivisible units in complex}

Species in libmoleculizer are constructed by combining together one or
more indivisible reaction units, called mols.  Mols have binding
sites, and are joined together between their binding sites.  By taking
one or more mols and joining them together, a structurally unique
complex species is formed, which is a member of the set of biochemical
species in that reaction network.  

\section{Molecules represent indivisible units of reaction}
As discussed previously, molecules are the indivisible components of
species.  No submolecular reactions can take place in a moleculizer
biochemical network.  The smallest level of interaction is at the
molecule level.  

\subsection{Molecules have a structure consisting of binding sites and
  modification sites}

Molecules themselves have internal structure.  Moleculizer recognizes
two types of molecules: modifiable-molecules, which may have any
number of unique binding sites that may participate to bind the
molecule into a larger complex , as well as any number of modification
sites that can accept different post-translational modifications.

There are also small-molecules, which are molecules that can
participate in reactions but which do not have differentiable binding
sites, or modification sites.  These molecules can be a part of a
complex species, but only in a simple way: they can only participate
in a single binding event at one time.  This level of resolution is
appropriate for modeling biological small molecules, which can be
considered a modular molecular component of species in which that
occur, but for which it would be innapropriate to model at higher
degrees of detail.

\section{Association and dissociation reactions between complex
  species are caused by the associations and dissociations on the
  binding sites belonging to the mols that make up that complex}

Association interactions are specified in terms of specific binding
sites on specific molecules (modifiable-molecules as well as
small-molecules). 



\section{Species transformation reactions are made possible by the
  species possessing a ``transformation-enabling'' subcomplex}


\section{Allosteric reaction rates are differing rates of association
  and dissassociation amongst molecules, conditional on the states of
  the complexes the reaction molecules are found in}  

\section{There are many names that can be given to a particular
  complex species, but only one UniqueID}  

\section{}
\section{}


1. A biochemical reaction network consists of a set of chemical
species and a set of reactions between them

2. chemical species consist of one or more indivisible units in complex

3. Molecules represent indivisible units of interaction
  a) Molecules have internal structure, consisting of binding sites
and modification sites.
  b) Binding sites can be defined as having multiple states/shapes.

4.  The law of moleculizer:  Interactions between molecules are caused
by interactions between the structural elements of those molecules.
Reactions between species are a result of interactions between the
molecules that make up the participating species.

XXXX -- MAKING A LIBMOLECULIZER MODEL MEANS DEFINING A SET OF
MOLECULES (indivisible reacting elements) AND A SET OF INTERACTIONS
BETWEEN THEM

4. Association and dissociation reactions between complex species are
caused by the associations and dissociations of binding sites
belonging to the mols that make up that complex.

5. Species transformation reactions are made possible by the species
possessing a ``transformation-enabling'' subcomplex.

6.  Allosteric reaction rates are differing rates of association and
dissassociation amongst molecules, conditional on the states of the
complexes the reaction molecules are found in.
a)  By specifying conditions under which participating binding sites
may take on different shapes/states, as well as specifying alternate
kinetics for association reactions based on the states of the
participating binding sites, any allosteric profile can be
represented.

7.  There are many names that can be given to a particular complex
species, but only one UniqueID.