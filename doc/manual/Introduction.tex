\chapter{Introduction}

\section{Overview of Libmoleculizer}
Libmoleculizer is a computer library that takes a set of rules that
describe a collection of types of proteins and small molecules, as
well as a set of interactions amongst those types, such as
protein-protein, protein-small molecule, and enzymatic reactions
(protein modifications), and from that information explicitly
generates the network of species and reactions described by those
rules.  

The capacity to describe networks implicitly, i.e. by using rules to
generate a network can be be very adventageous.  This is because of
the exponential growth of the explicit networks relative to the number
of protein/molecule types and interactions amongst them.  

%% TODO -- Finish writing the introduction of what libmoleculizer is 
%%      -- here.

In fact, these systems can often be so large in terms of absolute
numbers of potential chemical species and reaction channels, that, in
spite of the fact that number of molecule types and numbers of
biochemical interactions are relatively small, an explicit enumeration
of these networks is virtual impossible by human hands.  For example,
the alpha response signal transduction pathway contains approximately
15 types of protein, and approximately 20 biochemical interactions.
Taking into account allosteric effects, the total number of
interactions may be as high as 40.  In spite of the relative
comprehensibility of the set of these proteins and interactions, the
absolute size of the reaction network is something like 100000 species
and 300000 reactions.  

Libmoleculizer is a computer tool that manages the absolute set of
species and reactions, by expanding the saner network of biochemical
interactions that users provide.  Libmoleculizer is intended to be
used as component of a larger simulator.

%% END TODO

\section{Manual Overview}
This manual aims to provide introduction to the functionality and use
of the libmoleculizer library.  

This manual can be divided up into three aspects.  The first objective
is to describe what Moleculizer does; this is treated in the chapters
'Conceptual Overview' and 'Using Moleculizer'.  The second is understanding the
MZR file format in which reaction rules are specified for
libmoleculizer.  This is treated in the 'Rules' chapter as well as the
'Mzr File Format Reference' chapters.  Finally, the libmoleculizer API
must be learned, in the 'Using', 'Interfaces', and 'API Reference'
chapters.

Finally, although libmoleculizer is a relatively standard source code
package that should be fairly portable, installation instructions and
some troubleshooting help is available in the 'Installing' chapter.

\section{History}
Test, libmoleculizer was first described in \cite{lok05}.

Libmoleculizer is the decendant of a prior program called Moleculizer,
develeped by Dr. Larry Lok at the Molecular Sciences Institute to
study the alpha mating pathway in yeast. The standard modeling
techniques at the time all involved explicitly enumerating a set of
chemical species and reactions. However, it was determined for the
alpha signal transduction pathway in yeast that there were so many
species and reactions possible, that they could never be practically
enumerated. The typical response to this problem had previously been
to simplify the system, combining species and reactions. However,
this approach was non-optimal, as it required making fundamental
assumptions about what was and was not important, prior to
investigating system behavior. 

The response was to develop a new simulator called Moleculizer that
took in a description of the basic monomeric proteins and small
molecules, as well as a set of rules that describing the basic
interactions between proteins. These rules were used within a
Gillespie simulator, which would alternate between executing Gillespie
reaction events and using the rules in order to generate enough of the
reaction network so that the immediately following reaction event
would be accurate.  

Realizing the versatility of this new rule-based approach, a new
software project was begun by Nathan Addy at the Molecular Sciences
Institute, to adapt the reaction network generating components of
Moleculizer into a generic library, called libmoleculizer, that could
be used within a variety of simulators and in a variety of contexts.

This manual corresponds to version \currentversion of that software.