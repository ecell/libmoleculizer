\chapter {API Reference}
\label{chap:apiReference}

This chapter lists the functions that occurs in each of the APIs and
documents them in a flat style.

\section{C++ API Reference}

Note that all these functions are contained in the mzr namespace.
Therefore to use libmoleculizer, either a ``using namespace mzr''
directive must be used, or the ``mzr::'' namespace must be appended to
certain commands.  

\subsection{A short word on including, compiling, and linking}
To use this software, the file libmoleculizer/mzr/moleculizer.hh must
be included.  However, all header files are installed to
\${Prefix}/libmoleculizer-\currentversion, where \${Prefix} is the
base installation directory, e.g /usr or /usr/local.  Therefore you will have to
add the compiler flags ``-I\${Prefix}/libmoleculizer-\currentversion''
to your compiler flags.  While this can be done manually, a much, much
easier method is to use pkg-config.  While you should consult the
pkg-config man page for more information, if you simply add the
results of ``pkg-config libmoleculizer-\currentversion --cflags'' to
your compilation step, and ``pkg-config libmoleculizer-\currentversion
--libs'' to your linking step, everything will be taken care of
automatically.  


Conclusion: add \#include ``libmoleculizer/mzr/moleculizer.hh'' to any
files that use libmoleculizer, add the flags listed in ``pkg-config
libmoleculizer-\currentversion --cflags'' to your compilation steps,
and add the flags listed in ``pkg-config
libmoleculizer-\currentversion --libs'' to your linking steps.  

\subsection{The moleculizer class}
The major object of libmoleculizer is the moleculizer class, which
represents a reaction network defined by a set of rules.  

\subsubsection{moleculizer::moleculizer(void)}
This is the one and only constructor of moleculizer.

\subsubsection{void moleculizer::setGenerateDepth( unsigned int
  generateDepth )}
This function sets the generation depth.  The value of the generation
depth determines the behavior of expanding species and reactions.
When the function expandReactionNetwork is called on either a reaction
or a species, the network will be expanded to that many levels of
generationDepth.  If the generationDepth is set to 0, no expansion and
no species/reaction generation will ever be done.  If the
generationDepth is 1 species and reactions will be generated, but they
themselves will not be expanded.  If the generation depth is 2,
species and reactions will be generated, and each of those will
themselves be expanded, however the species and reactions they
generate will not be expanded.  A generation depth of 3 will expand
those, but not the next level, and so on. \textbf{The default
  generation depth is 1.}  Setting a generation depth that is too high
may impede performance.  

\subsubsection{int moleculizer::getGenerationDepth( void ) const}
Calling this function will return the generation depth as currently
set in a moleculizer object.  

\subsubsection{void moleculizer::setRateExtrapolation( bool
  rateExtrapolation )}
This function sets rate extrapolation for the moleculizer object.
Accordingly, it must be set prior to loading a model; if it is not,
this function throws a mzr::modelAlreadyLoadedXcpt exception.  If
it is set to false, the reaction rate for a created reaction is the
same as the reaction rate supplied in the reaction rule.  If it is set
to true and the generated reaction is binary, the reaction kinetics
supplied in the rules will be multiplied by the reduced masses of the
two species: $\sqrt{ \frac{ m_1 * m_2}{m_1 * m_2}}$.  Please see
\ref{chap:conceptualOverviewChapter} for more details.  The default
property of moleculizer is to not use reaction rate extrapolation.

\subsubsection{bool moleculizer::getRateExtrapolation() const}
This function returns the current value of the rate extrapolation.

\subsubsection{void moleculizer::attachFileName( const std::string\&
  fileName)}

This function takes a string consisting of the filename of a mzr rules
file.  Calling it reads the file and loads it into the moleculizer
object.  If a file has already been loaded into moleculizer, this will
throw a utl::modelAlreadyLoadedXcpt exception.  If the file does not
exist, or otherwise cannot be read or parsed, this function will throw
a utl::dom::noDocumentParsedXcpt.

\subsubsection{void moleculizer::attachString( const std::string\&
  fileString)}
Calling this function with a string containing the contents of mzr
file will cause it to be loaded into moleculizer.  If a file has
already been loaded into moleculizer, this will throw a
utl::modelAlreadyLoadedXcpt exception.  If the string cannot be
parsed, this function will throw a utl::dom::noDocumentParsedXcpt
exception.

\subsubsection{void moleculizer::attachDocument( xmlpp::Document*
  pDoc)}
Calling this function with a document that has already been parsed
using libxml++ will load it into the model.  {\em It is not expected
  that this function will be generally used by users}.  If a model has
already been loaded into moleculizer, this function will throw a
utl::dom::noDocumentParsedXcpt. 

\subsubsection{virtual void moleculizer::generateCompleteNetwork()}
Calling this function will cause the moleculizer object to generate
the complete reaction network (i.e. continue expanding species and
reactions until no new species and reactions can be generated
according to the rules).  If a model has not yet been loaded, this
function will throw a mzr::ModelNotLoadedXcpt exception.
\textbf{Beware that this function may take a very long time to run, if
  it returns at all.}  

\subsubsection{ CachePosition
        generateCompleteNetwork(long maxNumSpecies, long maxNumRxns =
        -1) }

Use this function to generate a network that is bounded by
maxNumSpecies and maxNumRxns (if a maxNumRxns parameter is
specified).  This function returns a CachePosition, which is a
std::pair, where the first entry is an iterator to a position
in the deltaSpeciesList of type std::list<mzr::mzrSpecies*>::iterator
and the second entry is an iterator to a position in the
deltaReactionList of type std::list<mzr::mzrReaction*>::iterator.  By
iterating from theDeltaSpeciesList.begin() to the CachePosition.first,
and by iterating from the DeltaReactionList.end() to the
CachePosition.second, a complete network which possesses fewer than
maxNumSpecies and (if specified) fewer than maxNumRxns.

\subsubsection{bool moleculizer::getModelHasBeenLoaded() const}
This function returns true if a model has been loaded into the
moleculizer object and false otherwise.  

\subsubsection{int moleculizer::getNumberOfPlexFamilies() const}
A plex family in moleculizer is a structurally identical collection of
species.  That is, for a given complex species, its corresponding plex
family is the set of all complex species that only differ in terms of
their modification values.  This function returns the number of plex
species in moleculizer.  

\subsubsection{void moleculizer::getSpeciesStreams(
  std::vector<std::string>\& speciesStreamNames) const}
This function takes a reference to a vector of strings (semantically
the vector probably should be empty, but does not have to be), and
places into it the names of each of the species streams (see Chapter
\ref{chap:conceptualOverviewChapter} for a discussion of species
streams) that have been defined in the rules files.  

\subsubsection{int moleculizer::getNumberOfSpeciesInSpeciesStream(
  const std::string\& speciesStream) const}

This function returns the number of species that are present in a
particular species stream.  I.e. if a species stream ``Species
containing an A bound to B'' has been defined in the rules file,
calling getNumberOfSpeciesInSpeciesStream(``Species containing an A
bound to B'') will return the number of species matching that criterion.

\subsubsection{void moleculizer::getSpeciesInSpeciesStream(const
  std::string\& streamName, std::vector<const mzr::mzrSpecies*>\&
  speciesVector) const }

This function takes the name of an existing species stream as well as
a reference to a vector of const mzr::mzrSpecies*.  It inserts into
the vector pointers to each of the species generated thus far that
match the conditions of the species stream.

\subsubsection{void moleculizer::getSpeciesInSpeciesStream(const
  std::string\& streamName, std::vector<mzr::mzrSpecies*>\&
  speciesVector) }

This function does the same thing as ``void
moleculizer::getSpeciesInSpeciesStream(const std::string\& streamName,
std::vector<const mzr::mzrSpecies*>\& speciesVector) const'', except
that it does it with a vector of mzrSpecies*, as opposed to const
mzrSpecies*.  Use this if you wish to expand any of the species, as
opposed to just recording information about them.

\subsubsection{const mzrSpecies* moleculizer::getSpeciesWithName(
  const std::string\& speciesName )}
This function takes a unique id and returns a pointer to a const
mzrSpecies that corresponds to that name.  This function will first
look up the name in the list of species and reactions.  If it finds
it, it will return it.  If it does not, it will construct it from
scratch (note that it will do so assuming coherency -- that the mols
that occur within the name also exist in the rules and have the same
definitions).  Therefore, this is the function to be used when dealing
with serialized data.  

\subsubsection{void moleculizer::recordUserNameToGeneratedNamePair(
  const std::string\& userName,  const std::string\& genName )}

This function takes two string parameters.  The first is a user
provided name, the second is the unique id of a species that must
already exist in the network.  This function will allow the species to
thereafter be looked up using the user name.  This function throws a
fnd::NoSuchSpeciesXcpt if the species does not exist.  

\subsubsection{bool moleculizer::nameIsUserName(const std::string\&
  possibleUserName) const}
This function takes a string corresponding to a potential user name,
and returns either true or false depending on whether or not the
string parameter has been recorded as a user-name.  

\subsubsection{std::string
  moleculizer::convertUserNameToGeneratedName(const std::string\&
  possibleUserName) const}

This function takes a string parameter corresponding to a user name
and returns the generated unique id of the corresponding species.
This function throws a mzr::unknownUserNameXcpt exception if the user
name has not been registered.  

\subsubsection{ const mzrSpecies* moleculizer::findSpecies( const
  std::string\& name) const  }

This function takes a string containing a unique id and returns a
corresponding pointer to a const species.  If the species does not exist,
the function throws a fnd::NoSuchSpeciesXcpt exception.  


\subsubsection{ mzrSpecies* moleculizer::findSpecies( const
  std::string\& name) }

This function takes a string containing a unique id and returns a
corresponding pointer to the species.  If the species does not exist,
the function throws a fnd::NoSuchSpeciesXcpt exception.  

\subsubsection{bool
  moleculizer::findReactionWithSubstrates(const mzr::mzrSpecies*
  pSpecies, std::vector<const mzr::mzrReaction*>\& rxnVector) }

This function is the way to find all unary reactions that a particular
species participates in.  Pass in the species of interest by pointer,
as well as a reference to a vector of const mzrReaction pointers.
Moleculizer will insert into that vector the set of all unary
reactions that have the species as their one and only substrate.  The
function returns true if at least one const mzrReaction* is inserted
into the vector and false otherwise -- i.e. the function returns
whether any such reactions were found.  


\subsubsection{bool moleculizer::findReactionWithSubstrates(const
  mzr::mzrSpecies* spec1, const mzr::mzrSpecies* spec2,
  std::vector<const mzr::mzrReaction*>\& rxnVector)}

This function is the way to find all binary reactions that a
particular set of reactions participates in.  To use this function,
pass in pointers to the two species of interest (if you are interested
in finding all reactions of the form A + A -> ? for some species A,
simply pass in the pointer to species A twice) as well as a reference
to a vector of const mzr::mzrReaction pointers.  This function will
insert pointers to each binary reaction that have both spec1 and spec2
as substrates into the vector reference and return true if there are
reactions between these two species and false otherwise.  

\subsubsection{ bool moleculizer::checkSpeciesIsKnown( const
  std::string\& speciesName ) const}
This function takes a string parameter corresponding to a unique id
and returns whether or not this a species with this unique id has been
recorded with moleculizer.

\subsubsection{const SpeciesCatalog\& moleculizer::getSpeciesCatalog()
  const }
Every species that has been generated during the current run of
libmoleculizer is stored in a data structure of type SpeciesCatalog,
which is defined as a std::map<std::string*, mzr::mzrSpecies,
aux::compareByPtrValue<mzrSpecies*> >.  \footnote{For those unfamiliar, the
last parameter of the SpeciesCatalog type list describes how things
are stored and looked up in the map.  By default, maps store and look
up pointers by the value of pointers themselves.  Because
aux::compareByPtrValue<mzrSpecies*> is passed into the map, the
SpeciesCatalog instead looks things up by the dereferenced value of
the pointer.  That is, the map behaves as though it is a
map<std::string, mzr::mzrSpecies>, except that you must pass in
pointers to the strings instead.}.  This datastructure maps pointers
to strings containing unique ids (as opposed to tags, although this
may change in a future release).  

Calling this function returns a const reference to this
data structure.  

\subsubsection{SpeciesCatalog\& moleculizer::getSpeciesCatalog() }

Does the same thing as the non-const version of this function, except
that in this version, the species in the species catalog can be
manipulated -- although the only manipulation you do should be to
expand them.  Anything else will likely result in ill-defined behavior
at best, segfault at worst.  

\subsubsection{const std::list<const mzr::mzrReaction*>\& moleculizer::getReactionList()
  const}

This function returns a const reference to a list containing every
reaction that has been generated during this session of moleculizer.  

\subsubsection{bool moleculizer::recordSpecies( mzr::mzrSpecies*
  pSpecies ) }

Pass a pointer to a mzrSpecies into this function to attempt to
register the species with moleculizer.  If the species has already
been added, nothing happens and the function returns false.  If the
species has not been recorded before, its name is generated and it is
recorded in the map of names to mzrSpecies* (the same one that can be
checked with the moleculizer::getSpeciesCatalog() functions).  The
species is also added to the delta species list.  

This function is probably relatively useless to the user herself,
as rule expansion is typically the procedure that generates species,
rather than direct user intervention (the rule expanding components
use the ``recordXXX'' functions extensively).  But if you have a use
for it, knock yourself out!

\subsubsection{void moleculizer::mustRecordSpecies( mzr::mzrSpecies*
  pSpecies )}

This function does the same thing as moleculizer::recordSpecies,
except that it \em{must} succeed.  If it does not, this function
throws a DuplicatedCatalogEntryXcpt exception.

\subsubsection{bool moleculizer::recordSpecies( mzr::mzrSpecies*
  pSpecies, std::string\& name )}
This function simultaniously records a pSpecies, as in the
moleculizer::recordSpecies(mzr::mzrSpecies* pSpec) function, and
places the generated name for the species into the string reference
parameter (the recorded name is placed into the name parameter no
matter whether the function returns true or false).

\subsubsection{bool moleculizer::recordReaction( mzr::mzrReaction* pRxn
  )}

This function records pRxn by checking it for consistency (ensuring
that it has 0, 1, or 2 substrates -- the function throws a
utl::FatalXcpt exception if the reaction has 3 or greater substrates),
and then records it in theCompleteReactionList (the list of all
reactions that can be accessed with the moleculizer::getReactionList()
function) as well as theDeltaReactionList (accessed with
moleculizer::getDeltaReactionList and cleared with
moleculizer::clearDeltaReactionList).  If the reaction is recorded,
the function returns true, otherwise false. 

Please note: beacuse of the computational difficulties in comparing
reactions, this function ALWAYS succeeds.  The reaction that is passed
in is always recorded, and the function always returns true.
Libmoleculizer is able to make the guarentee that it only creates and
attempts to insert reactions once.  Consequently, no duplicates will
ever be inserted on behalf of libmoleculizer.  However, it is entirely
possible to manually create new reactions and insert them as new
ones.  This will cause problems including having an inaccurate number
of reactions generated as well as having too many reactions
potentially returned for some queries (the right queries will have duplicates).

\subsubsection{void moleculizer::mustRecordReaction( ReactionTypePtr
  pRxn )}
This function does the exact same thing as
moleculizer::recordReaction( mzr::mzrReaction* pRxn) does, except that
it \em{must} succeed.  If it does not, it throws a
DuplicatedCatalogEntryXcpt exception.  Note that this function,
because of compromises made in the name of computational efficiency,
will never fail.  See moleculizer::recordReaction for more details.  

\subsubsection{unsigned int moleculizer::getTotalNumberSpecies()}
This function returns the number of species that have been generated
and registered with moleculizer so far - this is the same as
moleculizer::getSpeciesCatalog().size().

\subsubsection{unsigned int moleculizer::getTotalNumberReactions()
  const}
This function returns the number of reactions that have been generated
and registered with moleculizer so far - this is the same as
moleculizer::getReactionList().size(). 

\subsubsection{const std::list<const mzr::mzrSpecies*>\&
  moleculizer::getDeltaSpeciesList() const}
This function returns the delta species list, which is the list of all
species that have been generated since the last time
moleculizer::resetCurrentState() was called.  

\subsubsection{const std::list<const mzr::mzrReaction*>\&
  moleculizer::getDeltaReactionList() const}
This function returns the delta reaction list, which is the list of all
reactions that have been generated since the last time
moleculizer::resetCurrentState() was called.  

\subsubsection{unsigned int moleculizer::getNumberDeltaReactions()
  const}
This function returns the number of reactions that are in the delta
reaction list (this is the same as
moleculizer::getDeltaReactionList().size()).  Please see chapter
\ref{chap:usingLibmoleculizerChapter} for more information on the
delta reaction and species lists.  

\subsubsection{unsigned int moleculizer::getNumberDeltaSpecies()
  const}
This function returns the number of species that are in the delta
species list (this is the same as
moleculizer::getDeltaSpeciesList().size()).  Please see chapter
\ref{chap:usingLibmoleculizerChapter} for more information on the
delta reaction and species lists.  

\subsubsection{void moleculizer::resetCurrentState()}
This function clears the delta species and delta reaction lists.  

\subsubsection{void moleculizer::incrementNetworkBySpeciesName( const
  SpeciesID\& rName )}
This function looks up the species with the supplied species id and
calls its expandReactionNetwork function, presenting that species to
the network, and generating any 1) unary reactions with that species
as the substrate and all 2) binary reactions involving that species
and any other species that have already been expanded.

\subsection{The mzrSpecies class}

All species in moleculizer are mzr::mzrSpecies, with the vast majority of
those being mzrPlexSpecies.  This section discusses the functions in
use these classes.  \footnote{This list probably isn't an absolutely
  complete reference, merely a reference to any function the authors
  think anyone might ever want to use.  As always, the source code is
  the final arbiter for the complete capabilities of this code.  Check
  src/mzr/mzrSpecies.hh and src/plex/mzrPlexSpecies.hh for the basics
  of this class}

\subsubsection{std::string mzrSpecies::getTag() const}
This function returns a tag to the mzrSpecies, which is a kind of name
that can and should only be used within a single instance of
moleculizer (it has no persistance -- see the section on naming in
chapter \ref{chap:conceptualOverviewChapter} for more information).

\subsubsection{std::string mzrSpecies::getName() const}
This function returns a unique id to the mzrSpecies, which is a kind
of name that uniquely specifies the mzrSpecies in question, even
between different instances of the moleculizer object that created
it.  See the section on naming and unique ids in chapter
\ref{chap:conceptualOverviewChapter} for more information. 

\subsubsection{double mzrSpecies::getWeight() const}
This function returns the weight of the species.  This is typically
done by adding up each of the weights of the mols that make up the
species, along with the masses of any modifications those mols might
possess.  

\subsubsection{void mzrSpecies::expandReactionNetwork()}
This function expands the reaction network around mzrSpecies.  If the
species has not been previously examined, each of its features are
registered with moleculizer and compared with each feature that has
already been registered, according to the rules, with possible
generation of species and reactions.  If this species has already been
expanded, this function does nothing.

The result is that any unary reactions that this species can
participate in are created according to the rules, as well as all
binary reactions that have this species along with another
already-expanded species as substrates.  All these reactions are
registered with moleculizer.  Furthermore, any species that occur as
products to any of these reactions and which are have not yet been
registered are, in an unexpanded state (assuming that the generation
depth has its default value of 1).  See chapter
\ref{chap:conceptualOverviewChapter} for more details on expanding
species to generate new species and reactions.

\subsubsection{void mzrSpecies::expandReactionNetwork(unsigned int i)}
This function expands the reaction network around mzrSpecies, to a
depth of i.  This means that the species is expanded, species and
reactions are potentially generated, and for any reaction that is
generated, each of the product species is called with
expandReactionNetwork( i - 1 ).  If a species either has already been
expanded before, or if i has a value of 0, nothing happens and the
function returns.  

Typically this function is not used.  Instead
mzrSpecies::expandReactionNetwork() calls
mzrSpecies::expandReactionNetwork( DEFAULT\_GENERATION\_DEPTH ) with the
default generation depth.  However, if desired, this function can be
called to temporarily override this behavior (if a more shallow or
more deep expansion around a particular species is preferred).  If the
user wishes to override this behavior permanently, do so by calling
moleculizer::setGenerateDepth.  

\subsection{The mzrReaction class}
\subsubsection{mzrReaction::expandReactionNetwork()}
This function calls mzrSpecies::expandReactionNetwork() on each of the
product species of the reaction.

\subsubsection{bool mzrReaction::hasReactant(const mzrSpecies*
  species) const}
This function returns true if the passed in species is one of the
substrates of this reaction, false otherwise.  

\subsubsection{bool mzrReaction::hasProduct( const speciesType*
  species ) const}
This function returns true if species is one of the products of this
reaction, false otherwise.  

\subsubsection{int mzrReaction::getReactantStochiometry( const
  speciesType* species ) const}

\subsubsection{const std::map<mzrSpecies*, int>\& mzrReaction::getReactants() const}
This function returns a constant reference to a map from mzrSpecies*
to integers, that represents the reactants to the reaction.  Each
of the keys in the map is a pointer to one of the substrates, with its
value being equal to the multiplicity of that substrate in the reaction.

\subsubsection{const std::map<mzrSpecies*, int>\&
  mzrReaction::getProducts() const}
This function returns a constant reference to a map from mzrSpecies*
to integers, that represents the products to the reaction.  Each
of the keys in the map is a pointer to one of the products, with its
value being equal to the multiplicity of that product in the reaction.

\subsubsection{const std::map<mzrSpecies*, int>\&
  mzrReaction::getDeltas() const}
This function returns a constant reference to a map from mzrSpecies*
to integers, that represents both the substrates and the products to the reaction.  Each
of the keys in the map is a pointer to one of the species involved in
the reaction.  Its value is the overall effect to that species
population the reaction has: a positive value means the species
population increases by that value and thus that species is a product
of the reaction; a negative value means the species population
decreases by that value: the species is a substrate of the reaction.

\subsubsection{int mzrReaction::getArity() const}
This returns the number of reactants in the reaction -- the sum of all
the multiplicities of each of the reactions. 

\subsubsection{void mzrReaction::addReactant( mzrSpecies* pSpecies,
  int multiplicity )}

This function will modify the reaction by adding this species with the
given multiplicity to it.  If the species is already a reactant or
product, this function will deal with it appropriately (if it is
already a substrate, this function will add the multiplicity parameter
to the reactants multiplicity chart found within the reaction; if it
is a product, it will be added as a substrate per usual, but the
corresponding entry in the delta chart will have the multiplicity
parameter subtracted from it, and so on).  

If you have a good reason, and know what you are doing, go ahead and
use this function. However, it is hard for us to think of a case in
which this would be appropriate (these functions are typically used by
the reaction generators, which correspond to rules that look for
features and new species to generate new reactions and species) for
you to use this.  But I guess you know best....

\subsubsection{void mzrReaction::addProduct( speciesType* pSpecies,
  int multiplicity )}
This function is basically the same as the mzrReaction::addSpecies
function, except that it works oppositely, in terms of products and
not reactants.  Accordingly, the same warnings appear: unless you have
a really, really good reason to use this function (reasons that we
cannot really imagine as we write this), you really shouldn't.  

\subsubsection{double mzrReaction::getRate() const}
This function returns the rate of the reaction as a double.  If
reaction rate extrapolation has not been enabled, this will have the
same numerical value as the appropriate on or off-rate provided in the
rule that generated this reaction.  If reaction rate extrapolation has
been enabled, it will be equal to that value times the reduced mass of
the substrates.  Please consult chapter
\ref{chap:conceptualOverviewChapter} for more information.

\subsubsection{void mzrReaction::setRate( double newRate )}
Updates the reaction rate to newRate.  You probably shouldn't ever
have to use this, except if you were potentially extending
libmoleculizer.  Pass it a value, and the getRate() function will
return that value thereafter.  

\subsubsection{std::string mzrReaction::getName() const}
This function generates a string that can be used to output this
reaction.  If the reaction has substrates A with multiplicity 1 and B
with multiplicity 2 and a single product B, this string will look
something like ``A + 2 B -> C''.

\subsection{Other global functions for working with mzrSpecies and mzrReactions}
The functions here all help with estimating various spatial parameters
for the mzrSpecies and mzrReactions.  This section expalains each of
them.  

All these functions are located in ``mzr/spatialExtrapolationFunctions.hh''.

\subsubsection{double extrapolateIntrinsicReactionRate(const
  mzr::mzrReaction* pRxn)}

This function calculates the intrinsic reaction rate of the reaction.
This is done by solving the equation $\frac{1}{k} = \frac{1}{k_a} +
\frac{1}{k_D}$, where $k$ is the overall reaction rate, $k_a$ is the
intrinsic reaction rate, and $k_D$ is the diffusion rate, defined as
$4 * \pi * (r_1 + r_2) * (D_1 + D_2)$, where $r_i$ is the radius of
the $i^{th}$ species, estimated by assuming the species is roughly
globular and has density equal to 1.22 $\frac{g}{cm^3}$, and $D_i$ is
the diffusion coefficient of the species.  See the entry for
getDiffusionCoeffFromSpecies to see how this is estimated.
    
\subsubsection{double extrapolateMolecularRadius( const
  mzr::mzrSpecies* pSpecies)}
This function returns an estimate of the molecular radius of the
species in meters.  This is done by calculating volume by assuming
average protein density, $1.22 \frac{g}{cm^3}$ and then solving for
radius by assuming the protein is globular ($V = \frac{4}{3}\pi r^3$).

\subsubsection{double extrapolateMolecularRadius(const double\& mass)}
This function returns an estimate for the radius of a protein with the
given mass.  It works exactly the same as other version of the
function: extrapolateMolecularRadius( const mzr::mzrSpecies*
pSpecies), except that it takes masses instead of species.

\subsubsection{double getDiffusionCoeffForSpecies( const
  mzr::mzrSpecies* pSpecies)}
This function checks the value of the cutoff that determines the
estimation as to whether a species is a small mol or a protein species.
By default, objects which weigh more than 1500 daltons are treated as
protein and those which weight less are treated as small mols -- this
cuttoff value can be inspected and set via the
getSmallMolProteinCutoff and setSmallMolProteinCutoff functions.  If
it is determined to be a small mol, a diffusion coeff of value
getSmallMolDiffusionCoeff() is returned; if it is a protein, a
diffusion coeff of value getProteinDiffusionCoeff() is returned.  See
any of this functions along with the setSmallMolDiffusionCoeff and
setProteinSmallMolDiffusionCoeff functions for more information.

\subsubsection{double getSmallMolProteinCutoff()}
This function returns the current value of the cuttoff that is used to
determine whether a species is a small mol or a protein in the
getDiffusionCoeffForSpecies function. 

\subsubsection{void setSmallMolProteinCutoff(const double\& cut)}
This function sets a new value of the cutoff that is used to determine
whether a species is a small mol or a protein in the
getDiffusionCoeffForSpecies function. 

\subsubsection{double getProteinDiffusionCoeff()}
This function returns the diffusion coeff for proteins.  By default it
is set to $3.0\frac{\mu m}{cm^2}$, but can be set by using the
setProteinDiffusionCoeff function.

\subsubsection{double getSmallMolDiffusionCoeff()}
This function returns the diffusion coeff for small molecules.  By default it
is set to $100.0\frac{\mu m}{cm^2}$, but can be set by using the
setSmallMolDiffusionCoeff function.

\subsubsection{void setProteinDiffusionCoeff(double rate)}
This function sets the value - by default set to $3.0\frac{\mu
  m}{cm^2}$ that is returned by the getProteinDiffusionCoeff function.

\subsubsection{void setSmallMolDiffusionCoeff(double rate)}

This function sets the value - by default set to $100.0\frac{\mu
  m}{cm^2}$ - that is returned by the getSmallMolDiffusionCoeff function.

\section{The C-Interface API}
This portion of the chapter is devoted to discussing all the functions
that exist in the C-Interface API.  For an introduction to how to use
this, please consult the relevant section in chapter
\ref{chap:interfacesChapter}.  As always, the best possible
``reference'' to the code is the code itself.  This can be found in
``mzr/libmzr\_c\_interface.h'' and ``mzr/libmzr\_c\_interface.cc''.

\subsubsection{moleculizer* createNewMoleculizerObject()}
This function creates a new moleculizer object and returns a pointer
to it.  This pointer is owned by the caller and must be freed using
the freeMoleculizerObject function, or else the memory is lost and
leaked.  This pointer should be included as the paramter to all the
other functions in this interface.    

\subsubsection{int setRateExtrapolation( moleculizer* handle, int
  extrapolation)}
Call this function prior to calling one of the functions for loading a
model to set reaction rate extrapolation on the moleculizer object (if
extrapolation is != 0) or to turn it off (if extrapolation == 0) --
extrapolation is disabled by default.

If reaction rate extrapolation is enabled, when a binary reaction is
created, its rate is set as the rate in the rule which created it multiplied
by the reduced mass of the products.  If reaction rate extrapolation
is disabled, the rate provided in the rule is simply passed along as
the rate of the reaction as is.  Please consult chapter
\ref{chap:conceptualOverviewChapter} for more details as to what this
is and why it might be used.

This function returns error codes: 0 for success, 1 to indicate an
unknown error, and 2 to indicate that this function was incorrectly
called on a moleculizer object that already has had rules loaded into
it.  

\subsubsection{void freeMoleculizerObject( moleculizer* handle)}
Call this function with a moleculizer* that has been created by the
createNewMoleculizerObject to free it.  This is the only way to
properly release the memory held by the pointer.

\subsubsection{int expandNetwork( moleculizer* handle)}
Call this function on a moleculizer object that has had some rules
loaded into it to have that network completely expanded out.
\textbf{Beware!!!}  This function may take a very long time to run, if
  it returns at all.  

This function returns the following error codes: 0 for success, 1 to
indicate an unknown error, and 2 to indicate that the moleculizer
object has not had rules loaded into it.  

\subsubsection{int incrementSpecies( moleculizer* handle, char*
  speciesName)}

This function takes a moleculizer object, and a null-terminated char*
containing the name of a species.  It finds the corresponding species
and expands it - any unary reactions that species can participate in
will be created as well as any binary reactions that involve the
parameter species along with as any other already expanded species, along
with all product species of those reactions that have not yet been
registered.  

This function returns the following error codes: 0 for success, 1 to
indicate an unknown error, 2 indicates the species name passed in is
not listed in the species within the reaction network.

\subsubsection{int loadRulesFile(moleculizer* handle, char* fileName)}
This function takes a moleculizer object as well as a null-terminated
string that contains a file name (which in unix is the entire relative
path -- values like ``models/simple/simple-rules.mzr'' or
``~/models/my-model.mzr'' or ``the-model.mzr'' are all perfectly
valid).

This function returns the following error codes: 0 for success, 1 to
indicate an unknown error, 2 to indicate the file was unparsable, 3 to
indicate that rules have already been loaded here, 4 to indicate that
the file was not found.

\subsubsection{int loadRulesString( moleculizer* handle, char* file)}
This function takes a moleculizer object as well as a null-terminated
string that contains the contents of a rules file.  The result of the
file is to load the information into file into the moleculizer object.

This function returns the following error codes: 0 for success, 1 to
indicate an unknown error, 2 to indicate the document was unparsable,
3 to indicate that rules have already been loaded.

\subsubsection{int convertTagToID( moleculizer* handle, char*
  speciesTag, char* speciesID, int idSize)}
The purpose of this function is to take a tagged name of a complex
species and convert it to a unique id.

This function takes as its first two parameters a moleculizer pointer
as well as a null-terminated string that contains a tagged name of a
species in the moleculizer object; the result of this function is to
find the corresponding unique id of the species possessing that tagged
name, and insert it into a charecter buffer: passed in as the third
parameter, with the 4th being its size.

This function returns the following error codes: 0 for success, 1 to
indicate an unknown error, 2 to indicate that the tag was not found in
moleculizer, 3 to indicate that the supplied buffer does not have
enough memory to contain the resulting name.

\subsubsection{int convertIDToTag( moleculizer* handle, char*
  speciesID, char* speciesTag, int tagSize)}
This function converts a unique id to a tagged name.  To do this, pass
in a moleculizer object, the unique id in question, along with a
charector buffer and a parameter describing the size of that buffer.
The interface will perform the conversion and insert the result into
the speciesTag charector buffer.

\textbf{Note:} Because of the nature of tagged names, virtually all
systems will need no more than 9 charecters (length of 8 plus a null
terminator) and no system should need more than 17 (16 plus a
terminator) to hold the tag.  Furthermore, for any system, all tagged
names will have constant size on that system. 

This function returns the following error codes: 0 for success, 1 to
indicate an unknown error, 2 to indicate the unique id was not found
in moleculizer, and 4 to indicate (if you get this you're probably
doing it wrong) that the provided buffer is not large enough to hold
the tagged name.

\subsubsection{int getReactionsBetween(moleculizer* handle, char*
  speciesName1, char* speciesName2, reaction*** ptrReactionPtrArray,
  int* numReactions)}

This function finds all binary reactions between two speciesand returns them to the user.

To use this function, create a reaction pointer array and an integer,
and pass their addresses into the function, along with a
null-terminated string containing the name of the species in
interest.  

This function returns the following error codes: 0 for success, 1 to
indicate an unknown error, 2 to indicate the species name that was
passed in could not be found.



\subsubsection{int getUnaryReactions(moleculizer* handle, char*
  speciesName, reaction*** ptrReactionPtrArray, int* numReactions)}

This function finds all unary reactions involving a particular
species and returns them to the user.

To use this function, create a reaction pointer array and an integer,
and pass their addresses into the function, along with a
null-terminated string containing the name of the species in
interest.  The result is that moleculizer will place into
*ptrReactionPtrArray an array of reaction*s that represent each of
these reactions, and into *pNumReactions the number of elements in the array.

The reaction array that is placed into the reaction pointer array is
owned by the caller of the function and must be freed using the
freeReactionArray function in this interface.

This function returns the following error codes: 0 for success, 1 to
indicate an unknown error, 2 to indicate the species name that was
passed in could not be found.

\subsubsection{int getReactionsInvolving(moleculizer* handle, char*
  speciesName, reaction*** ptrReactionPtrArray, int* pNumReactions)}

This function returns all reactions that involve a 


\subsubsection{int getNumberOfSpecies(moleculizer* handle)}
This function returns the number of species that the moleculizer
handle that has been passed in has recorded thus far in the
simulation.

\subsubsection{int getNumberOfReactions(moleculizer* handle)}
This function returns the number of reactions that the moleculizer
handle that has been passed in has recorded so far in network expansion.

\subsubsection{int getAllStreamSpecies(moleculizer* handle, char*
  streamName, species** pSpeciesArray, int* numberSpecies)}


\subsubsection{int getAllSpecies(moleculizer* handle, species***
  pSpeciesArray, int* numberSpecies)}

This function returns an array containing species*'s to each of the
species that have been recorded thus far during the course of
moleuclizer's network expansion.  

It works the same way as most other functions in this interface:
define a species** object as well as an int, and pass in their
addresses into the function.  This function will make the species**
point to a array containing freshly allocated pointers to each species
in libmoleculizer, and will put the size into the integer parameter.
This array is owned by the user, and should be freed using the
freeSpeciesArrays functions.

This function returns the following error codes: 0 for success, 1 to
indicate an unknown error.

\subsubsection{int getAllReactions(moleculizer* handle, reaction***
  pReactionArray, int* numberReactions)}

This function returns an array of pointers to all the reactions that
have been generated thus far in moleculizer's network expansion.  


\subsubsection{int getAllExteriorSpecies(moleculizer* handle,
  species*** pSpeciesArray, int* numberSpecies)}

This function returns an array of all species that have not yet been
expanded.  Create a species** (array of species pointers) and an int
that will hold the number of elements in that array.  Pass in a
moleculizer object, the addresses of the species pointer array and the
integer value.  Moleculizer will insert into the array each of the
species that is have been recorded, but not yet expanded, and place
into the integer the number of elements that have been inserted.  

Each of these species pointed to by elements of the array are owned by
the caller and should be freed using the freeSpeciesArray function.

This function returns the following error codes: 0 for success, 1 to
indicate an unknown error.  

\subsubsection{void freeReactionArray( reaction** pRxnArray, unsigned
  int numArrayElements)}

Use this function to free an array of reactions, such as the arrays
returned via the getAllReactions, the getReactionsBetween, and the
getReactionsInvolving functions.  Pass in the array as well as the
number of elements in the array and they will all be freed.

\subsubsection{void freeSpeciesArray( species** pSpeciesArray,
  unsigned int numArrayElements)}

Use this function to free an array of species, such as the array
returned by the getAllSpecies or getAllStreamSpecies functions.  Pass
in the array of species pointers as well as the number of elements in
the array and they will be freed.


\subsubsection{void freeReaction( reaction* pRxn )}
This function will free a pointer to a reaction.  Simply pass the
pointer in to this function and it will be freed.

\subsubsection{void freeSpecies( species* pSpecies )}
This function will free a pointer to a species.  Simply pass the
pointer into this function and it will be freed.

\subsubsection{int convertUserNameToSpeciesID(moleculizer* handle,
  char* theUserName, char* correspondingSpeciesID, int bufferSize)}


\subsubsection{int convertUserNameToSpeciesTag(moleculizer* handle,
  char* theUserName, char* correspondingTag, int bufferSize)}


\subsubsection{int expandSpecies( moleculizer* handle, species*
  mzrSpecies)}
This function will expand the reaction network around a particular
species.  That is, calling this function with a species pointer will
cause several things to occur.  1) if this species has been expanded
before, either using this function directly or by having had a
reaction that includes this species as a product previously expanded.
2) if this species has not been expanded, all unary reactions
involving this species as a substrate will be created and all binary
reactions that can occur in the rules between this species and all
other already registered and already expanded species will be created;
furthermore, any unknown products of any of those reactions will be
registered as new, unexpanded species with moleculizer.  

This function returns the following error codes: 0 for success, 1 to
indicate an unknown error.


\subsubsection{int expandSpeciesByTag( moleculizer* handle, char*
  theTag)}


\subsubsection{int expandSpeciesByID( moleculizer* handle, char*
  theID)}

\subsubsection{int getDeltaSpecies( moleculizer* handle, species***
  pSpeciesArray, int* pNum)}
This function returns a list of the delta species -- the list of all
species that have been created and registered with libmoleculizer
since the last time the clearDeltaState function was called.  

Pass it a moleculizer variable and the addresses of a species** and
integer variables.  This function will setup *pSpeciesArray such that
it contains an array of *pNum species pointers; this array is the list
of species that have been produced since the last time clearDeltaState
was called.

As always, this array of species pointers is owned by the caller, and
should be freed using the freeSpeciesArray function lest the memory is
lost.

This function returns the following error codes: 0 for success, 1 to
indicate an unknown error.

\subsubsection{int getDeltaReactions( moleculizer* handle, reaction***
  pReactionArray, int* pNum)}
This function returns a list of the delta reaction -- the list of all
reactions that have been created and registered with libmoleculizer
since the last time the clearDeltaState function was called.

Pass it a moleculizer variable and the addresses of reaction** and
integer variables.  This function will setup *pReactionArray such that
it contains an array of *pNum reaction pointers; this array is the list
of reactions that have been produced since the last time clearDeltaState
was called.

As always, this array of reaction pointers is owned by the caller, and
should be freed using the freeReactionArray function lest the memory is
lost.

This function returns the following error codes: 0 for success, 1 to
indicate an unknown error.

\subsubsection{int clearDeltaState( moleculizer* handle)}
This function clears the delta state of the moleculizer object.  

This function returns the following error codes: 0 for success, 1 to
indicate an unknown error.

\subsubsection{int expandReaction(moleculizer* handle, reaction*
  mzrReaction)}
This function will expand the reaction network around a particular
reaction -- this is equivalent to expanding the network around each of
the products of this reaction.  

Use this function by (surprise, surprise!) passing in pointers to a
moleculizer object as well as a pointer to a reaction.  

This function returns the following error codes: 0 for success, 1 to
indicate an unknown error.

% \section{The Python API} 