\chapter{MZR Format Reference}
\label{chap:mzrReference}

This chapter serves as a complete reference to the MZR file format
specification.  

To understand it, at least a basic knowledge of the Relax NG format
for specifying xml schemas will be assumed. (Never fear!  Even if the
format is unknown to you, dear reader, it is fairly straightforward
and can be largely understood without difficulties!)

\section{/moleculizer-input}

This mandatory root element of a moleculizer model contains a single
model subelement.

\begin{ExampleRNG}
<?xml version="1.0" encoding="UTF-8"?>
<grammar
    xmlns="http://relaxng.org/ns/structure/1.0"
    datatypeLibrary="http://www.w3.org/2001/XMLSchema-datatypes">
  <start>
    <element name="moleculizer-input">
      <element name="model">
	<ref name="model-content" />
      </element>
    </element>
  </start>
</grammar>
\end{ExampleRNG}

\section{/moleculizer/model}

The model element contains all the biology in the model.  It consists
of 

\begin{ExampleRNG}
<element name="model">
  <element name="modifications">
    <ref name="modifications-content" />
  </element>
  <element name="mols">
    <ref name="mols-content" />
  </element>
  <element name="allosteric-plexes">
    <ref name="allosteric-plexes-content" />
  </element>
  <element name="">
    <ref name="" />
  </element>
  <element name="">
    <ref name="" />
  </element>
  <element name="">
    <ref name="" />
  </element>
  <element name="">
    <ref name="" />
  </element>
</element>
\end{ExampleRNG}





\section{A Complete Specification of MZR Format in RelaxNG Format}

