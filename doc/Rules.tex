\chapter{The MZR Language for expressing rule-based models}
\label{chap:theRulesChapter}
A libmoleculizer input file (Moleculizer Rule File, or MZR
file) is an xml document in a particular syntax, representing a set
of molecule and species definitions as well as a set of reaction rule
descriptions.

This chapter talks about MZR file syntax, and discusses how to express
sets of rules representing a variety of biological pathways within it.

\section{MZR File Overview}
A MZR file consists of a 'moleculizer-input' root tag that contains a
single 'model' element.  The model element contains several mandatory
subelements, which contain all of the description of the model:
'modifications', 'mols', 'allosteric-plexes', 'allosteric-omnis',
'reaction-gens', 'explicit-species', and 'explicit-reactions'. 

\begin{ExampleXML}[caption=Overall structure of a MZR file, label=overallmzrstruct]
<!-- Overall structure of a MZR file -->
<moleculizer-input>
  <model>
    <modifications />
    <mols />
    <allosteric-plexes />
    <allosteric-omnis />
    <reaction-gens />
    <explicit-species />
    <explicit-reactions />
  </model>
</moleculizer-input>
\end{ExampleXML}

Generally speaking, the information in the modifications section
describes and defines each of the legal modification values, typically
different chemical groups, that can be used within the model.  

The elements in the 'mols' section correspond to molecule definitions for
the different molecules that can be used within the model.

Allosteric-plexes and allosteric-omnis both contain information that
describes how specific binding sites in a complex change their shape
relative to features of the containing complex (such as specific
binding sites being bound, specific regulatory sites being modified in
specific ways, presence of specific mols in complex, or other
conditions).  

The reaction-gens element contains each of the different reaction
rules in the model; each of these corresponds to a single biochemical
interaction between different combinations of functional and
regulatory sites.  Typically these will correspond to dimerization
reactions, decomposition reactions, and more complicated enzymatic
reactions. 
 
The explicit-species section is where complexes can be specified and
given names, names that can be used later in reactions or within
libmoleculizer to refer to those complexes.  The explicit-reactions
section containts so called explicit reactions, reactions where all
substrates, all products, and all kinetics are explicity stated.  

For more information on the any section, including the all-important
question of how to write it, please consult the appropriate section in
this chyapter.  For further information, please check out the chapter
'API Reference'.  

\section{The 'modifications' Section -
  /moleculizer-input/model/modifications } 
\label{sec:modifications}
\index{modifications}

The modifications section consists of zero or more modification elements,
each of which represents a particular modification that may be present
on a protein, e.g. 'none', 'phosphorylated', 'ubiquitinated', at one
of its modification sites.

Each modification must have a unique name attribute and a weight-delta
subelement, representing the weight that that modification adds to the
basic molecular weight of the mol that it appears on.  

\lstset{language=XML}
\begin{lstlisting}[caption=A Sample Modifications Section, label=modifications ]
<modifications>
  <modification name="none">
    <weight-delta daltons="0.0" />
  </modification>
  <modification name="phosphorylated">
    <weight-delta daltons="42.0" />
  </modification>
  <modification name="doubly-phosphorylated">
    <weight-delta daltons="84.0" />
  </modification>
</modifications>
\end{lstlisting}

Each of the modifications that will be used and referred to in the
model must be defined here. 

\section{The 'mols' section - /moleculizer-input/model/mols}

Complex species in libmoleculizer are combinations of molecules bound
together between different binding sites, along with modifications
associated with each modification site for each molecule in the complex.  

In the mols section, each of the molecule types that can occur in
complexes within the model are specified.

Libmoleculizer supports two kinds of molecules in its networks:
structural molecules, typically representing proteins, that have some
number of binding and modification sites, and small molecules, which
represent small particles that participate in reactions amongst
complex species, but do not have any structure themselves.

These two kinds of molecules correspond to the two kinds of
subelements that may go in the mols section: 'mod-mol' elements and
'small-mol' elements.   

\subsection{Defining 'mod-mol' elements}
Each mod-mol element defines a single protein type in the reaction 
network.  A mod-mol is defined with a unique name, a weight, and any number
of binding and modification sites. Mod-mols and their structural
components (binding sites, modification sites) are the basic entities 
referred to by reaction rules in the reaction-gens section of the MZR
file, and are the basic element of multi-protein complexes.

A single mod-mol element corresponds to a single protein definition
and has a mandatory name attribute that corresponds to the protein's
id.  This name must be unique amongst mols in the model.  

A mod mol element has a mandatory weight subelement with a mandatory
'daltons' attribute, which takes the default, unmodified weight of
that mod-mol.

\lstset{language=XML}
\begin{lstlisting}[caption=Basic mod-mol structure, label=basicmodmolstructure ]

<!-- Example of a mod-mol definition with mandatory name attribute -->
<!-- and mandatory weight subelement -->
<mod-mol name="Gpa1">
  <weight daltons="54075.9"/>
  <!-- binding-site and mod-site elements -->
</mod-mol>

\end{lstlisting}

Finally a mod-mol element has zero or more binding-site subelements
and zero or more mod-site subelements.  

\subsubsection{Defining 'binding-site' elements}
Mod-mols can have one or more binding sites, each one described by a
binding-site subelement, that represents a location of iteraction
where the mod-mol can complex with other proteins.  

Each binding site is associated with a list of legal binding
site states, called 'shapes' in libmoleculizer, an idea that
encompasses binding site conformation as well as other physical
charectoristics, such as occlusion, to binding-sites.  The use of
binding site shapes, particularly to express allosteric protein
interaction behavior and reaction kinetics is a critical property of
the way libmoleculizer works, and must be understood.  

Each binding site element must be given a mandatory name attribute
that must be unique among all the parent mod-mol's binding sites. Each
site must also have one or more site-shape subelements (at least one must be
specified in order to specify the default site-shape for this binding
site), representing the different conformational shapes the binding
site may possess at any given time.  Finally, a single subelement
'default-shape-ref' with a mandatory name attribute must be present,
which must take as its value the name of some site-shape defined for
the same binding site.

\lstset{language=XML}
\begin{lstlisting}[caption=Defining binding sites in a mod-mol,
  label=bindingsitedefinitions,
  keywords=]
<mod-mol name='Gpa1'>
  <weight daltons='101.1' />
  <binding-site name="to-GXP">
    <site-shape name="default"/>
    <default-shape-ref name="default" />
  </binding-site>
  <binding-site name="to-Ste2">
    <!-- The ordering of site-shapes and default-shape-ref can come -->
    <!-- in any order.  -->
    <default-shape-ref name="default" />
    <site-shape name="default" />
    <site-shape name="occluded" />
  </binding-site>
  <binding-site name="to-Ste4">
    <default-shape-ref name="inactive" />
    <site-shape name="inactive" />
    <site-shape name="partially-active" />
    <site-shape name="fully-active" />
  </binding-site>
</mod-mol>
\end{lstlisting}


\subsubsection{'A note on the binding-shape state constraint of
  libmoleculizer'}
As mentioned elsewhere in this manual, one of the primary modeling
metaphors used by libmoleculizer is to enforce each binding site in
complex has a site-specific shape at all times and that the specific
shapes of interacting components determine allosteric kinetics.  

For instance, suppose two mod-mols: Fus3 and Ste5 are defined in the
mols section of the MZR rules file.  Fus3 has one binding site
'to-Ste5' that has only one state possible: 'default'.  Ste5 has two
binding sites, 'X' and 'to-Fus3', and 'to-Fus3' can be in either a
'default' state or an 'occluded' state.  This information is specified
in Fus3 and Ste5's mod-mol element definitions.

In an appropriate place (either the allosteric-plexes or
allosteric-omnis sections of the model, or in an allosteric-state
subelement of the mod-mol itself -- please see documentation on these
sections elsewhere in the chapter for more information), conditions
under which different binding sites obtain different shapes can be
specified.  For instance, in the previous example, a rule that states
that when Ste5 is complexed with a mod-mol 'Ste11' at its 'X' site,
then its 'to-Fus3' site has the shape 'occluded' could be added to the
allosteric-omnis section of the model.

Finally, in the specific reaction-gen (reaction rule) they apply to,
reaction kinetics are specified for each grouping of relevant of
binding site-states (for instance, in a dimerization reaction, a
reaction rate must be given for each pair of binding-site shapes that
can occur in the two participating binding sites).  

\subsubsection{'mod-site'}
A mod-site element represents single modification site on the mod-mol;
i.e. a site where the protein can undergo post-translational 
modifications as a part of the reaction network.  

Each mod-site-element has a mandatory 'name' attribute, and a
mantatory default-mod-ref subelement, which itselfhas a mandatory 'name' attribute.
The default-mod-ref name must be identical to one of the modification elements in
the modifications section.

\lstset{language=XML}
\begin{lstlisting}[caption=Defining a modification site in a mod-mol, label=modsiteexample]
<modifications>
  <modification name="none">
    <weight-delta daltons="0.0"/>
  </modification>
  <modification name="phosphorylated">
    <weight-delta daltons="42.0"/>
  </modification>
</modifications>

<mols>
  <mod-mol name="Gpa1">
    <mod-site name="phosphorylation-site-1">
      <default-mod-ref name="none"/>
    </mod-site>
    <mod-site name="phosphorylation-site-2">
      <default-mod-ref name="none"/>
    </mod-site>
  </mod-mol>
</mols>
\end{lstlisting}

\subsubsection{'allosteric-state'}
A mod-mol may contain zero or more 'allosteric-state' elements.  Each
of these elements describes a mapping of modification sites to
modification values that, when realized, induces a particular set of
binding-sites to be in particular binding site shapes.  <<<DOES THIS
MAPPING OF MODIFICATION SITES HAVE TO BE COMPLETE?  I THINK IT DOES
NOT...>>> For example, assume that we have defined a protein Fic2 with
a binding site 'func-site-1' that can have a shape 'active' or
'inactive', but is in default shape 'inactive'.  Suppose then that
Fic2 also has 2 regulatory sites 'phosphorylation-site-1' and
'phosphorylation-site-2', and suppose we would like to represent that
when both these sites are phosphorylated site 'func-site-1' becomes
active.  We could represent this using an 'allosteric-state' element.

Each allosteric state element has two mandatory elements, a mod-map
subelement and a site-shape-map subelement.  The mod-map element
consists of one or more 'mode-site-ref' elements, each with a name
attribute that names a mod-site on the mod-mol along with a single
'mod-ref-name' subelement, whose name attribute is the name of a
particular modification value.  

The second mandatory subelement of allosteric state is an element
called 'site-shape-map', which consists of one or more subelements
that each corresponds to a shape change to a binding site in that
mol.  Specifically, each subelement is a 'binding-site-ref' element,
with a name attribute that corresponds to the name of a binding site
in that mod-mol.  Each binding site-ref element contains a single
subelement called 'site-shape-ref', with a name that corresponds to
one of the predefined sites shapes for that site.  

For instance, suppose we have mod-mol called 'Fic3' defined as having
a binding site 'site-1' which by default has shape 'inactive' but
which is also defined as having an active shape 'active' and a single
modification site 'phos-site' with a default modification value of
'none'.  Suppose we would like to express that 'site-1' takes the
shape 'active' when 'phos-site' has a 'phosphorylation' modification
attached to it.  We would do this by adding an allosteric-state
element to this mod-mol.

\lstset{language=XML}
\begin{lstlisting}[caption=An example of an allosteric-state element, label=allostericstateexample]
<allosteric-state>
  <mod-map>
    <mod-site-ref name='phos-site'>
      <mod-ref name='phosphorylated' />
    </mod-site-ref>
  </mod-map>
  <site-shape-map>
    <binding-site-ref name='site-1'>
      <site-shape-ref name='active' />
    </binding-site-ref>
  </site-shape-map>
</allosteric-state>
\end{lstlisting}

If additional constrains need to be checked, these are specified as
additional mod-site-ref subelements of mod map.  Each of the
modification site conditions described in a mod-map must be satisfied
for the binding-site changes described in the site-shape-map to be
appled.  If additional binding shape changes need to be added for a
particular set of modification values, each additional change is added
as another binding-site-ref child element of site-shape-map.  Finally,
if an entirely different allosteric condition needs to be expressed,
it is added as an additional allosteric-state element.  In conclusion,
each mod-mol may be defined with any number of allosteric-state
elements.  Each allosteric-state of a mod-mol corresponds to a set of
modification states that when realized, induce a set of binding-site
shape changes.

\subsubsection{A More Extensive Mod-mol example.'}

To end this section, it may be helpful to define a basic protein in
its entirety using mod-mol syntax.

To do this, let us model the hypothetical protein Fic1.  In a
literature review, we see that it binds to a scafold protein where it
acts as the target of a kinase cascade.  When phosphorylated, Fic1
will dissociate from the scaffold and go and bind to a target.  IT is
found based on the literature review that Fic1 does not activate the
target in its off space, and that single phosphorylation partially
activates the pathway, although at lower levels then it does in its
doubly phosphorylated state.  

To represent this protein as a mod-mol, we must give it a name and
weight, a list of binding sites, and a list of modification sites.  We
must also define the various shapes that can be associated with each
binding in order to induce allosteric effects.  Here is an example of
how such a mod-mol could be defined.

\lstset{language=XML}
\begin{lstlisting}[caption=A complete mod-mol example, label=completemodmolexample]
<mod-mol name='Fic1'>
  <weight daltons='413.7' />
  <binding-site name='to-scaffold'>
    <site-shape name='default' />
    <site-shape name='half-inactive' />
    <site-shape name='inactive' />
    <default-shape-ref name='default' />
  </binding-site>
  <binding-site name='to-target'>
    <site-shape name='inactive' />
    <site-shape name='partially-active' />
    <site-shape name='active' />
  </binding-site>
  <mod-site name='phos-site-1'>
    <default-mod-ref name="none" />
  </mod-site>
  <mod-site name='phos-site-2'>
    <default-mod-ref name="none" />
  </mod-site>

  <!-- This represents the allosteric state when phos-site-1 -->
  <!-- is phosphorylated and phos-site-2 is not.             -->
  <allosteric-state>
    <mod-map>
      <mod-site-ref name='phos-site-1'>
	<mod-ref name='phosphorylated' />
      </mod-site-ref>
    </mod-map>
    <site-shape-map>
      <binding-site-ref name='to-scaffold'>
	<site-shape-ref name='half-inactive' />
      </binding-site-ref>
      <binding-site-ref name='to-target'>
	<site-shape-ref name='paritally-inactive' />
      </binding-site-ref>
    </site-shape-map>
  </allosteric-state>

  <!-- This represents the allosteric state when phos-site-1 -->
  <!-- is phosphorylated and phos-site-2 is not.             -->
  <allosteric-state>
    <mod-map>
      <mod-site-ref name='phos-site-2'>
	<mod-ref name='phosphorylated' />
      </mod-site-ref>
    </mod-map>
    <site-shape-map>
      <binding-site-ref name='to-scaffold'>
	<site-shape-ref name='half-inactive' />
      </binding-site-ref>
      <binding-site-ref name='to-target'>
	<site-shape-ref name='paritally-inactive' />
      </binding-site-ref>
    </site-shape-map>
  </allosteric-state>

  <!-- This represents the allosteric state both phos-sites  -->
  <!-- are phosphorylated. -->
  <allosteric-state>
    <mod-map>
      <mod-site-ref name='phos-site-1'>
	<mod-ref name='phosphorylated'>
      </mod-site-ref>
      <mod-site-ref name='phos-site-2'>
	<mod-ref name='phosphorylated'>
      </mod-site-ref>
    </mod-map>
    <site-shape-map>
      <binding-site-ref name='to-scaffold'>
	<site-shape-ref name='inactive' />
      </binding-site-ref>
      <binding-site-ref name='to-target'>
	<site-shape-ref name='active' />
      </binding-site-ref>
    </site-shape-map>
  </allosteric-state>
  
</mod-mol>
\end{lstlisting}

\subsection{'small-mol'}
A small-mol element represents a small molecule - typically an organic
molecule that participates in pathway reactions.  ADP/ATP are perhaps
the most common examples.

The most defining semantic charecteristic of the small molecule as
well as the small-mol, is a lack of structural information.  Small 
molecules participate in chemical reactions in toto.  They do not have
binding sites, nor do they have modification sites. 

The definition for a small-mol element is a mandatory name attibute
and a mandatory weight child with a mandatory daltons attribute.

\lstset{language=XML}
\begin{lstlisting}[caption=Examples of small-mol definitions, label=smallmolex]
<mols>
  <small-mol name="ADP">
    <weight daltons="427.20"/>
  </small-mol>
  <small-mol name="ATP">
    <weight daltons="507.181"/>
  </small-mol>
</mols>
\end{lstlisting}

\section{Defining 'allosteric-plex' elements}

This section is used to specify allostery in a similar method to
allosteric-omnis (see next section).  For now, it should probably be
ignored, as it has been virtually entirely subsumed by
allosteric-omnis (I think).  

\section{Defining 'allosteric-omni' elements}

In the mod-mol section, we learned how we could easily specify
allosteric changes to binding site shape using allosteric-state
elements, which tell moleculizer how binding-site states on a mod-mol
may be changed by the states of different modification sites on that
same mod-mol.  This is well and good, but how can we specify more
complicated types of allostery.

For instance, Cdk proteins typically have multiple regulatory sites
and multiple binding sites where the activation of a functional site
depends on multiple regulatory sites being phosphorylated, as well as
a specific binding site being bound to a cyclin.  For the functional
site to be activated, all three - both regulatory sites being
phosphorylated and the appropriate binding site must be bound to
cyclin.  Because of the third binding condition, this type of binding
site shape change cannot be listed in relation to a specific molecule
in a specific regulatory state; instead it must be specified relative
to a complex (in this case a Cdk-cyclin dimer) complex.  This is where
allosteric-omnis come in.

Each allosteric-omni subelement of allosteric-omnis represents a
generalized form of binding-site shape allostery.  

An allosteric-omni element has two mandatory subelements, a plex
subelement and a allosteric-sites subelement.

The plex element describes a subcomplex that, if found present in a
complex species, will induce allosteric shape changes to binding sites
within that subcomplex.  

\subsection{Defining the plex element}

\subsubsection{Defining the allosteric-sites element}



\section{'reaction-gen'}

The reaction-gens section is where all of the rules for extrapolating
reactions go.  Currently, there are two
reaction-generators: dimerization-gen\glossary{name={modification},
  description={A modification is a changable part of the mols.}}, which represents a
protein-protein binding interaction as well its reverse decomposition
reaction, and the omni-gen, which represents a generic reaction that
can express unary and binary enzymatic reactions and other reaction
forms.  

\section{'dimerization-gen'}
A dimerization-gen reaction generator represents a generic
dimerization/decomposition reaction between a pair of modular binding
domains belonging to two proteins.  

The first information needed by a dimerization-gen is two mol-ref
elements, each with a site-ref subelement, which each specify one of
the binding sites that participates in this dimerization reaction.
For instance, if the model has defined (in the mod-mol section) a
protein named 'Substrate' that has a binding site 'to-Enzyme', the
proper way to specify this binding site is by writing 

\begin{lstlisting}
<mol-ref name=''Substrate''>
  <site-ref name="to-Enzyme" />
</mol-ref>
\end{lstlisting}

Following two of these elements, two elements: default-on-rate-value
and default-off-rate value must be provided.  Each must have a value
element, which corresponds to the basic reaction rate.  Units of this
are (Hz)(l)/mol.  See /ref{simpleDimerizationExample} for a basic
example of a dimerization reaction between two proteins.

\begin{lstlisting}
<dimerization-gen>
  <mol-ref name="Substrate">
    <site-ref name="to-Enzyme" />
  </mol-ref>
  <mol-ref name="Enzyme">
    <site-ref name="to-Substrate" />
  </mol-ref>
  <default-on-rate value="1.0e12" />
  <default-off-rate value="1.0" />
</dimerization-gen>
\end{lstlisting}

Each allosteric form of the reaction can be listed in its own
allo-rates element.  Each allo-rates element consists of two
site-shape-ref elements, and on-rate and off-rate elements.  When the
first binding-site in the dimerization reaction has the first
binding-site shape, and the second binding-site has the second shape,
then the allo-rates on and off rates are used instead.  So in
/ref{alloRatesExample}, Substrate will bind at its 'to-Enzyme' site
with Enzyme at its 'to-Substrate' site at a rate of of 1.0e12 (Hz) (l)
/ sec, unless the 'to-Enzyme' site has the shape 'occluded', in which
case the binding rate is 0.0.

\begin{lstlisting} 
<dimerization-gen>
  <mol-ref name="Substrate">
    <site-ref name="to-Enzyme" />
  </mol-ref>
  <mol-ref name="Enzyme">
    <site-ref name="to-Substrate" />
  </mol-ref>
  <default-on-rate value="1.0e12" />
  <default-off-rate value="1.0" />
  <allo-rates>
    <site-shape-ref name="Obstructed" />
    <site-shape-ref name="default" />
    <on-rate value = 0.0 />
    <off-rate value = 1.0e12 />
  </allo-rates>
</dimerization-gen>
 \end{lstlisting}


\section{uni-mol-gen}
A uni-mol-gen is a reaction generator which responds to the presence
of a single enabling mod-mol.  By adding enabling modification, you
can arrange that the reaction generator only creates reactions whose
enabling substrate is in a particular modification state.  The
generated reaction can ghange the modification state of the mol by
adding modification-exchanges.  This reaction can be made
binary by adding additional product species.  Users can likewise add
additional product species in the 'additional-product-species'
section.  

\begin{lstlisting}[caption='A simple uni-mol-gen that represents a protein
  with two phosphorylation sites that is ubiquitinated']
<uni-mol-gen>
  <enabling-mol name='Protein' />
  <enabling-modifications>
    <mod-site-ref name='PhosphorylationSite_1'>
      <mod-ref name='phosphorylated' />
    </mod-site-ref>
    <mod-site-ref name='PhosphorylationSite_2'>
      <mod-ref name='phosphorylated' />
    </mod-site-ref>
  </enabling-modifications>
  <modification-exchanges>
    <modification-exchange>
      <mod-site-ref name='UbiquitinationSite' />
      <installed-mod-ref name='Ubiquitinated' />
    </modification-exchange>
  </modification-exchanges>
  <rate value='1e10' />
</uni-mol-gen>  
\end{lstlisting}

\section{'omni-gen'}

An omni-gen is a generic reaction generator that can generate, for
each complex species containing a particular omniplex, a reaction with
flexible characteristics. The generated reactions can change any one
small-mol component of the omniplex into another small-mol, if
desired. The generated reactions can change the modification at any
one modification site on any one mod-mol component of the omniplex, if
desired. The generated reactions can be binary, if desired, requiring
any explicit reactant species as a co-reactant with the species that
is recognized as containing the omniplex. The generated reactions can
produce an arbitrary explicit product species, in addition to the
transformed primary reactant, if desired. Each of these separate
activities on the part of the generated reactions is engaged simply by
including the appropriate optional elements in this reaction generator
specification. 

\begin{lstlisting}
\end{1stlisting}

\section{'explicit-species'}

This section is where users describe species explicitly because you
want to create molecules of the species at some point during the
simulation, because you want to track the population of the species
through the simulation, or because you want to define explict
reactions involving the species. Some things that you might think
would be explicit species, such as a monomeric protein, may not be;
rather, they are constituents of species of complexes. (The monomeric
protein is a "mol" but you can define an explicit plex-species that
contains just the one mol.)

\begin{lstlisting}
<explicit-species>
  <plex-species name='AA_Dimer'>
    <mol-instance name='First_A'>
      <mol-ref name='A' />
    </mol-instance>
    <mol-instance name='Second_A'>
      <mol-ref name='A' />
    </mol-instance>
    <binding>
      <mol-instance-ref name='First_A'>
        <binding-site-ref name='to-A' />
      </mol-instance-ref>
      <mol-instance-ref name='Second_A'>
        <binding-site-ref name='to-A' />
      </mol-instance-ref>
    </binding>
  </plex-species>
</explicit-species>
\end{1stlisting}

\section{'explicit-reactions'}

Most of the reactions that libmoleculizer works with are generated
automatically by reaction generators, but you can enter reactions
one-at-a-time, too. The participants in an explicit reaction have to
be explicit species, so that they have names and you therefore have a
way to refer to them in the explicit reaction's definition. 


\begin{lstlisting}
<explicit-species>

  <plex-species name='A-singleton'>
    <mol-instance name='the-A'>
      <mol-ref name='A' />
    </mol-instance>
  </plex-species>

  <plex-species name='B-singleton'>
    <mol-instance name='the-B'>
      <mol-ref name='B' />
    </mol-instance>
  </plex-species>

  <plex-species name='AA_Dimer'>
    <mol-instance name='First_A'>
      <mol-ref name='A' />
    </mol-instance>
    <mol-instance name='Second_A'>
      <mol-ref name='A' />
    </mol-instance>
    <binding>
      <mol-instance-ref name='First_A'>
        <binding-site-ref name='to-A' />
      </mol-instance-ref>
      <mol-instance-ref name='Second_A'>
        <binding-site-ref name='to-A' />
      </mol-instance-ref>
    </binding>
  </plex-species>

  <plex-species name='AB_Dimer'>
    <mol-instance name='the-A'>
      <mol-ref name='A' />
    </mol-instance>
    <mol-instance name='the-B'>
      <mol-ref name='B' />
    </mol-instance>
    <binding>
      <mol-instance-ref name='the-A'>
        <binding-site-ref name='to-B' />
      </mol-instance-ref>
      <mol-instance-ref name='the-B'>
        <binding-site-ref name='to-A' />
      </mol-instance-ref>
    </binding>
  </plex-species>

</explicit-species>

<explicit-reactions>

  <reaction>
    <substrate-species-ref name='A-singleton' multiplicity='2' />
    <product-species-ref name='AA_Dimer' />
  </reaction>

  <reaction>
    <substrate-species-ref name='A-singleton' multiplicity='1' />
    <substrate-species-ref name='B-singleton' multiplicity='1' />
    <product-species-ref name='AB_Dimer' />
  </reaction>

</explicit-reactions>  
\end{lstlisting}

%%% Local Variables: 
%%% mode: latex
%%% TeX-master: "user-manual"
%%% End: 
