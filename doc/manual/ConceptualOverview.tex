\chapter{Comceptual Overview of libmoleculizer}

The objective of libmoleculizer is to take a description of proteins
and other molecular entities, as well as a set of descriptions of
various biochemical interactions, and use this description to generate
information in a variety of forms about the explicit reaction network
and all the species within it that are implicitly described by those
rules.  

In order to understand in detail how libmoleculizer generates reaction
networks, including their kinetics, it is importand to cover some
background information on the kinds of interactions libmoleculizer
represents and the way that libmoleculizer expands reaction networks.  

\section{Combinatorial Reaction Networks}
%% TODO - Write this section in detail 
Biological reaction networks can usually be described as a set of
interactions between proteins.  

\subsection{Signal Transduction Pathways}
Signal transduction pathways are possibly the most fruitful target for
this software.  

%% END TODO

\section{Rule-Based Reaction  Network Specifications}

The fundamental idea behind libmoleculizer is reaction networks can
be specified by describing the proteins and molecules (including
small-molecules, but also including other 'large-molecules' such as
DNA) that participate in that network, along with a set of rules, each
of with describes an individual biochemical interaction amongst those
types.  


\subsection{Species are made from structural bindings between
  structural molecules}

The fundamental physical objects in libmoleculizer are biochemical
{\it species}, which are sets of {\it mols} bound together between
their {\it binding sites}

Mols are the physical abstraction used in libmoleculizer.  Each mol is
a structural obect that must be defined by the user in the model. Mols
represent different indivisible biochemical entities, typically
proteins, but also small molecules, DNA, and potentially other objects
such as organelles. 

Each mol has its own name and is defined with a set of modular binding
sites.  Each of these binding sites is given a name so that it can
be uniquely identified, as well as a collection of {\it shapes}, which
represent the different conformational states of that binding site and
affect the allostery of reactions involving that binding site.

Mols can be bound together between their binding sites in order to
form {\it complex species}, also referred to as {\it complexes} or
{\it species}.  This includes the special case where a complex species
consists of only a single mol that is bound to nothing.  

Mols may also be defined with {\it modification sites}, locations on
the mol that can accept {\it modifications}, post-translational
modifications such as phosphorylation groups that are relevant to the
state of the mol, but which do not fit into the category of a binding
between two mols.  

\subsection{Species have  structural features based on their
  constituent mols}

The important thing about species is that they have structural
features based on the structural features of the mols that make them
up.  For instance, if a species is composed of an A protein and a B
protein, and the A protein has a modification site that is
phosphorylated, then when we bind that A protein together with the B
protein, in a molecular binding between two of their binding sites,
the complex species, the A-B dimer, will also posess a modification
site that is phosphorylated. Likewise, the dimer will posess all the
same binding sites as did the individual mols before they were bound.  

{\it Features} in libmoleculizer refer to specific structural
features, possessed by mols or sets of mols, that can be searched for
within species.  For instance, a simple feature might be a
particular modification site on a particular mol with a particular
modification state, for instance a particular phosphorylated
modification site on a Ste11, a kind of protein kinase.  Any complex
species that contains a Ste11 mol with that modification site
phosphorylated will match that feature.

A more complicated feature might be a binding between a particular binding
site on a Ste11 to a particular binding site between a Ste5.  Any
complex containing a Ste11 bound to a Ste5 at the correct binding
sites will match that feature.  

There is no limit to the complexity that features in libmoleculizer
can have (although very complicated features are probably not so
biologically relevant).  It is entirely possible to talk about a
feature which is a complicated sub-complex, called an {\it omniplex}
in the language of libmoleculizer, consisting of many mols bound
together in just the right way and possessing various modifications.

\subsection{Rules are amongst features}

Rules in libmoleculizer specify features to look for and a
transformation to apply to those features.  For instance, using a {\it
  dimerization-gen} rule, two features are specified, each of which is
a particular free binding-site on a particular mol.  The result of
finding these two features is to join them with a binding.  The
dimerization rule is defined between features of mols.  When this rule
is applied to two species that possess these features, the result is
to bind them together in a $Complex1 + Complex2 \arrow
DimerizedComplex$ reaction.  

Other reactions may look for specific binding features between
proteins (in order to generate a decomposition reaction), particular
modification site features, or omniplex features, in order to use them
in their reactions.

In fact, the three basic kinds of reaction rules are
dimerization-gens, uni-mol-gens, and omni-gens. Dimerization-gens
create dimerization reactions as well as their decomposition
reactions by looking for features corresponding to free binding sites
amongst pairs of species.  Uni-mol-gens create new reactions by
looking for features that are specific types of molecules, possibly
with certain sets of modification states associated to them.  Finally,
omni-gens look for omni-plex features (these are subcomplexes that may
or may not be defined with associated modifications as well).  

Each of these reaction types will be better defined and discussed in
later chapters.  For now, the important information is the definition
of a {\it feature}, a collection of structural information that may
exist at either the mol, or the sub-complex (collection of mols)
level, and the knowledge of how rules look for complex species
containing specific features in order to apply their transformations,
such as dimerization, decomposition, small-mol exchanges, and
modification-exchanges.  

\subsection{Reactions are amongst species}

When a rule, which looks for either one or two features depending on
whether it will produce unary or binary reactions, finds two species
that contain those features, it will be applied.  A dimerization-gen
will find two species with two free binding sites, a uni-mol-gen will
find a species containing a specific mol in a particular state, and an
omni-mol-gen will find a species containing a particular omni-plex.  





Each rule describes a set of 'features' and a transformation that when
contained in a particular species, in the case of an unary reaction
rule, or amongst a pair of species, in the case of a binary reaction
rule, will generate

   

, if found in a particular
species or pair of species, will
imply 

(in the case of an unary reaction rule, e.g. a reaction rule that
specifies that a Ras protein will hydrolize bound GTP to GTP at a 
specific rate ) 

(in the case of binary
reaction rules, e.g. a reaction rule that specifies that a Ste5
protein with particular binding domain free may bind with a Ste11 with
one of its specific binding domains free at a particular rate)


For example, a reaction rule might be represent a protein-protein
dimerization.  If a model has defined a Ste5 protein that has a
binding site 'ste5-binding-site1' as well as a Fus3 protein that has a
binding site 'fus3-binding-site1', a reaction rule can be specified
that says Ste5 can bind to Fus3 from the 'ste5-binding-site1' to the
'fus3-binding-site1' at a particular on-rate $k_{on}$, and a particular
off-rate $k_{off}$.  This rule will be applied in two ways.  First,
any time a pair of molecular complexes, the first of which contains a
Ste5 molecule with a free 'ste5-binding-site1' site, the second of which
contains a Fus3 molecule with a free 'fus3-binding-site1', this rule
will be applied to generate an explicit dimerization reaction between
those two complexes.  At the same time, the off-rate provided will be
using to generate an explicit decomposition reaction for the dimerized
complex.  

The on- and off- rates used in reaction rules are used to generate
appropriate on- and off-rates in the explicit reaction.  This process
can be configured in a variety of ways.  Among other things, values
can either be passed directly through to explicit reactions, or
reaction rate extrapolation can be used, where libmoleculizer will
approximate explicit reaction rates for the generated reactons, based
on the kinetics of the rule, as well as the masses of the substrate
species.  Please see the section dealing with Reaction Rate Kinetics
for more infomation.  

\section{Allostery Based on Binding Shape}

Biochemical reaction networks, particularly signal transduction
pathways, will oftentimes regulate themselves using allosteric
proteins.  Allosteric proteins are proteins that have functional sites
(binding sites) and regulatory sites.  Typically different small
molecules are bound to the regulatory sites, and this induces
conformation changes to to functional site, which modify its
functional kinetics.  

At the physical level, states of the regulatory sites in a complex
species affect the conformation of functional binding sites, and
presence and state of subcomplexes within a complex may affect other
kinetic properties of functional sites (e.g. a particular protein
being bound at a particular binding site may physically block a
functional site, preventing it from being used -- in affect, turing
all the rates to 0 for that binding site).  MZR rules use the idea of
binnding-site shape, a comcept that encompasses both allosteric
changes to conformation, as well as other non-conformational physical
effects that affect reaction kinetics.  

Each binding site within libmoleculizer must be defined with a default
binding-site shape, as well as a list of the other shapes that can
occur at that binding-site. In another section of the rules,
the conditions under which complex state induces binding site shape
changes are specified.

Finally, in reactions that occur between functional sites, allosteric
rates are specified in terms of the relevant binding site shapes.  

\subsection{An Argument For Using Binding Shapes to specify allostery,
  over other Methods}




\section{Reaction Rate Kinetics}



\subsection{Mass-based reaction rate extrapolation}



\section{Unique Complex Naming}