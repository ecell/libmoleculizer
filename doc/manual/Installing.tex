\chapter{Installing Libmoleculizer}
\label{chap:installingChapter}

\section{Compiling and installing from source}
The libmoleculizer package uses the gnu autotools and should be
pretty easy to install on a wide variety of platforms. 

\subsection{Prerequisites}
Libmoleculizer has several prerequisites: 3rd party libraries used by
libmoleculizer in order to function that are needed in order to
compile and run libmoleculizer.

The necessary prerequisites are
\begin{enumerate}
\item The Boost C++ libraries, 
  \emph{http://boost.org}
\item The Gnu Scientific Libraries, 
  \emph{http://gnu.org/software/gsl}
\item Libxml++v2.6, 
  \emph{http://libxmlplusplus.sourceforge.net/}
\end{enumerate}

These libraries can either be installed from source, or by using a
package manager, if one is available on the target system.  For
instance, on the Ubuntu system I use, these dependencies can be
fulfilled by running the commands ``sudo apt-get install
libxml++2.6-dev'', ``sudo apt-get install libboost-dev'', ``sudo apt-get
install libboost-python1.34.1'', ``sudo apt-get install
libgsl0-dev''.  This will quickly install the necessary files
automatically, in a way that should guarantee proper configuration and
detection by libmoleculizer.  Once the prerequisites have been
installed, compilation and installation of libmoleculizer can begin.

\subsection{Simplest compile/installation Procedure}
Depending on your system configuration, the following procedure will
probably be all that is necessary to do.

Go to \libmzrwebsite and download the latest
  libmoleculizer.tar.gz source file, and make sure it is unpacked.  

%\begin{ShellCommands}
./bootstrap.sh
./configure
make
sudo make install
%\end{ShellCommands}

Difficulties, if any, will probably occur in the {\bf ./configure} step.  

\subsection{Can't or Dont want to install libmoleculizer globally}
By default {\bf ./configure} installs libmoleculizer to a global
location, usually /usr/lib or /usr/local/lib depending on your
system.  This may either be unacceptable or impossible, say if you do
not have administrator access on your computer.  This can be fixed py
passing a {\bf --prefix=\emph{location}} flag to the {\bf ./configure}
command.  

\subsection{./configure cannot find necessary libraries}
Configure may fail and say that one of the prerequisites cannot be
found.  The first thing to do, of course, is make sure these libraries
are installed on your machine.  If they are, then configure must be
explicitly told where to find the libraries and/or the corresponding
header files. The easiest way to do this is to pass the header file
locations to the CXXFLAGS environmental variable, and the library
locations to the LDFLAGS environmental variable.  

For instance, suppose that both GSL as well as libxml++ are not found
by the configure script, but their header files can be found in
directories /foo/bar/gsl and /fizz/wizz/libxml respectively and
likewise the libraries themselves can be found in /a/b/gsllib and
/c/d/xmllib then by running the command 

{\bf CXXFLAGS=''-I/foo/bar/gsl
  -I/fizz/wizz/libxml'' \\
LDFLAGS=''-L/a/b/gsllib -L/c/d/xmllib'' \\
  ./configure'} 

will run configure such that it scans the appropriate
  directories for the needed files.  

\subsection{Dynamic Libraries or static build}
Finally, one additional configure command should be mentioned here.
By default, libmoleculizer consists of several libraries, and to use
it, a client program must link to all of them.  If desired however,
the library can also be built as a single large library, by passing
the flag {\bf --enable-static-build} to the configure script.
