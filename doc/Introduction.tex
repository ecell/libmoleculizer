\chapter{Introduction}
\label{chap:introduction}

\section{Overview of libmoleculizer}
Libmoleculizer is a computer library that takes a set of rules
describing a collection of protein and small molecule types as well as
a set of descriptions of interactions amongst those types,
i.e. different types of protein-protein, protein-small molecule, and
enzymatic reactions (protein modifications), and using that
information explicitly generates a network of multi-protein species
and their reactions that are described by those rules.

The ability to describe networks implicitly using rules can be be
very advantageous for modeling biochemical networks.  This is because
for many common cases, the size of the explicit network grows
combinatorially relative to the number of protein/molecule types and
interactions types that network possesses.  For instance, in a particular
model of the alpha mating pathway in S. cerivisiae, defined
with fewer than 20 molecule types and fewer than 40 interactions, the
explicit network consists of over 350,000 species and over 1,000,000
reactions.

Libmoleculizer is a computer tool that helps manage the complexity of
the absolute set of species and reactions in a reaction network, by
expanding the typically much smaller and much more comprehensible
network of biochemical interactions that users provide.  Typically,
Libmoleculizer will be used as a component of a larger simulator,
which will use libmoleculizer for species and reaction management.
For instance, a simulator using libmoleculizer could take in a set of
rules, use libmoleculizer to expand the corresponding network, and
then use the generated network to run an ODE or stochastic simulation.

Where libmoleculizer goes beyond other software packages of its type
however, is in the generation of these networks.  Biochemical networks
do not have to be expanded all at once.  This has two advantages: it
allows libmoleculizer to be able to handle networks other packages
cannot, such as unbounded species growth, and it also allows for
faster operation.  In a spatial simulation, if two species collide,
the user will not care to know every species and reaction that can
occur in the model, they will only want to know whether those two
species react or not.  Using libmoleculizer, users do not have to
computationally pay for the time to generate huge portions of the
network they do not care about.  

\section{Manual Overview}

\subsubsection{Disclaimer}
At the current moment, this manual is highly incomplete.  We expect
that in the next couple weeks, the next version of libmoleculizer will
be released which will greatly flesh out this user-manual.  For the
current release, we have decided to include all, including in
progress, documentation -- however, certain aspects will be very, very
roughly sketched out, to the point of being unreadable.  

For the time being, chapters marked with a star are considered
incomplete. 

\subsubsection{Overview}

This manual aims to provide introduction to the functionality and use
of the libmoleculizer library.  

This manual has three major goals.  The first of the manual is to
describe what Moleculizer does; this is treated in chapters
\ref{chap:conceptualOverviewChapter} and
\ref{chap:usingLibmoleculizerChapter}.  The second objective is to
describe the MZR file format in which reaction rules are specified;
this is given an overview in chapter \ref{chap:theRulesChapter}, and
more explicitly in chapter \ref{chap:mzrReference}.  Finally, the
libmoleculizer API must be learned, in order to understand how to pass
MZR files into libmoleculizer, expand the network, and read out
the generated species and reactions.  This topic is dealt with in
chapters \ref{chap:usingLibmoleculizerChapter},
\ref{chap:interfacesChapter}, and \ref{chap:apiReference}.

Finally, although libmoleculizer is a written in standard C++ and C,
and packaged using standard methods, we cannot guarantee you won't
have problems installing it.  Start by reading and following the
instructions in the INSTALL file that came with libmoleculizer.  There
is a 99\% chance that will work.  However, if you do have any trouble,
consult the installation instructions and troubleshooting help
available in the chapter \ref{chap:installingChapter}.

\section{History}

Libmoleculizer is the descendant of a prior program called Moleculizer,
developed by Dr. Larry Lok at the Molecular Sciences Institute to
study the alpha mating pathway in yeast. The standard modeling
techniques at the time all involved explicitly enumerating a set of
chemical species and reactions. However, it was determined for the
alpha signal transduction pathway in yeast that there were so many
species and reactions possible, that they could never be practically
enumerated. The typical response to this problem had previously been
to simplify the system, combining species and reactions, in order to
produce a simpler but non-exact simulation of the system
involved. This approach was non-optimal, as it required making
fundamental assumptions about what was and was not important, prior to
investigating system behavior.

The response was to develop a new simulator called
Moleculizer\cite{lok05} that took in a description of the basic
monomeric proteins and small molecules, as well as a set of rules that
described basic interactions between proteins. These rules were used
within a simulator implementing Dan Gillespie's first reaction
algorithm \cite{gillespie77}.  As it ran, Moleculizer would alternate
between executing reaction events in the manner of Gillespie and using
the rules to generate enough of the reaction network so that the
 the next reaction event would always be accurate.

Realizing the versatility of this new rule-based approach, a new
software project was begun by Nathan Addy at the Molecular Sciences
Institute, to adapt the reaction network generating components of
Moleculizer into a generic library, called libmoleculizer, that could
be used within a variety of simulators and in a variety of contexts.

This manual corresponds to version \currentversion of that software. 