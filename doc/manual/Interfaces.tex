 \chapter{Libmoleculizer Interfaces}
Libmoleculizer has multiple wrappings that allow users of different
languages -- C++, C, or Python -- to use it in their own programs.
This chapter describes each of these interfaces and how to use them, 
and is followed by a section discussing how to extend this procedure
to new languages.  

\subsection{C++ Interface}
The C++ interface is the standard interface for libmoleculizer, as
well as the language that libmoleculizer has been implemented in.
Consequently, the C++ interface is the most powerful as well as the
most detailed. 

The C++ interface controls creating a libmoleculizer intance,
configuring it, loading a rules file into the instance, and expanding
the reaction network out.  The interface also allows inspecting the
network to see whether or not certain species have been created, see
what reactions might be possible for a set of products, as well as
other questions.  These parts are given an overview below, although
for a complete overview of the API, please consult either reference
portions of this documentation, or browse the source code (the
moleculizer interface can be found in src/mzr/moleculizer.hh and
src/fnd/reactionNetworkDescription.hh).


\subsubsection{Overview}
Use of libmoleculizer will usually follow, a ''create, configure,
load, run'' pattern, with the first three parts being relatively
constant from time to time.  

\subsubsection{Create}
To use libmoleculizer in C++ code, a Moleculizer object, which
represents a reaction-network that can be generated, must be
created. This is easy, as Moleculizer only has one constructor:
moleculizer::moleculizer().  An individual wishing to use generative
reaction-networks libmoleculizer library to be used in his/her work
then would begin by creating a moleculizer object.  

\lstset{language=C++}
\begin{lstlisting}[caption=Creating a moleculizer object, label=creatingmzrexample]
mzr::moleculizer m; // Note that this takes place in the namespace mzr,
                    // as does all the C++ interface.
\end{lstlisting}


\subsubsection{Configure}
Now that a new Moleculizer instance has been created, it must be
configured as far as what type of reaction-network expansion to do, as
well as configured to determine how deep to expand the
reaction-network.  

Primary here is to determine whether or not to use spatial or
non-spatial reaction network generation.  In each mode, new species
and new reactions will be generated, however the parameters generated
will be different depending on the case.  Spatial reaction network
generation generates species-reaction networks with parameters
appropriate for simulating the network in a spatial simulator: when a
new species is generated, a mass, species radius, and diffusion
coefficient is generated for that new species, and when a new reaction
is generated, an intrinsic reaction rate is generated for that new
reaction.  In non-spatial mode, parameters appropriate for either a
Gillespie type stochastic simulation or a determanistic ODE simulation
are produced: when a new species is generated, a mass is generated
along with it, and when a new reaction is generated, a reaction rate
is generated along with it.  Please see previous chapters for a
description of these modes.

To configure a moleculizer object to use either of these modes, please
use either the ''enableSpatialReactionNetworkGeneration'' or the
''enableNonSpatialReactionNetworkGeneration(bool)'' functions.  The
boolean parameter needed by
''enableNonSpatialReactionNetworkGeneration'' controls massed-based
extrapolation.  If false, the reaction rate extrapolated for a
reaction is the same as the reaction rate described in the basic
reaction-rule in the model file.  If it is true, the reaction-rate is
modified by a ''mass-correction factor'', as described in the Brent,
Lok paper.  

\lstset{language=C++}
\begin{lstlisting}[caption=Enabling spatial reaction generation]
  mzr::moleculizer m;                          // Create a Moleculizer object.
  m.enableSpatialReactionNetworkGeneration();  // Configure it to generate spatial partameters.
\end{lstlisting}

\begin{lstlisting}[caption=Enabling non-spatial reaction generation]
  mzr::moleculizer m;                                   // Create a Moleculizer object.
  m.enableNonspatialReactionNetworkGeneration( true );  // Configure it to generate non-spatial parameters, using mass-based extrapolation.
\end{lstlisting}

If spatial species-reaction network extrapolation is going to be used,
please note that the 'explicit-species' elements must have an
additional 'parameters' subsection, that provides a diffusion
coefficient for each initial species in the model.  Failure to do so
will result in a runtime error.  Please see the chapter ''Rules'' for
more information.

The second option that can be configured prior to runtime is the
expansion depth of species-reaction generation.  At any given moment,
beginning with the list of explicit-species provided in the rules as
well as the generated set of reactions between them, moleculizer
maintains a generated reaction-network, consisting of a list of
species and a list of reactions.  Some of these species are
'expanded'; that is, Moleculizer has presented each of those species
to the list of species, has generated any new reactions based on them,
and has added the new products of those reactions as 'unexpanded'
species to the list.  Others are 'unexpanded', meaning that the
reactions that species can participate in in, and any subseqeuent
products of those reactions, have not yet been generated and added to
the reaction network generation.  

By default, Moleculizer expands species out to one ply.  This means
that when a species is expanded, new reactions that involve already
expanded species and the species being expanded are created, and
reaction products are added to the network as being unexpanded.  

Higher plys imply more species and reactions are generated with each
expansion step, at the expense of more computational time and memory.
For instance, expanding a species to a ply of two means that new
reactions involving that species are generated, along with reaction
products.  However, each of those products are also expanded.
Expanding each of these products will cause more species and reactions
to be generated.  Expanding to a third-ply will take each of these
species and expand them in turn.  The standard ply of one is fine for
most applications.  However, if desired, it can be checked using the
'getGenerationDepth' function, and set using the
'setGenerateDepth(unsigned int)' functions.  

\lstset{language=C++}
\begin{lstlisting}[caption=Setting a non-default network generation depth]
  mzr::moleculizer m; 
  m.enableNonspatialReactionNetworkGeneration( true );
  m.setGenerateDepth( 5 );
\end{lstlisting}
  


\subsubsection{Loading}
Once configuration parameters have been set up, a model can be
loaded.  This can be accomplished using one of three functions:
\lstinline@attachFileName( const std::string& aFileName )@.
\lstinline@attachString( const std::string& documentAsString )@,
or \lstinline@attachDocument( xmlpp::Document* pDoc )@.  Each of these takes a
model, in one form or another, loads it, parses it, and loads it,
preparing Moleculizer for species and reaction generation.  

A libmoleculizer model is an xml description of basic molecules (mols) and
reaction rules defined in terms of molecules, that usually describe
either protein-protein interactions or enzymatic reactions on
proteins.  A description of the full syntax of these files can be
found in the chapter 'Rules'.  An xml specification can be either
passed to libmoleculizer as a string containing file contents, a path
to a model file, or can be passed in as an xmlpp::Document*
(corresponding to an xml document parsed as a dom object using
libxml++, the xml library used by libmoleculizer).

Use cases:
1.  Model file is contained in a file.
If filePath is a string containing either an absolute or relative path
to the file, then the function moleculizer::attachFileName can be
called as follows.  

\lstset{language=C++}
\begin{lstlisting}[caption=Loading a MZR file into a running
  moleculizer instance]
// Create a new reaction-network generator.
mzr::moleculizer m;

// Setup extrapolation here. 
m.enableSpatialReactionNetworkGeneration(); // For instance.

// Load the file into the moleculizer object.
m.attachFileName( filePath );
\end{lstlisting}

2..Model is contained in a string.
If modelContents is a string containing a model file, then the
function moleculizer::attachString can be used to load this model file
as follows

\begin{lstlisting}[caption=Loading a MZR string into Moleculizer]
// Create a new reaction-network generator.
mzr::moleculizer m; 

// Setup extrapolation here.
m.enableNonspatialReactionNetworkGeneration( true ); // For instance.

// Load the string holding the model file into the moleculizer object.
m.attachString( modelContents );
\end{lstlisting}

3.  Model has already been parsed and is contained in an
xmlpp::Document* object called ptrModelDoc.

\lstset{language=C++}
\begin{lstlisting}[caption=Loading an XML document into moleculizer]
// Create a new reaction-network generator.
mzr::moleculizer m;

// Setup extrapolation here.
m.enableNonspatialReactionNetworkGeneration( false );  // For
instance.

// Load the Document* into moleculizer.
m.attachDocument( ptrModelDocument );
\end{lstlisting}

Any of these will load the necessary information into a model
instance. Note however, that information can only be added into
moleculizer once per instance.  Should the model have to be updated,
the moleculizer object must be destroyed and recreated and loaded with
the new model.  This is not optimal, and may change in upcoming
versions.  Users can determine whether a model is loaded by calling
the ''bool moleculizer::getModelHasBeenLoaded() const'' function.  

\subsubsection{Running}
Once a Moleculizer object has been created, configured, and a
rules-file has been loaded, there are several things that can be done.

The things that can be done are inspecting the species and reactions
that have been generated so far, expanding species and reactions in
order to generate additional portions of the species and reaction
network, inspecting the portion of the species and reaction network
that were created during the most recent expansion phase, and
searching for reactions that match certain parameters.  

The most basic operation that can be performed on a species and
reaction network is simply inspecting its contents.  These can be done
using the interface functions ''getSpeciesCatalog'' and
''getReactionList'' functions.  The 'getSpeciesCatalog' function
returns a map from std::string pointers to mzr::mzrSpecies*.  (Please
note that this map compares pointers based on their dereferenced
values rather than the pointer values themselves.)  By scanning
through this map, users can find out the identities of every species
that has been created, by using its unique identifier.  Using this
map, information can be gained for each species, using the relevant
'getKDForSpecies', 'getRadiusForSpecies', 'getMassForSpecies'
functions.  

\lstset{language=C++}
\begin{lstlisting}[caption=Finding spatial properties of a species
  using moleculizer]
mzr::moculizer m;
m.enableSpatialReactionNetworkGeneration();
m.loadFile(''moleculizer-rules.xml'');

std::pair<std::string, const mzr::mzrSpecies*> speciesPair =
m.getSpeciesCatalog().first;

const mzr::mzrSpecies pSpecies = speciesPair.second;

Real speciesMass = m.getKDForSpecies( pSpecies );
Real speciesRadius = m.getRadiusForSpecies( pSpecies );
Real speciesKd = m.getKDForSpecies( pSpecies );
\end{lstlisting}

Note that every configuration of libmoleculizer generates masses for
newly generated species, and so ''getMassForSpecies'' works under all
situations.  However, species radii and diffusion coefficients are
only generated when spatial extrapolation mode is configured by
calling the ''configureSpatialGenerationMode'' prior to loading the
model.  

\begin{lstlisting}[caption=Finding out information about species in
  non-spatial mode]
mzr::moculizer m;
m.enableNonSpatialReactionNetworkGeneration( false );
m.loadFile(''moleculizer-rules.xml'');

std::pair<std::string, const mzr::mzrSpecies*> speciesPair =
m.getSpeciesCatalog().first;

const mzr::mzrSpecies* pSpecies = speciesPair.second;

Real speciesMass = m.getKDForSpecies( pSpecies ); // ok

// This raises a BadModeXcpt exception, because this parameter is 
// not generated for this mode.
Real speciesRadius = m.getRadiusForSpecies( pSpecies ); 
\end{lstlisting}

Likewise, for determining parameters associated with the reaction,
functions ''



\subsubsection{The ReactionNetworkDescription interface.}
The ReactionNetworkDescription interface
(fnd::ReactionNetworkDescription<mzr::mzrSpecies, mzr::mzrReaction> in
code, found in src/fnd/reactionNetworkDescription.hh) manages questions
about species and reactions that have been created within a session of
libmoleculizer.  The basic, relevant functions to using this interface
are listed below.  Full a full description of this interface, please
consult class documentation provided elsewhere in this documentation.

Generally speaking, the ReactionNetworkDescription interaface can do
three things:

1.
Find species with certain names, using the findSpecies function.  This
function takes the name of a species and either returns a constant
pointer to that species, or throws a fnd::NoSuchSpeciesXcpt
exception.  Note that an exception may be thrown even if the species
name is a legal one for this model, so long as that species has not
been constructed yet.  Typically this is appropriate.  For instance,
any species formed as a procuct of a generated reaction is guaranteed
to have been created such that it will be returned by findSpecies.
That said, there may be special cases, such as working with data from
a previous run of the same model, where legal species names must be
used to create species even though the current expansion of the model
has not expanded that portion yet.  In this case,
mzr::moleculizer::getSpeciesWithName, as described in the previous
section, should be used.  

2.  Find reactions with one or more substrates.
This is accomplished using the findReactionWithSubstrates functions.
If unary reactions are desired (what decompositions can a particular
species undergo), use the function 
\lstinline@bool findReactionWithSubstrates(SpeciesCPtr A, std::vector<ReactionTypeCPtr>& reactionVector)@ function.  To use 
this, a species pointer that has been gotten using either the
findSpecies or getSpeciesWithName functions is passed, along with an
empty vector containing ReactionTypeCPtrs.  The function places a
unique pointer to every unary reaction involving that species as a
substrate, returning true if any exist.  

For instance:
m = mzr::molecuculizer();
m.attachFile( ``Simpledemo-rules.xml''); 

std::vector<const mzr::mzrReaction*> speciesDecompositionRxns;;

// This function also converts user names, described in the rules, to
// species pointers.
const mzr::mzrSpecies* species = m.getSpeciesWithName(``A-B-dimer'');

// When this function returns, speciesDecompositionRxns will contain
// one element, corresponding to the ``AB -> A + B'' reaction.  
m.findReactionWithSubstrates( species, speciesDecompositionRxns);

For reactions involving two substrates (libmoleculizer does not
support reactions involving three or greater substrates at this time),
the same idea is followed, except involving two species.  

Example:
\begin{lstlisting}
m = mzr::moleculizer();
m.attachFile(``Simpledemo-rules.xml'');
std::vector<const mzr::mzrReaction*> binaryRxns;;

const mzr::mzrSpecies* species1 = m.findSpecies(``___1A______'');
const mzr::mzrSpecies* species2 = m.findSpecies(``___1B______'');

// After this function call, binaryRxns will contain one element
// corresponding to the reaction ``A + B -> AB'').
m.findReactionWithSubstrates(species1, species2, binaryRxns);


ReactionNetworkDescriptionInterface also provides general information
on the state of the generated network through several functions:

unsigned int getTotalNumberSpecies() const, 
unsigned int getTotalNumberReactions() const,
bool checkSpeciesIsKnow( const std::string& speciesName) const.
\end{lstlisting}

These functions all provide information about the state of the
generated network.  The first two tell how large it is, and the third
helps determine what is and is not in it.  

Finally, there are other functions for manipulating and expanding the
reaction network.  Each species and reaction is of classtype
\lstinline@fnd::reactionNetworkComponent@, and had a function
\lstinline@expandReactionNetwork(unsigned int)@.  This function will expand the
reaction network around that object.  For any species X, calling this
function will create all new reactions X + Y -> Z, where Y is a species
already present in the system such that X and Y have a reaction rule
which applies to them and Z is the product, as determined by the
reaction rule.  Furthermore, any unary reactions X->? will be created
and recorded in the system.  Finally, all species that occur on the
right hand side of newly created reaction rules will be created and
added to the system.  This sort of expansion can be done in two ways.

1.  Calling \lstinline@expandReactionNetwork(unsigned int i)@ on a particular
reactionNetworkComponent.

All species \lstinline@mzr::mzrSpecies@ and reactions \lstinline@mzr::mzrReaction@ are
also of type \lstinline@fnd::reactionNeworkComponent@, and as such, have a function
called \lstinline@expandReactionNetwork(unsigned int i)@.  This function can only
be called once for any such reactionNeworkComponent (calling a second
time has no effect), but will expand the generated reaction network
around that component, generating new species and reactions.


2.  Calling
\lstinline@fnd::reactionNetowrkDescription::incrementNetworkBySpeciesName(const SpeciesID& name)@.  

Calling this function and giving it the name of a species already
recorded within the system has the same effect as calling
expandReactionNetwork on the mzr::mzrSpecies* returned by calling
fnd::ReactionnetoworkDescription::findSpecies using that same name. 

One the reaction network has been expanded, users may be interested to
find the difference with the old network: the set of species and
reactions that were generated in that step.  





\subsection{C Interface}
  The c interface is a c language version of the full C++ interface.  Using the c interface is as easy as including the file mzr/ 
\subsection{Python Interface}







