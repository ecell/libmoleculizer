\chapter{Rules}
A libmoleculizer input file (Moleculizer Rule File, or MZR
file) is an xml document in a particular syntax that represents a set
of molecule and species definitions as well as a set of reaction rule
descriptions.

This chapter discusses MZR file syntax, and describes how to express
systems of rules describing signal transduction pathways or other
biochemical systems in order to input it into libmoleculizer and
generate a network with it.

\section{MZR File Overview}
A MZR file consists of a 'moleculizer-input' root tag that contains a
single 'model' element.  The model element contains several important
subelements, encoding the description of the network rule-based model:
'modifications', 'mols', 'allosteric-plexes', 'allosteric-omnis',
'reaction-gens', 'explicit-species', and 'explicit-reactions'.

<!-- Overall structure of a MZR file -->
<moleculizer-input>
  <model>
    <modifications />
    <mols />
    <allosteric-plexes />
    <allosteric-omnis />
    <reaction-gens />
    <explicit-species />
    <explicit-reactions />
  </model>
</moleculizer-input>

\section{Modifications}
The modifications element contains zero or more modification
elements.  Modification elements represent the different modification
states ( e.g. 'none', 'phosphorylated', 'ubiquitinated' ) that can
occur at complex species modification sites the model. Modification
values at modification sites must always correspond, and all reaction
rules and explicit reactions must respect this contraint.  

The 'modification' element itself has a mandatory name attribute, as
well as a mandatory child element 'weight-delta' that itself has a
mandatory 'daltons' attribute, which describes the mass that
modification adds to the base weight of the complex in which it
occurs.  Note that this is only important if weight-based reaction
kinetics are used.  See the 'Using' chapter in this manual for more details.

<modifications>
  <modification name="none">
    <weight-delta daltons="0.0" />
  </modification>
  <modification name="phosphorylated">
    <weight-delta daltons="42.0" />
  </modification>
  <modification name="doubly-phosphorylated">
    <weight-delta daltons="84.0" />
  </modification>
</modifications>

\section{mols}

Complexes in libmoleculizer are sets of molecules that are bound
together between their binding sites in some configuration, with each
modifcation site in each molecule in the complex associated with some
modification in the 'modifications' section. Reaction rules describe
interactions between various combinations of these molecules, and
libmoleculizer applies these rules to each of the complexes that
match the correct subcomplexes.  

In the mols section, each of the molecule types that can occur in
complexes within the model are specified.

Libmoleculizer supports two kinds of molecules in its networks:
structural molecules, typically representing proteins, that have some
number of binding and modification sites, and small molecules, which
represent small particles that participate in reactions amongst
complex species, but do not have any structure themselves.

These two kinds of molecules correspond to the two kinds of
subelements that may go in the mols section: 'mod-mol' elements and
'small-mol' elements.   

\subsection{mod-mol}
The mod-mol element represents a protein in the reaction netwwork.  It
may have one or more binding sites  and one or more modification
sites.  These mod-mols are the basic objects referred to by
reaction-gens (a reaction rule), and can be a part of large
multi-protein complexes.  

A mod mol element has a mandatory weight subelement, zero or more
binding-site elements and one or more mod-site elements.  

The weight subelement has a mandatory 'daltons' attribute, which takes
the default, unmodified, weight for that mod-mol.

<mod-mol name="Gpa1">
  <weight daltons="54075.9"/>
  <!-- binding-site and mod-site elements -->
</mod-mol>

\subsubsection{binding-site}
Mod-mols can have one or more binding sites, described by one or
more binding-site subelements.  Each binding site represents a 
location of iteraction where the mod-mol can complex with other proteins.

Each binding site is also associated with a list of binding
site states, called 'shapes'.  The use of binding site shapes
semantic is a critical point to libmoleculizer.  Moleculizer makes the
assumption that all kinetics

In a MZR model,
allosteric kinetics are specified in terms of 
combinations of binding site shapes.   




represents the  of possible shapes that the binding site may exist
in due to allosteric interactions.  These shapes are then referenced
in reaction rules in order to specificy allosteric reaction rate
kinetics.   

Each binding site element has a name attribute, that must be unique
amongst the other binding-sites of the parent mod-mol.
Each site must have on or more site-shape elements, representing the
the list of binding site states which can affect binding kinetics for
that binding site.  Also, an element 'default-shape-ref' with a
mandatory name attribute must be present, representing the default shape
taken by that bindinng site.  


### Note that no information about the conditions under which the binding
### site takes one site shape or another, other than the
### default-site-shape, is recorded here.  That information is listed in
### the allosteric-plexes and allosteric-omnis sections.

<mod-mol name='Gpa1'>
  <!-- Otherwise mandatory 'weight' has been omitted --> 
  <binding-site name="to-GXP">
    <!-- A list of the different site-shapes this binding site -->
    <!-- can posess.  -->
    <site-shape name="default"/>
    <default-shape-ref name="default" />
  </binding-site>
  <binding-site name="to-Ste2">
    <!-- The list of site-shapes and the default-shape-ref can be -->
    <!-- intertwined.  -->
    <default-shape-ref name="default" />
    <site-shape name="default" />
    <site-shape name="Ste4-bound" />
  </binding-site>
  <binding-site name="to-Ste4">
    <default-shape-ref name="default" />
    <site-shape name="default" />
    <site-shape name="Ste2-GTP-bound" />
    <site-shape name="Ste2-GDP-bound" />
    <site-shape name="GTP-bound" />
    <site-shape name="GDP-bound" />
  </binding-site>
</mod-mol>


\subsubsection{A note on the binding-shape state constraint of
  libmoleculizer}
As mentioned elsewhere in this manual, one of the primary modeling
metaphors used by libmoleculizer is to ensure each binding site has a
site-specific shape at all times.  Binding site shapes are then the
mechanism by which allosteric kinetics are specified in Moleculizer.

For instance, suppose two mod-mols: Fus3 and Ste5 are defined in the
mols section of the MZR rules file.  Fus3 has one binding site
'to-Ste5' that has only one state possible: 'default'.  Ste5 has two
binding sites, 'X' and 'to-Fus3', and 'to-Fus3' can be in either a
'default' state or an 'occluded' state.  This information is specified
in Fus3 and Ste5's mod-mol element definitions.
v
In either the allosteric-plexes or the allosteric-omnis section of the
file, conditions under which different binding sites obtain different
shapes can be specified.  For instance, in the previous example, a
rule that states that when Ste5 is complexed with a mod-mol 'Ste11' at
its 'X' site, then its 'to-Fus3' site has the shape 'occluded', this
can be specified in the allosteric-omniss sections.  

Finally, allosteric kinetics are specified in the specific
reaction-gen (reaction rule) they apply to, and are specified relative
to the binding site and binding site shape participating in the
reaction.

For instance if the general reaction rule is 'Ste5 binds with
Fus3 at its 'to-Fus3' site at an on rate of k' and the allosteric
exception is '...unless the 'to-Fus3' site has the shape 'occluded',
in which case the reaction proceeds at a rate of 0,' the allosteric
exception is specified in the reaction gen that defines the general
reaction rule.

\subsubsection{mod-site}
A mod-site element represents a modification site on the mod-mol.
That is, a site where the protein can undergo post-translational
modifications as a part of the reaction network.  

Each mod-site-element has a mandatory 'name' attribute, and a
default-mod-ref subelement, which has a mandatory 'name' attribute
itself, that must be identical to one of the modification elements in
the modifications section.

<modifications>
  <modification name="none">
    <weight-delta daltons="0.0"/>
  </modification>
  <modification name="phosphorylated">
    <weight-delta daltons="42.0"/>
  </modification>
</modifications>

<mols>
  <mod-mol name="Gpna1">
    <mod-site name="phosphorylation-site-1">
      <default-mod-ref name="none"/>
    </mod-site>
    <mod-site name="phosphorylation-site-2">
      <default-mod-ref name="none"/>
    </mod-site>
  </mod-mol>
</mols>


\subsubsection{allosteric-state}
This optional section describes how one or more binding-site shapes in
a mod-mol can change based on the states of its modification sites.

\subsection{small-mol}
A small-mol element represents a small molecule - typically an organic
molecule that participates in pathway reactions.  ADP/ATP are perhaps
the most common examples.

The most defining semantic charecteristic of the small molecule as
well as the small-mol, is a lack of structural information.  Small 
molecules participate in chemical reactions in toto.  They do not have
binding sites, nor do they have modification sites. 

The definition for a small-mol element is a mandatory name attibute
and a mandatory weight child with a mandatory daltons attribute.

<mols>
  <small-mol name="ADP">
    <weight daltons="427.20"/>
  </small-mol>
  <small-mol name="ATP">
    <weight daltons="507.181"/>
  </small-mol>
</mols>

\section{allosteric-plex}

\section{allosteric-omni}

\section{reaction-gen}

The reaction-gens section is where all of the rules for extrapolating
reactions go.  In the current version of libmoleculizer, there are two
main reaction-generators: dimerization-gen, which represents a
protein-protein binding interaction as well its reverse decomposition
reaction, and the omni-gen, which represents a generic reaction that
can express unary and binary enzymatic reactions and other reaction
forms.  

\section{dimerization-gen}
A dimerization-gen reaction generator represents a generic
dimerization/decomposition reaction between two proteins.  



\section{omni-gen}

\section{explicit-species|}

\section{explicit-reactions}




