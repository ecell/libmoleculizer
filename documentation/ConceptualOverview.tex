\chapter{Useful Concepts in libmoleculizer}
\label{chap:conceptualOverviewChapter}

Hopefully you already have some idea of what the libmoleculizer
program is: a computer system that models complicated biochemical
pathways (such as signal transduction pathways), by representing chemical
species as being composed of one or more indivisible molecule types
bound together and reactions between species as being special cases of
interactions between interactions between molecules.  

This chapter gives an overview of the key concepts needed to
understand and use the libmoleculizer system.  

\section{Moleculizer creates new species and reactions by expanding
  species}

When Moleculizer starts up, it reads in a collection of rules that
defines how a set of biochemical molecules interact; then adds a user
specified set of complex-species to an initial list of known
complex-species and reactions.  Upon initialization, Moleculizer uses
the rules to calculate the reactions that can occur between those initial
complex-species, and adds the generated reactions to its list of
known reactions.  Additionally, it adds any products calculated to be
generated as products of those reactions into the list of known species,
as an {\it unexpanded} complex-species.

As it runs, Moleculizer has a list of species and reactions it
knows about that make up the generated network.  However, this
generated network usually isn't complete, meaning that that
Moleculizer can still generate additional species and reactions using
the rules, but has not yet done so.  

Each species generated by Moleculizer is either in a state of having
been expanded or it has not yet been expanded.  Expanded species are
species that have been analysed by Moleculizer for matching the rules
in the model.  When a species is expanded, its free binding sites are
matched against the free binding sites on the already expanded
species.  If any of the new free binding sites pair up with any of the
already expanded complexes, libmoleculizer takes the two species and
adds them as the two substrates into a new association reaction.  It
forms the product by physically binding the two species together and
examining the result.  If the result is a structurally new species,
moleculizer adds it as a new but unexpanded species to the list of
species and reactions.  Regardless, the reaction is added as a new
reaction to its lists.  

Transformation reactions work similarly.  When a species is expanded,
it is checked to see if it contains any of the sub-complexes needed for
any of the transformation reactions.  If it contains any of them, a
new reaction is created, with the matching species as its substrate.
The transformation is performed on that species to create the product
species.  If the product species has not been seen by moleculizer, it
is added as a new but unexpanded species.  Regardless, the reaction is
added as a new reaction.  

Thus the process of expanding the reaction network is a matter of
taking an initial set of species (by default moleculizer starts with
the set of species consisting of a monomeric molecule and any
explicitly defined species)  and iteratively expanding them one by
one.  

By expanding them, libmoleculizer checks them for interesting
{\it ``features''} that can participate in a reaction, as defined in the
rules.  If libmoleculizer finds they have features that match rules in
the model, perhaps in combination with other, already expanded
species, libmoleculizer generates a new reaction, with the newly
expanded species as one of the substrates.  Depending on
circumstances, the product(products) of the reaction may also be new, in 
which case it is added as a new, unexpanded species to the network.  

Expanding the reaction network fully is simply a matter of expanding
unexpanded species until there is nothing left to expand.  


\section{Biochemical networks consist of chemical species and sets of
  chemical reactions between them}

The libmoleculizer library takes a convenient description of a
biochemical reaction network that includes descriptions of the
network's constituent proteins as well as {\it rules} defining
protein-protein and enzymatic interactions between those proteins.
Libmoleculizer uses this description to generate all or part of the
network, emitted as a list of variables (chemical species types) and
equations (describing reactions between variables) which can be used
in other simulation systems to simulate the network.  Emitted
Libmoleculizer reaction equations are compatible with all common
simulation formalisms, including differential equations, Gillespie
solvers, particle-based simulators.

Importantly, libmoleculizer is organized as a library, which means
that the ability to convert a human-readable rule-based model into the
more conventional species/reaction lists needed to simulate them can
be added to any simulator.  These simulators can expose rule-based
modeling to their users by using libmoleculizer internally to generate
all or part of the network and add the equations generated to their own
structures.  The equations can either be generated and added all at
once, before the simulation begins, or can be generated and added
periodically, using an {\it on-the-fly} generation scheme, in which
only enough of the network is generated at any moment to get through
the next time steps.

This invites the question of how libmoleculizer actually
generates reaction networks from systems of rules. Generally speaking,
once a rules file is loaded into libmoleculizer, libmoleculizer
creates an initial list of species and an empty list of reactions.
The initial list of species consists of all the monomeric protein
species for all proteins described in the network, as well as any
other species explicitly defined by the user.  Libmoleculizer also
records each of the species in the initial species list as {\it
  unexpanded}.  At this initial moment, libmoleculizer has not yet
generated any reactions.

In an iterative process, either automatically-directed (used in
pre-generative schemes) or user-directed (as used in on-the-fly
generation), libmoleculizer goes through the list of species it knows
about and {\it expands} them one at a time.  What this means is that
libmoleculizer takes that species and compares it against each of the
rules, checking for the correct {\it features} needed by that
rule.  In the case of binary reaction rules, the species is checked
against both the rules as well as features that have already been
found in other already expanded features.  When the expanding
species matches a rule (or a rule and a second species, in the case of
binary reaction rules).  The matching species is taken, possibly along
with its already expanded mate, and the transformation defined in the
transformation is performed upon the species ( for instance, a kind of
enzymatic reaction may look for molecules that have been singly
phosphorylated; the transformation could turn the singly
phosphorylated modification to a doubly phosphorylated
one). Libmoleculizer then examines the product(s).  If any of them are
new - not in the list of species libmoleculizer knows about - they
are added there as a newly generated, unexpanded species.  The
reaction is added as a newly generated reaction to the system.  As
libmoleculizer expands species, the network grows; if and when there
are no more species to generate the entire network has been generated
and libmoleculizer halts.  


At any time during this process, either the complete lists of
species/reactions or partial lists of recently created
species/reactions can be read out by client programs.  Additionally,
libmoleculizer also records and manages other facts about the network,
such as keeping track of sets of species that have certain structural
properties, which can also be used in client programs to add features
for their users.

\section{Biochemical networks consist of chemical species and sets of
  chemical  reactions between them}
Libmoleculizer takes in models that represent an abstraction of a
biochemical network and explicitly generates all or a part of that
network.  What this means is that libmoleculizer generates an explicit
list of species along with all reactions those species can participate
in.  

This list of species and reaction can be sent to another simulator, as
a list of variable names and reaction equations between them, and can
be used to simulate the portion of the network that was expanded.
This is what is meant when we say libmoleculizer expands biochemical
reaction networks: it uses a set of rules to generate lists of species
identities and of reactions between species.

\section{Species are complexes of one or more indivisible units bound together}

Species in libmoleculizer are constructed by combining together one or
more indivisible units called {\it molecules}.  Molecules have binding
sites, and are joined together by creating bindings between their
binding sites.  

When one or more molecules joined together, a structurally unique
{\it complex species} is formed, which has a unique identity as a biochemical
species in that reaction network.

\section{Molecules represent indivisible units of reaction}
As discussed previously, molecules are the indivisible components of
complex-species.  Reactions can have structural effects like binding two
unbound molecules together, breaking bonds between molecules, or
exchanging one molecule in a complex species for another, but
molecules cannot be be subdivided or combined. 

\subsection{Molecules have a structure consisting of binding sites and
  modification sites}

Moleculizer recognizes two types of molecules.  The first type are the
{\it modifiable-molecules}.  These molecules may have any number of unique
binding sites, each of which may participate in binding this molecule
to other molecules.  Modifiable-molecules may also have any number of
unique modification sites, which can each be associated with different
post-translational modifications e.g phosphorylation, methylation, etc.

The second type of Moleculizer molecules are the {\it small-molecules}.
These are molecules that can participate in reactions but which only
have a single non-differentiable binding site (which is considered to
be structurally indistinguishable from the small-molecule itself) and no
modification sites.

Typically, modifiable-molecules are appropriate structures for
modeling individual proteins whereas small-molecules are appropriate
for representing small molecules such as GTP and ATP.

\section{Association and dissociation reactions between complex
  species are caused by the associations and dissociations on the
  binding sites belonging to the molecules that make up that complex}

In the libmoleculizer system, all dimerization reactions between
complex species result from an association reaction between two
binding sites on two types of molecules, one belonging to each of the
two complex species.  

These association reactions between molecules are recorded in the
network model as an {\it ``association-reaction''} rule.  As libmoleculizer
generates new complex species by expanding the network, when it
notices pairs of species with free binding sites corresponding to the
rules, it ensures that the dimerization reaction is generated between
them.

At the time the reaction is created, it is also reversed and the
decomposition reaction is also created.  

\section{Species transformation reactions are made possible by the
  species possessing a ``transformation-enabling'' sub-complex}

The second kind of reaction supported by libmoleculizer is very broad:
the transformation-reactions, which are generated by {\it
  ``transformation-reaction''} rules.  Transformation reactions are
those that involve (generally) enzymatic function.  Complex species
can have any number of modifications changed and any number of
small-molecule swaps (e.g. an ATP turning to an ADP) as desired in a
transformation reaction.  Any reaction in which a species undergoes a
post-translational modification or has one small-molecule exchanged
for another will be a transformation reaction.

Libmoleculizer generates these reactions by looking for complex
species that have a particular {\it transformation-enabling complex}, as
specified by the user in different ``transformation-reaction'' rules
in the model. (For instance, a transformation-enabling complex may be
an enzyme bound to a target which itself is bound to an ADP
molecule).  When libmoleculizer finds a species that matches, it
transforms it according to the rule provided by the user (possibly
creating a new species in the process) and adds the new
reaction to the list.  

Depending on how the user defined the rule, optional extra reactants
or optional extra products may be generated at this time.  

\section{Allosteric reaction rates are differing rates of association
  and dissociation among molecules, conditional on the states of
  the complexes the reaction molecules are found in}  

One complicating but absolutely key feature observed in real
biochemical networks is allosteric binding and unbinding.  What this
means is that, generally speaking, particular association reactions (a
particular affinity between two binding sites on two molecules) occur
at an 'intrinsic' rate which reflects the physical relationship
between the two participating binding domains.  

However, when the species in which the associating molecules live are
in certain specific forms, the default rates of interaction change
dramatically. This is typically for one of two reasons.  Either
another one or more molecules in the complex have blocked one of the
participating binding sites; or the identities of other molecules in
the complex have resulted in a conformational change in shape to one
of the participating binding sites.  This conformational shape change
causes the physical relationship between the two binding sites to
change, resulting in a change to the intrinsic reaction rate.  

In libmoleculizer, allosteric interactions are modeled first by
allowing users to define binding sites with multiple states, called
shapes, which represent different conformational states, resulting in
different kinetic profiles.  In a section of the model where where
allosteric conditions are listed, any number of conditions that cause
the binding site to go from its default shape to one of its
non-default allosteric state can be defined.

Finally, when entering the association interactions involving binding
sites that have different conformational shape states, a default rate
-- for the default shape -- is given, followed by any number of
non-default allosteric rates.

In this way, any degree of ``arbitrary'' kinetics can be added to a
moleculizer model.

\section{Names in libmoleculizer}

Libmoleculizer generates explicit species and reactions listings of
biochemical reaction networks, which can be read out with the
libmoleculizer API to use in other programs for other purposes.  

Some words, however, should be said about the facilities for naming
species that exist in libmoleculizer, because otherwise difficulties
arise.  

The problem with naming complex species is that they are just
represented internally as sets of connected parts.  Because the sets
of parts (molecule types and bindings) have no preferred ordering to
them, there isn't any one name for the species -- in fact there are at
least as many names for a species as there are ways of describing its
components ( simply, if a species consists of $x$ molecules, 
there are at least $x!$ ways of describing the species).  

First, what IS NOT the problem.  For any species, we can get a name
that describes the species.  This is never a problem.  We can always
take a name we've been given and get back the same species every
single time.

What is the problem?  The problem is that because species have so many
names that can describe them, we might look at two identical species,
and generate names for the two of them, and, because the names are
different, conclude the two species must be different.  Even though
the two species are really the same, they could have different names.
The problem we have is in uniquely identifying complexes -- with the
problem of making sure that a complex has a uniquely identifying
name.  

Because of this, libmoleculizer has two solutions.  The first is a
unique, temporary name, which is called a ``Tag'' or ``Tagged Name''
in other parts of the manual.  This ``Tag'' is unique over a
particular run of libmoleculizer.  As long as tags are only compared
during a single program run they are your best option. However,
because they are only unique for a given run, they should never be
used to compare species generated during different runs.  For this, we
must use something more permanent.  

The second solution is a permanent name, which we call a ``Unique
ID''.  The unique id is a name that persists beyond program runs.  An
identical species will always generate the same unique id, no matter
what.  Accordingly, they are good for activities where biochemical
networks are saved to disk, and need to be compared between simulation
runs.  The downside to using unique ids is that they are slower to
generate, when compared with tags.  Thus, tags are preferred in all
situations in which they can be used.
