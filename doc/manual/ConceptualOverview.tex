\chapter{Comceptual Overview of libmoleculizer}

The objective of libmoleculizer is to take a description of protein
and other molecular entities, as well as a set of descriptions of
various biochemical interactions, and use it to generate information
in a variety of forms about the explicit reaction netowrk and all the
species within it.  

In order to understand how libmoleculizer generates reaction networks,
including their kinetics, it is importand to cover some background
information on the kinds of interactions libmoleculizer represents and
the way that libmoleculizer expands reaction networks. 

\section{Combinatorial Reaction Networks}

\subsection{Signal Transduction Pathways}

\section{Rule-Based Reaction  Network Specifications}

The fundamental idea behind libmoleculizer is that reaction rules can
be applied to chemical species in order to possibly generate new
species.  

For example, a reaction rule might be represent a protein-protein
dimerization.  If a model has defined a Ste5 protein that has a
binding site 'ste5-binding-site1' as well as a Fus3 protein that has a
binding site 'fus3-binding-site1', a reaction rule can be specified
that says Ste5 can bind to Fus3 from the 'ste5-binding-site1' to the
'fus3-binding-site1' at a particular on-rate $k_{on}$, and a particular
off-rate $k_{off}$.  This rule will be applied in two ways.  First,
any time a pair of molecular complexes, the first of which contains a
Ste5 molecule with a free 'ste5-binding-site1' site, the second of which
contains a Fus3 molecule with a free 'fus3-binding-site1', this rule
will be applied to generate an explicit dimerization reaction between
those two complexes.  At the same time, the off-rate provided will be
using to generate an explicit decomposition reaction for the dimerized
complex.  

The on- and off- rates used in reaction rules are used to generate
appropriate on- and off-rates in the explicit reaction.  This process
can be configured in a variety of ways.  Among other things, values
can either be passed directly through to explicit reactions, or
reaction rate extrapolation can be used, where libmoleculizer will
approximate explicit reaction rates for the generated reactons, based
on the kinetics of the rule, as well as the masses of the substrate
species.  Please see the section dealing with Reaction Rate Kinetics
for more infomation.  

\section{Allostery Based on Binding Shape}

Biochemical reaction networks, particularly signal transduction
pathways, will oftentimes regulate themselves using allosteric
proteins.  Allosteric proteins are proteins that have functional sites
(binding sites) and regulatory sites.  Typically different small
molecules are bound to the regulatory sites, and this induces
conformation changes to to functional site, which modify its
functional kinetics.  

At the physical level, states of the regulatory sites in a complex
species affect the conformation of functional binding sites, and
presence and state of subcomplexes within a complex may affect other
kinetic properties of functional sites (e.g. a particular protein
being bound at a particular binding site may physically block a
functional site, preventing it from being used -- in affect, turing
all the rates to 0 for that binding site).  MZR rules use the idea of
binnding-site shape, a comcept that encompasses both allosteric
changes to conformation, as well as other non-conformational physical
effects that affect reaction kinetics.  

Each binding site within libmoleculizer must be defined with a default
binding-site shape, as well as a list of the other shapes that can
occur at that binding-site. In another section of the rules,
the conditions under which complex state induces binding site shape
changes are specified.

Finally, in reactions that occur between functional sites, allosteric
rates are specified in terms of the relevant binding site shapes.  

\subsection{An Argument For Using Binding Shapes to specify allostery,
  over other Methods}




\section{Reaction Rate Kinetics}



\subsection{Mass-based reaction rate extrapolation}



\section{Unique Complex Naming}