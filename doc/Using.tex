\chapter{Using Libmoleculizer}
\label{chap:usingLibmoleculizerChapter}

Libmoleculizer is used by providing it a list of rules that describe
proteins and small molecules as well as rules that describe
protein-protein reactions, enzymatic reactions, protein-small molecule
interactions, and then using the various functions in its API to tell
it to generate all or part of the 

These rules must be written in either xml format or text file format
in the mzr format described in chapter \ref{chap:theRulesChapter}.  

Once rules have been provided to libmoleculizer which describe the
reaction types, there are several ways in which libmoleculizer can be
used to describe the resulting system.  

Moleculizer has two methods for extrapolating networks (that is to
say, methods for expanding the reaction network and informing clients
as to what is in the network) and two methods for extrapolating
parameters within the network.  

\subsection{Choosing an Extrapolation Method}

Libmoleculizer allows spatial and non-spatial extrapolation methods.
For the time being, the simplest way to chose an extrapolation method
is to add a <runtime-info> section to the <moleculizer-input> portion
of the rules.  Within that section, add one of the following sections: 

1.
\begin{verbatim}
<generation-method mode=''non-spatial''\>\<extrapolation-method
mode=''extrapolate-reaction-rates''/>
</generation-method>
\end{verbatim}


2.
<generation-method mode=''non-spatial''><extrapolation-method
mode=''no-extrapolate-reaction-rates'' /></generation-method>

3.
<generation-method mode=spatial'' />


\subsubsection{Non-spatial Species and Reaction Generation}
\subsubsection{Spatial Species and Reaction Generation}

Preconditions:



