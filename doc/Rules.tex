\chapter{The MZR Language for expressing rule-based models}
\label{chap:theRulesChapter}
A libmoleculizer input file is called a Moleculizer Rule File (MZR).
MZR files are text files that contain a model of a particular
biochemical network, written in a language that describes a set of
molecule types, a set of reaction rules describing interactions
between those molecular types, along with other information needed to
describe these types of systems.  This rule-based representation is
used by libmoleculizer to create the explicit species and reactions
that make up the expanded network.

This chapter describes the MZR file syntax, and tells how to use it to
write rule-based models representing virtually any biological pathway
that can be expanded into explicit lists of species and reactions that
comprise the pathway.  

\section{MZR File Overview}
A MZR file consists of multiple sections, each of which describes a
portion of the information needed to represent a biochemical reaction
network in libmoleculizer.  There are sections for describing
different molecule types, types of relevant post-translational protein
modifications, kinds of reaction families, allosteric conditions,
families of species and more.

When a MZR file is supplied to libmoleculizer, it is compiled into an
intermediate xml format, which is subsequently used internally by
libmoleculizer.  For more information on the internal format either
consult other parts of this manual (the information should be in here
somewhere), or consult the original moleculizer documentation, which
has a full discussion of it.

\section{MZR File Syntax}
A MZR rules file consists of one or more of the following sections:
``Modifications'', ``Molecules'', ``Explicit-Allostery, ``Allosteric-Classes,
``Association-Reactions'', ``Transformation-Reactions'', ``Species-Classes'', and
``Explicit-Species''.  Most of these sections are optional.

Each section must begin with a header declaring the name of that
section.  A header is any line that begins with the string ``==='' and
contains the exact type-sensitive name of that section somewhere in
the line.  Although this format may sound unusual, the end result is
that a typical MZR file will have a top-down structure that looks like
the following example.

\begin{ExampleMZR}[title={the caption}, caption={a caption}]
===== Modifications =====
  # Modification definitions...

===== Molecules ==============
  # Molecule definition...

=====  Association-Reactions =====
  # Dimerization type reaction definitions....

# And so on....
\end{ExampleMZR}
 
Each section is comprised of one or more statements, each of which
adds a single piece of information to the biochemical network
model. (There is a technical exception: many statements are not truly
standalone, beause they implicitly require the existance of another
standalone statement in the model in order to be complete.  An example
could be a rule defining a protein interaction resulting in a
posttranslational modification to a target protein.  While the
reaction rule itself is essentially standalone, it requires a
statement defining the modification in question to be located in the
Modifications section in order for the reaction rule to be completely
understood by libmoleculizer.)

In a MZR file, statements are seperated by semi-colons; all other
white-space within them, including newlines is ignored.  Each statment
consists of a comma-seperated list of substatements, which can either
describe complexes or parts of complexes (what are referred to herein
as complex forms), give assignments, or define reaction rules
patterns.  What this actually means and how it is used will
become clear over the next few sections.  For now, understand that
every statement ends with a semi-colon, and consists of one or more
comma-seperated components.

Finally, comments are permitted, and use the comment-start character
``\#''.  Comments extend from their beginning to the end of the line on
which they appear.

\subsection{Modifications}
The modifications section consists of a set of statements that each
defines a unique modification that can be used and referred to within
the model.  Each such definition consists of either one or two
components, each of which takes the form of an assignment.  The first
assignment is mandatory and gives the modification being defined a
name; it takes the form ``name=MODIFICATION\_NAME''.  If mass-based
reaction rate extrapolation is used, an additional, otherwise
optional, assigment of ``mass=MASS\_OF\_MODIFICATION'' is needed as a
part of the modification definition statement as well.

The first example gives an example of a modification section in a
model that does not use reaction rate extrapolation (the default).

\begin{ExampleMZR}
=== Modifications =====================
     name = None;
     name = Phosphorylated;
     name = DoublyPhosphorylated;
     name = GDP;
     name = GTP;
\end{ExampleMZR}

The next example gives an example of the same modification section,
but which now supplies mass assignments as well.  Notice how within
each statement, the two component substatements (the name assignments
and the mass assignments) are only seperated with commas.

\begin{ExampleMZR}
=== Modifications =====================
     # If these modifications are to be used in reactions using mass-based
     # rate extrapolation, then must be given masses.
     name = None, mass = 0.0;
     name = Phosphorylated, mass = 42.0;
     name = DoublyPhosphorylated, mass = 84.0;
     name = GDP, mass = 100.0;
     name = GTP, mass = 110.0;
\end{ExampleMZR}


\subsection{Molecules}
The molecules section consists statements that each defines either a
mod-mol or a small-mol in the model.  Both types of molecules correspond to
types of indivisible physical units that participate in reactions; the
difference between the mod- and small- mol types lies in the level of
resolution of structural detail they are given, and resultingly in the
kinds of reactions they can participate in. Mod-molecules are structural
molecules that have one or more binding sites and zero or more
modification sites; they also may have masses, which can be used for
mass-based reaction rate extrapolation if desired.  Small-molecules are
molecules that participate in binding reactions with other molecules, but which
do not themselves have differentiated binding sites or any
modiification sites at all. However, like mod-molecules, they may have
a mass if mass based reaction rate extrapolation is used.

Typically mod-molecules are used to represent biochemically active
proteins, whereas small-molecules are used to represent small molecules.  

\subsubsection{Statements defining Mod-Molecules}
Each statement that defines a mod-mol is made up of a complex-form
specification that defines it, followed by an optional weight
assignment (as always, weights are only mandatory if mass-based
reaction rate extrapolation is used).

The meat of the definition occurs in the specification of the mol's
complex-form statement: a complex form that defines a mod-mol consists
of the mol's name followed by a parenthesis enclosed description of the
structure of that mol -- its binding sites and its modification sites.

Within a complex-form, each binding site can simply be defined by
writing the name of that binding site in the parenthesis.  If this is
all the information provided, the binding site will be created with
one and only one binding site shape, called ``default''.  If the
binding site exhibits biochemically allosteric behavior (if it
displays multiple distinct profiles of kinetic behavior, depending on
conditions, allosteric states must be specified here (for instance, a
binding site may be more or less active depending on whether or not
other binding sites are bound or not, or depending on the state of a
particular modification site on that mod-mol, or depending on what
kind of complex the mod-mol might be in).  To do this, all the legal
binding site shapes are declared in a comma seperated list inside a
pair of braces immediately following the name of the site.  The first
value is assumed to be the default shape for the binding site.

Each modification site consists of the modification site's name,
proceeded by a * character (see example 4 in Fig
\ref{examplemodmolssec}) for an example).  The name of the
modification site must be immediately followed by a pair of braces
containing a list of modifications the modification site can have.
The first in the list is assumed to be the default modification value
for that modification site.  Moreover, each modification type referred
to by the list must have a corresponding definition in the
modifications section.  These references, like everything else in
these files, are case-sensitive.

\begin{ExampleMZR}[label=examplemodmolssec]
==== Molecules ==============
    # 1. Defines a mol named Ste20 with a single binding site named
    # 'to-Ste4'. 
    Ste20(to-Ste4),
        mass = 100.0;

    # 2. Defines a mol with four binding sites.
    Ste5( to-Ste4, to-Ste11, to-Ste7, to-Fus3),
         mass = 100.0;

    # 3. Defines a mol with three binding sites.  The third binding site,
    # to-Ste20, is defined with two possible shapes: the default
    # shape, named 'default' and a second shape called 'obstructed'.
    # In the allosteric sections, conditions under which the binding
    # site can be in either state can be described, and reaction rules
    # involving the to-Ste20 binding site can be given different rates
    # depending on the state of to-Ste20
    Ste4(to-Gpa1, to-Ste5, to-Ste20 { default, obstructed} ),
         mass = 100.0;

    # 4. Defines a mol with a single binding site and a single
    # modification site (modification sites begin with *'s). The
    # modification site, called ``PhosSite'' can have three different
    # modifications, a default modification named ``None'', as well as
    # modifications called Phosphorylated and DoublyPhosphorylated.
    # These modifications must be defined in the modifications section.
    Ste7( to-Ste5, *PhosSite { None, Phosphorylated, DoublyPhosphorylated } ),
         mass = 100.0;

    # 5. Defines a mod-mol with two binding sites, the first of with with
    # shape information, and one modification site.
    ReceptorGpa1Complex(to-Ste4 { GDP-bound-shape, GTP-bound-shape},  
                        to-alpha, *GXP-site { GDP, GTP} ), 
   	mass = 100.0;
\end{ExampleMZR}

\subsubsection{Defining Small-Molecules}
Each small-mol statement consists of a name assignment for the new
small-mol, as well as a mass assignment (although only needed if
mass-based reaction rate extrapolation is used).  

Several examples of small-mol statements are provided below.  
\begin{ExampleMZR}
  ==== Molecules ==============
     # If no mass is specified
     massless_alpha; 
     alpha, mass = 42.3;

\end{ExampleMZR}

Mod-mol defining statements and small-mol defining statements can be
freely mixed within the Molecules section.

\section{Defining Allosteric Conditions}
Allosteric behavior is represented in libmoleculizer as differering
rates of interaction within dimerization reactions, conditional on the
specific shapes of interacting binding sites.  Specification of the
conditions under which allosteric binding sites transition between
their different shapes occurs in the two allosteric sections:
allosteric-plexes and allosteric-omnis.  Both these sections are used
for the same purpose: to give structural conditions such that binding
sites of interaction change state.  (The different states on a binding
site of interaction are used in the rules defining those interactions
to specify different interaction strengths depending on what shapes
the binding sites have).  The difference between the two sections is
in where libmoleculizer looks for the structural conditions.  That is,
in both sections statements are given that describe two things: a
biochemical entity in a specific enabling state, and one or more
resulting binding site shape changes the binding sites within that
entity.  The difference between the two is that in the
allosteric-plexes section, the structural entity specified much match
exactly; in the allosteric-omnis section, the structural entity
specified matches all complexes which contain the specified entity.
(They may either match it exactly or the specified entity may be a
subcomplex of the biochemical entity being matched.)

\section{Allosteric-Complexes}
An allosteric-plex is a rule that tells how the binding sites within
specific complexes (specific up to state of its unspecified
modification) have their shapes changed.  Binding site shape
information may then be used for specifying allosteric binding rates
within the dimerization-gens section.

Each statement in this section consists of a single complex-form
specification of a whole complex, that includes special syntax to
describe how the binding sites in that complex change their shape.

The first step in the construction of the appropriate
allosteric-statement is the construction of the complex-form that
describes the relevant complex.  This is done by specifying the
complex as a set of molecules, with binding-site binding information
joining binding sites between molecules.  


These are just examples of complex-forms.  Because they do not yet use
the specical allosteric syntax that describes updated binding sites,
they are not yet valid allosteric-plex statements.

\begin{ExampleMZR}

A;  # This is the complex form that describes a particular singleton
    # mol.

A(first!1).B(second!1); # This is a complex form that describes a
                        # particular dimer of an A bound to a B, from
                        # the A's ``first'' binding site to B's
                        # ``second'' binding site.

# This complex describes a B-A-C trimer.  Looking at the binding
# information, we see that A is bound at its ``first'' site to B at 
# its ``b-site''.  Likewise A is bound at its ``third'' site to the C 
# at the C's ``c-site.''
A(first!1, third!2).B(b-site!2).C(c-site!1);

\end{ExampleMZR}


Although these complex-forms are syntactically correct (although they
do not yet contain enough information to be valid allosteric
statements, or any other valid statement at this moment), they may not
be valid complex-forms.  To be valid, the objects referred to be be
consistent with other parts of the model.  By referring to A, B, and C
molecules, these each must be defined in the ``Molecules'' section of the
model.  Furthermore, all three must be mod-molecules and be defined with
the appropriate binding sites: A must have ``first'' and ``third''
binding sites and so forth.  

By getting this far, in the context of an allosteric-plex, we have
completely specified a complex type.  (In the examples, these are A
singletons, AB dimers, or ABC trimers respectively).  That is, they
completely specify the {\bf the binding state} of the complex, they
are so far silent on the modification state of the complex.  If, for
instance, the A molecule has a single modification site, which can
either be phosphorylated or not, then AB dimers (complex-type) can
have two forms of corresponding species: the AB dimer with the A
phosphorylated and the AB dimer with the A unphosphorylated.  To
specify further conditions on the modification state of the specified
complex, further modification site information must be known.  

\begin{ExampleMZR}
  === Modifications ===
  name = None;
  name = P;

  ==== Molecules  =====

  A(to-b,*M{None,P}};
  AP( to-b, *M1{None,P}, *M2{None,P});
  B(b);

  ==== Allosteric-Complexes =====
  
  # Because no mention is made of A's modification site ``M'' here, 
  # this allosteric-plex will match AB dimers in which A is both 
  # phosphorylated and non-phosphorylated.
  1.  A(to-b!1).B(b!1);
  
  # Because this complex specifies a state for the modification site
  # in the A mol, this complex will only match AB dimers in which the
  # A is unphosphorylated.
  2.  A(to-b!1, *M{None}).B(b!1);

  # Like 2, but will match only AB dimers in which the A is
  # phosphorlated.  
  3.  A(to-b!1, *M{P}).B(b!1);

  # Partial specification is allowed (as well as full and no
  # specification), this will identify all AP.B dimers in which 
  # the M2 phosphorylation site on the AP is phosphorylated.  The 
  # M1 site may be in either phosphorylation state to match.
  4.  AP( to-b!1, *M2{P}).B(b!2);

\end{ExampleMZR}

At this point, any specific structural form can be spefified, with any
level of selection on the modification state of that complex.  To
completely specify an allosteric-plex, only one thing remains: to
specify the binding-site shape changes that occur for complexes that
match the specification.  

This is accomplished by specifying the binding-sites in question by
making sure to include them in the complex form specification and then
following them with the allosteric sections' special syntax:
``{FinishingState <- *}'', where FinishingState is the specific name
shape the associated binding site ends up in.  

\begin{ExampleMZR}
=== Allosteric-Complexes =======

  # 1. This rule says that when there is an AB dimer, the ``to-c''
  # site on the A will be occluded.
  A(to-b!1, to-c {occluded<-*}).B(to-B!1)

  # 2. This rule states that when an A is doubly phosphorylated, its
  # ``binding-site'' ends up in the active shape/state.
  A(*M1{P}, *M2{P}, binding-site {active<-*});

  # 3. This rule states that when there is a ReceptorGpa1Complex bound to
  # an alpha and also bound a Ste4 exactly, then the to-Ste20 site on
  # the Ste4 in that complex takes the shape obstructed.
  ReceptorGpa1Complex(to-alpha!1, to-Ste4!2).alpha(!1).Ste4(to-Gpa1!2,to-Ste20 { obstructed <- * });

  # 4. This rule says that when there is an B-A-B complex, the
  # primary-site binding sites on the B's become active.
  A(left!1, right!2).B(to-A!1, primary-site {active <-} ).B(to-A!2, primary-site { active <- *);

\end{ExampleMZR}


\section{Allosteric-Omniplexes}
This section is virtually identical to the allosteric-plexes section,
with one key conceptual difference.  In the allosteric-plexes section
the complex forms specified correspond to structurally identical
(consisting of the same molecules in the same binding configuration)
complexes.  In the allosteric-omnis section, the complex forms match
classes of complexes who possess the subcomplexes specified.  See the
example below for more examples.


\begin{ExampleMZR}

# As an allosteric-plex, the following phrase will match singleton
# A's.  As an allosteric-omni, the following phrase will match all
# complexes that contain an A mol.
A; 

# As an allosteric-plex, the follwing phrase will match all A-B dimers 
# with a phosphorylated A.  As an allosteric-omni, the following will
match all complexes that contain a phosphorlyated A bound with a B.
A(to-b!1, *M{Phos}).B(to-a!1);

\end{ExampleMZR}

Essentially that is the only difference between an allosteric-plex
statements and allosteric-omni statements: that of how to interpret
them (which is also why they need seperate sections).  However,
because complex-forms in allosteric-omnis select for classes of
structural complex, there are two additional minor forms of
specifiers allosteric-omni complex-forms can use, that allow specifying
whether a binding site is or is not bound.  

To specify a site is explicitly unbound, simply list the binding site in the
complex form specification.

\begin{ExampleMZR}
==== Allosteric-Omniplexes =====
  # This specifies all complexes that contain an A bound to B.
  A(to-b!1).B(to-a!1);

  # By adding an otherwise empty binding site ``to-c'' in the
  # structural specification, we indicate that this specifies all
  # complexes that contain an A bound to B whether the A has an unbound
  # ``to-c'' site.
  A(to-b!1, to-c).B(to-a!1);


  # By adding a partial binding specification(an exclamation point
  # indicating a bound state, followed by a * which matches anything),
  # this specification will select all complexes that contain an A
  # bound to a B where the A's binding site ``to-c'' is also bound.
  A(to-b!1, to-c!*).B(to-a!1);

  # Note that by default the following complete rule does not require
  # that the ``to-c'' site which changes shape must be unbound.  
  A(to-b!1, to-c {occluded <- *}).B(to-A!1);

  # If we wish to force this constraint, we must explicitly force the
  ``to-c'' site to be unbound, using the ``!-'' syntax..
  A(to-b!1, to-c!- {occluded <- *}).B(to-A!1);

  # We can also combine syntaxes.  This example shows how binding-site
  # shapes can be updated on bound-and-specified(the !1 on A),
  # bound-and-not-specified (the to-something) on C, unbound (unbound on
  # B), or unspecified (default) on B) within a rule.

  A(to-b!1 {state1<-*}, to-c!2, fizzy) 
  .B(to-a!1,unbound!- {state2 <-* }, default {state3<-*} ) 
  .C(to-b!2, to-something!* {state4});

\end{ExampleMZR}

One final note is that all specified alloseric-omni complexes must be
simply connected.  That is, everything specified must form one contiguous
subcomplex in which every mol specified is unique.


\section{Explicit-Species Section}
The explicit-species section is where user specified names are
assigned to different non-trivial (more than one mol component)
complex species.  Each statement within this section assigns a
specific complex a name.  These names are recorded by libmoleculizer
and can subsequently be looked up and referred to.

The first component of an explicit-species statement is a complex-form
that specifies the complex-species in question.  Because
explicit-species must be specifies completely this means that every
modification state for every modification site on every mol must be
specified.  

To write the complex-form that represents the species in question,
join the molecules with periods.  For any binding between two species,
append a binding id, consisting of a ! followed by an index unique to
that binding, to the name of that binding site.  If one of the molecules is
a small-mol, simply write the associated binding syntax inside its
structural specification (the parentheses).  Finally each modification
site that exists in the complex must be written along with its
modification.  An example is given in Fig
\ref{exampleexplicitspecies}.

\begin{ExampleMZR}[label=exampleexplicitspecies]
=== Explicit-Species ====================

    # The following explicit-species statement represents a
    # ReceptorGpa1Complex bound to a Ste4, from the
    # ReceptorGpa1Complex's to-Ste4 site to the Ste4's to-Gpa1 site.
    # Additionally, the ReceptorGpa1Complex's GXP modification site
    # should be bound to GDP.
    ReceptorGpaComplex(to-Ste4!1, *GXP-site { GDP } ).Ste4(to-Gpa1!1), 
        name = ReceptorBoundGpa;

    # The following shows how to represent a small-mol (alpha) binding as part
    # of a threeway complex.
    alpha(!1).ReceptorGpa1Complex(to-alpha!1, to-Ste4!2, *GXP-site {GDP} ).Ste4(to-Gpa1!2),
        name = alpha-receptor-ste4-complex;
\end{ExampleMZR}


\section{Explicit-Species-Class}
Statements within species-stream section have a similar relationship
to statements in the explicit-species section as statements in the
allosteric-omnis section have to those within the allosteric-plexes section.

In the explicit-species section, statements give an exact
specification for a complex and associate a name with that exact
complex.  In the species-streams section, statements describe classes
of complex and associate a name with the class.

As before, the first component of the statement is a complex-form
specification.  However, the additional syntax and meaning that was
allowed and present in the allosteric-omnis section tends to apply
here.  For instance, to require a binding site as being unbound as a
condition of class membership, write that binding site out in the mol
it is contained in.  To specify that a binding site is bound as a part
of an unknown binding, write out the name of that binding site
followed by the ``!*'' syntax. Likewise, modification values at
modification sites may be specified or not, depending on user
preferences.  

The species-stream complex-form specifier should be followed by an assignment
``name=YOUR\_NAME\_HERE''.

\begin{ExampleMZR}
===== Explicit-Species-Class ============
# This will select for all complexes that contain a phosphorylated A.
A(*M{Phos}), 
       name=withaphosphorylated\_A;

A(*M{Phos}), 
       name=withaphosphorylated\_A;


A(*M{Phos}), 
       name=withaphosphorylated\_A;


A(*M{Phos}), 
       name=withaphosphorylated\_A;


A(*M{Phos}), 
       name=withaphosphorylated\_A;

\end{ExampleMZR}


\section{Association-Reactions}
The dimerization gens section consists of a set of statements, each of
which describes an association/decomposition reaction between specific
binding sites on specified molecules.  In addition to having default on and
off rates, the association and decomposition rates may be made
dependent on the states of the specified binding sites.  This
allosteric behavioral information is specified as further
subcomponents of the main rule statement.  

A dimerization-gen statement consists of one or more components.  The
first section describes the specific form of the reaction followed by
default on and off rate information.  Subsequent components describe
allosteric behavior and, for each allosteric condition, add three
components. In these cases, the first portion describes allosteric
states of binding sites participating in the first, master reaction
and the second and third portions describe the on and off rates for
that allosteric case, and so on.

As always, components referred to within dimerization-gens statements
must have corresponding definitions in other parts of the model.  For
instance, Molecules referred to in the reaction specification must be
defined in the Molecules section, along with the binding sites and binding
shapes referred to in the rules.  

\begin{ExampleMZR}

  # Basic Dimerization.
  A(to-B) + B(to-A) -> A(to-B!1).B(to-A!1),
       kon = 100.0,
       koff = 10.0;

    #
    # alpha/Receptor Binding.  This reaction shows how a binding of a
    mod-mol to a small-mol works
    alpha() + Receptor(to-alpha) -> alpha(!1).Receptor(to-alpha!1),
    	   kon = 100.0,
	   koff = 10.0;

    # The receptor@oligo-1 + receptor@oligo-1 reaction has on and off
    # rates of 10.0 when both receptors are in their default states.
    # When one of them is in state {enabled} and the other is in state
    # {default}, the on rate is 75 and the off rate is 10.0; when both
    # participating binding sites have shape {enabled}, the on rate
    # becomes 50.0.
    Receptor(oligo-1) + Receptor(oligo-1) -> Receptor(oligo-1!1).Receptor(oligo-1),
          kon = 10.0,
	  koff = 10.0,
       Receptor(oligo-1 { default} ) + Receptor(oligo-1 {enabled } )
         -> 
           Receptor(oligo-1!1).Receptor(oligo-1),
             kon = 75.0,
	     koff = 10.0,
       Receptor(oligo-1 { enabled } ) + Receptor(oligo-1 { enabled} ) 
         -> 
           Receptor(oligo-1!1).Receptor(oligo-1),
             kon = 50.0,
	     koff = 10.0;

\end{ExampleMZR}


 \section{Transformation-Reactions}
 The dimerization-gens section deals with all binding site interactions that bind
 or unbind proteins together.  A second important class of
 protein-protein reactions are the generic enzymatic reactions.  These
 are reactions in
 which post-translational modifications can be made and small molecules are
 exchanged, possibly with the aid of a helper molecule, and possibly
 resulting in an additional final product as well.

 The omni-gen rule represents a generic reaction that can generate,
 for each complex species containing a particular omniplex, a reaction
 with flexible characteristics. The generated reactions can change any
 small-mol component of the omniplex into another small-mol, if
 desired. The generated reactions can change the modification at
 modification site on any one mod-mol component of the omniplex, if
 desired. The generated reactions can be binary, if desired, requiring
 any explicit reactant species as a co-reactant with the species that
 is recognized as containing the omniplex. The generated reactions can
 produce an arbitrary explicit product species, in addition to the
 transformed primary reactant, if desired. 
 
 Each of these separate activities on the part of the generated
 reactions is engaged simply by including the appropriate elements in
 this reaction generator specification.
 
 The first half of an omni-gens statement, the reaction specification
 should be written in the form ``Omniplex + (optional\_explicit\_reactant) ->
 Transformed\_Omniplex + (optional\_explicit\_product)''.  Examples follow:

\begin{ExampleMZR}

==== Explicit-Species =====

A(to-b!1).B(to-a!1), name = AB_dimer;


==== Transformation-Reactions ============

# A simple phosphorylation reaction
E(b1!1).S(b1!1, *Mod{ None }) -> E(b1!1).S(b1!1, *Mod{ Phosphorylated }),
     k = 100.0;

# By listing a site, such as the b2 site on E here, and leaving it
# empty, this specification will only apply to those complexes with
# that site free.
E(b1!1, b2).S(b1!1, *Mod{ None }) -> E(b1!1).S(b1!1, *Mod{ Phosphorylated }),
     k = 100.0;

# Likewise, by adding the !+ syntax, you can force that a binding site
# simply must be bound.
E(b1!1, b2!+).S(b1!1, *Mod{ None }) -> E(b1!1).S(b1!1, *Mod{ Phosphorylated }),
     k = 100.0;

# The same phosphorylation, but this time adding a GTP -> GDP
# hydralization.  
E(b1!1,to-GxP!2).S(Sb1!1, *Mod{ None }).GTP(!2) ->
  E(b1!1,to-GxP!2).S(Sb1!1, *Mod{ Phosphorylated }).GDP(!2),
    k = 100.0;

# Multiple small molecules can be exchanged at once.
E(b1!1, b2!2).GTP(!1).GTP(!2) -> E(b1!1, b2!2).GDP(!1).GDP(!2), k = 100;

# Multiple phosphorylations can be done at once.  
A(b1!1, *M1{none}).B(b2!1, *M1{none}) -> A(b1!1, *M1{none}).B(b2!1, *M1{none}),
   k = 100.0;

# The same, but this time adding an additional optional
# explicit-species. 
E(b1!1,to-GxP!2).S(Sb1!1, *Mod{ None }).GTP(!2) + AB\_dimer->
  E(b1!1,to-GxP!2).S(Sb1!1, *Mod{ Phosphorylated }).GDP(!2),
   k = 100.0;

# an optional product species....
E(b1!1,to-GxP!2).S(Sb1!1, *Mod{ None }).GTP(!2) ->
  E(b1!1,to-GxP!2).S(Sb1!1, *Mod{ Phosphorylated }).GDP(!2)  + AB\_dimer,
    k = 100.0;

# With both an optional reactant and an optional product.  Remember,
# the names of singleton species (the singleton A mol) can be used as
# explicit names.  
E(b1!1,to-GxP!2).S(Sb1!1, *Mod{ None }).GTP(!2) + A ->
  E(b1!1,to-GxP!2).S(Sb1!1, *Mod{ Phosphorylated }).GDP(!2)  + AB\_dimer,
    k = 100;
\end{ExampleMZR}

\section{A simple modeling example}

\subsection{Two enzyme coversion of A->A*}
In this section, we will give as an example a two-enzyme, single
substrate reaction amongst hypothetical proteins.  Let us suppose we
have a system with a target protein Y, a protein with two enzyme
binding domains (E1 and E2) and a phosphorylation site and two enzymes
A and B.  A has a binding domain that binds to E2 and a small-mol
binding site ``to-Gxp'' that is bound to either GDP or GTP..  B has a
binding domain that bindings to E1 as well as a phosphorylatable
modification site.  

Suppose that the pathway under consideration is one in
which Y becomes bound by a GTP-possessing A and phosphorlated B,
whence A hydrolizes its GTP and B transfers its P03 group to Ys
unphosphorylated site.  To model this system we must define the 3
proteins, along with GTP and GDP (these are the 5 molecules involved
in the system), modification states of none and phosphorylated, along
with 2 dimerization rules (the rules that govern binding and unbinding
of A+Y and of A+B) and a single omni-gen rule (the rule that describes
the transformation of the A.B.Y trimer).

\begin{ExampleMZR}
===== Modifications ==========
# Note that this model, for the sake of using little space, does not
# use mass-based extrapolation, so no masses are included in the
# modifications or in the mol definitions.

name = none;
name = phosphorylated;


===== Molecules ===============
name = GTP;
name = GDP;
name = A(to-Y, to-Gxp);

# Note that by default, Bs start out phosphorylated.  We also could
# have had the default be none, although we would have either had to
# then explicitly create phosphorylated Bs, or provide another rule
# telling how Bs get phosphorylated.

name = B(to-Y, *mod-site {phosphorylated, none} );
name = Y( to-A, to-B, *mod-site {none, phosphorylated});

===== Dimer-Gens ===========

B(to-Y) + Y(to-A) -> B(to-Y!1).Y(to-A!1),
   kon = 50.0;
   koff = 500;

A(to-Y) + Y(to-A) -> A(to-Y!1).Y(to-A!1),
   kon = 50.0;
   koff = 500;

===== Transformation-Reactions =============

A(to-Y!1, to-GxP!2).B(to-Y!3, *mod-site {phosphorylated} ).Y(to-A!1, to-B!3, *mod-site {none}).GTP(!2) ->
  A(to-Y!1, to-GxP!2).B(to-Y!3, *mod-site {none} ).Y(to-A!1, to-B!3, *mod-site {phosphorylated}).GDP(!2),
  k = 100.0;


\end{ExampleMZR}


Suppose we wish to extend the model slightly to include dynamics that
say that upon phosphorylation of Y, A and B will dissociate
extra-quick.  To do this, we would add one new allostery rule (which
could either be an allosteric-omni or an allosteric-plex, depending on
what our intentions are) as well as new allosteric sections to both
dimerization-gens that describe speedy unbinding in the presence of
phosphorylated Y.  That example is shown below.  



\begin{ExampleMZR}
===== Modifications ==========
# Note that this model, for the sake of using little space, does not
# use mass-based extrapolation, so no masses are included in the
# modifications or in the mol definitions.

name = none;
name = phosphorylated;


===== Molecules ===============
name = GTP;
name = GDP;
name = A(to-Y, to-Gxp);

# Note that by default, Bs start out phosphorylated.  We also could
# have had the default be none, although we would have either had to
# then explicitly create phosphorylated Bs, or provide another rule
# telling how Bs get phosphorylated.

name = B(to-Y, *mod-site {phosphorylated, none} );
name = Y( to-A, to-B, *mod-site {none, phosphorylated});

===== Allosteric-Omni =========
# This must be an omni, as opposed to a plex, because these kinetics
# should apply to all complexes Y occurs in: AY, BY, and ABY - any of
# them should quickly dissociate.

Y(to-A {inactive <-*}, to-B {inactive <-*}, *mod-site {phosphorylated});


===== Dimer-Gens ===========

# Both these rules say that by default B and Y (or A and Y) bind
# together, but when Y is phosphorylated, they practically fly apart. 

B(to-Y) + Y(to-A) -> B(to-Y!1).Y(to-A!1),
   kon = 50.0;
   koff = 5.0,
  B(to-Y {default} ) + Y(to-A {inactive} ) -> B(to-Y!1).Y(to-A!1),
     kon = 1.0;
     koff = 500.0;

A(to-Y) + Y(to-A) -> A(to-Y!1).Y(to-A!1),
   kon = 50.0;
   koff = 5.0,
  A(to-Y {default} ) + Y(to-A {inactive} ) -> A(to-Y!1).Y(to-A!1),
     kon = 1.0;
     koff = 500.0;

===== Transformation-Reactions =============

A(to-Y!1, to-GxP!2).B(to-Y!3, *mod-site {phosphorylated} ).Y(to-A!1, to-B!3, *mod-site {none}).GTP(!2) 
 ->
  A(to-Y!1, to-GxP!2).B(to-Y!3, *mod-site {none} ).Y(to-A!1, to-B!3, *mod-site {phosphorylated}).GDP(!2),
  k = 100.0; 

\end{ExampleMZR}



%%% Local Variables: 
%%% mode: latex
%%% TeX-master: "user-manual"
%%% End: 
