\chapter{Conceptual Overview of libmoleculizer}

The objective of libmoleculizer is to take in a description of
proteins and other molecular entities along with a list of rules
describing various biochemical interactions, and then use this
information to generate information about the reaction network and
species that are implicitly described by those rules.

In order to understand in detail how libmoleculizer generates the
species and reactions, including kinetics, that makes up the implicit
reaction networks, it is important to understand some background
on what libmoleculizer represents, and how it understands and interprets
rules of interaction.

%% TODO - Write this section in detail 
%% \section{Combinatorial Reaction Networks}
%% \subsection{Signal Transduction Pathways}
%% END TODO

\section{Rule-Based Reaction Network Specifications}
\label{concRuleBasedSpec}
\index{rule-based modeling}

The fundamental principle underlying libmoleculizer is that reaction
networks can be specified by describing the basic constituents of
chemical species, such as types of proteins and other biochemical molecules
(including small-molecules, but also including other larger molecules
such as DNA) that exist in the network, along with a set of
rules, each of with describes an individual biochemical interaction
amongst those types.  We call this type of modeling rule-based
modeling.  


\subsection{Species are made from structural bindings between
  structural molecules}

In libmoleculizer, the physical objects that participate in reactions
are called species. Species are different structural orientations of
objects called mols. Each mol represents a variety of indivisible
biochemical entity; typically the types of mols will be various kinds
of proteins, but also can mean different small molecules, DNA, and
potentially even objects such as organelles as well.

Each mol has its own name and is defined with a set of modular binding
sites.  Each of these binding sites has its own uniquely identifying
name, as well as a collection of shapes, which represent the different
conformational states that binding site may be, thus affecting binding
kinetics.  

Mols bind together between their binding sites in order to form
species.  This includes the special case where a complex species
consists of only a single mol that is bound to nothing else.

Mols also may be defined with modification sites, which are different
locations on the mol are associated with different modifications,
post-translational modifications such as phosphorylation groups, that
are relevant to the state of the mol, but which do involve a binding
between two mols.  

\subsection{Species have structural features based on their
 constituent mols}

The relevance of species is that they have structural features based
on the structural features of the mols that make them up.  For
instance, if a species is composed of an A protein and a B protein,
and the A protein has a modification site that is phosphorylated, then
when we bind that A protein together with the B protein, in a
molecular binding between two of their binding sites, the complex
species, the A-B dimer, will also posess a modification site that is
phosphorylated. Likewise, the dimer will posess all the same binding
sites as did the individual mols before they were bound.

Features in libmoleculizer refer to specific structural features,
defined in terms of either mols or sets of mols, that individiual
species may or may not possess. For instance, a feature might be the
condition that a particular modification site on a specific type of
mol has a certain modification state -- e.g. that a particular
modification site on a Ste11 protein kinase must be
phosphorylated. Any complex species containing one or more a Ste11
mols with that modification site phosphorylated will match that
feature.

Another structural feature, this time concerning two mols, would be
the presence of a particular binding between two mols at two binding
sites - e.g. the condition that there is a binding that joins a Ste5
to a Ste11. Any complex that contains a Ste11 bound to a Ste5 between
the correct binding sites will match that feature.

There is no limit in principle to the complexity that features in
libmoleculizer can have. It is possible to talk about a feature which
is the condition of possessing a complicated sub-complex (called an
omni-plex in libmoleculizer), with a particular pattern of
modification states on it.  

\subsection{Rules are amongst features}
As stated before, rules in libmoleculizer specify the different types
of biochemical interactions in a MZR model.  These rules are then used
to generate new species and new reactions between them based upon an
initial set of initial species. The underlying metaphor in how
libmoleculizer specifies interaction rules is to say that all
interactions can be specified in terms of features and the features
react together.  Reaction rules specify features, and reactions
between the features.  Libmoleculizer applies the feature reaction
transformations specified in the rules to any sets of species
possessing the relevent features.  This process is applied
iteratively and thus, a network of species and reactions is generated.  

The features that are required by each type of rule is rule-specific
to the three kinds of rules.  The first type of rule, called a
dimerization-gen, which specifies a particular binding interaction -
it looks for two features, one matching a free binding site on a
particular mol type and the other with another free binding site on
another type of mol, and when two species are found that match these
features, the two free sites are bound together.  The second type of
rule is an omni-gen, which looks for a particular omni-plex, a mol
sub-complex of species, with any portion of its modification state
specified.  When it finds this omni-plex, any number of small molecule
exchanges or modification exchanges may be performed on that
omni-plex.  Finally, the uni-mol-gen, which looks for a specific type
of mol with a potentially partially specified modification state; on
this mol any number of small molecule exchanges or modification
exchanges may be specified.  Note that this is a special case of the
omni-gen, and will likely be depreciated in the next version.

Each of these three types of rule, dimerization-gen by looking for two
free site features, omni-gen looking for an omni-plex feature, and
uni-mol, looking for a single mol matching feature behaves similarly:
it looks for features, and when finding them, has a specified
transformation it applies to the species.  The dimerization-gen joins
the two free binding sites together, and both the omni-gen and the
uni-mol-gen perform some kind of state change - a small mol exchange
or a modification exchanage - on the matching structural feature.

\subsection{Reaction rules create species and reactions from an
  initial list of species}

Either in the set of rules or by using the libmoleculizer API, the
user must specify a list of initial species that exist within the
model. For sets of species containing the features needed for the
different rules, libmoleculizer will apply the rule's transformation
to the matching substrate species, in order to create a set of product
species, which may or may not be new.  This mapping of substrate
species to product species is added as a reaction to the list of
reactions.  In addition, kinetics are generated based on the reaction
rule and added to the reaction, a topic to be discussed later in this
chapter. At this point, the reaction rules have been applied to the
initial set of species, which, when applied, create a new reaction
with potentially new product species.  If any of the product species
are new, they are checked for features, which may set off the
application of rules, causing more new reactions with potentially new
product species to be formed.  The whole reaction network relative to
a set of initial species includes all species and all reactions that
might be created on any iterative step of the expansion process.  

\section{Specifying Reaction Kinetics} 

%% TODO fix X and Y units for the reaction rate.  
For any reaction rule, one or more reaction rates must be specified
that indicate the kinetics of the reaction.  Uni-mol and omni-gen
reaction rules take a single reaction rate.  A dimerization-gen
reaction rule takes two: dimerization reactions are assumed to be
reversible by default.  Irreversible reaction rules are expressed as a
reversible reaction with an off rate of 0.  

There are two ways this kinetic parameter can be interpreted.  If no
reaction rate extrapolation is used, then the on-rates supplied in the
rule are passed into the generated reaction directly.  Using this
method, users can use whatever units they would like.

Libmoleculizer also has the potential to extrapolate a reaction rate
from the kinetics provided based on the masses of the substrate
species involved. 

%% TODO figure this out and get a reference in here.
This assumption is used because of something in Gillespie.  This
paragraph references what this is and why it is relevant.

%% Figure out these units here.
When reaction rate extrapolation is used, the rate associated with a
binary reaction is multiplied by the reduced-mass of the the substrate
species, in order to better approximate the way that the masses of the
species affect the collisions due to collisions that underly the
reaction.  In this interpretation, units must be in either X or Y.

\section{Allosteric Kinetics are Based on Binding Shape}
Biochemical reaction networks, particularly signal transduction
pathways, often use allosteric proteins to regulate their
behavior. Allosteric proteins are proteins whose shape, and
consequently behavior, with regards to different interactions can
change into one of multiple different forms depending upon the state
of certain modifications on that protein as well as what entities that
protein is bound to.  One example would be a model in which a protein
Fic10 can bind to another Sca3, after it has been activated by being
doubly phosphorylated on two of its modification sites.  

How does libmoleculizer treat an allosteric reaction?  Say we would
like to express the above reaction: Sca3 binds with Fic10, say from
the ``to-Fic10'' and ``to-Sca3'' sites respectively, but only when
Sca3 has been doubly phosphorylated, say at it's ``phos-site-one'' and
``phos-site-two'' sites; how can this be done?  Libmoleculizer
decomposes a reaction like that into defining the charectoristics that
cause the behavior of that molecule binding site (in this case the
``to-Fic10'' site of Sca3) to change.  These charectoristics, another
feature that can be tested for in new species, are associated with
binding shape changes.  In this case, the binding site ``to-Fic10''
may have been defined with the mandatory default-binding-site-shape
parameter equal to ``half-activated''; specifying the feature that
a Sca3 has both of its phos-sites phosphorylated would be associated
with the binding-site-shape change that when this feature is matched
the matching Sca3 mol's ``to-Fic10'' site is updated to the shape
``fully-activated''.  Second, in the applicable reaction rule the
allostery applies to, describe the allosteric kinetic rates in terms
of the binding site shapes that supply the allostery. 

For instance, to represent above reaction describing the doubly
phosphorylated Sca3-Fic10 dimerization, we start by providing
information that the Sca3 binding site has default shape ``inactive''
and define that when it is doubly phosphorylated at its
``phos-site-one'' and ``-two'' sites, that shape becomes ``active''.
Next, we define the dimerization rule representing Sca3 and Fic10
dimerization with a dimerization-gen.  The default kinetics described
in that reaction rule applywhen the two binding sites are have their
default shapes.  In this case, we would want to express that that
reaction does not occur, so we give the on-rate as 0, and the off-rate
as a very high number to indicate that does not happen.  Inside that
rule, we would also specify the allostery, with an shape-based
'exception' to the default kinetics: i.e. if the ``to-Fic10'' site has
shape ``activated'', then in that case the on- and off-rates should be
given as the activated kinetic rates.  

%% TODO
%% \section{Some design rationale in the form of a Philosophical Argument for using shapes with allostery}

\section{Unique IDs for Naming Complex Species}
 For input and output capabilities, it is important that there is a
 consistent naming scheme for naming complex species.  For instance,
 it is desirable that structurally identical species will produce the
 same name, and that the structure can be determined from the name.
 For the time being, it is out of the scope of this document to
 discuss exactly how these names are generated.  However, it is
 important to know that each complex species has a unique id.  Two
 complexes are structurally identical if and only if the ids produced
 for them by libmoleculizer are the same.  