\chapter{Introduction}

\section{Overview of l ibmoleculizer}
Libmoleculizer is a computer library that takes a set of rules
describing a collection of protein and small molecule types as well as
a set of descriptions of interactions amongst those types,
e.g. various types of protein-protein, protein-small molecule, and
enzymatic reactions (protein modifications), and using that
information explicitly generates a network of multi-protein species
and their reactions that are described by those rules.

The capacity to describe networks implicitly, i.e. by using rules to
generate a network can be be very adventageous for modeling
biochemical networks.  This is because in many common cases, the size
of the explicit network grows exponentially relative to the number of
protein/molecule types and interactions types that network can
display.  For instance, in certain models of the alpha mating pathway
in S. {\it cerivisiae}, defined with fewer than 20 molecule types and
fewer than 40 interactions, tha explicit network may be over 100,000
species and over 300,000 reactions.  

Libmoleculizer is a computer tool that can help to manage the
complexity of the absolute set of species and reactions in a reaction
network, by expanding a typically saner network of biochemical
interactions that users provide.  Typically, Libmoleculizer will be
used as a component of a larger simulator, which will use
libmoleculizer for species and reaction management.  For instance, a
simulator using libmoleculizer could take in a set of rules, use these
to expand the network and then use the full species and reaction
information to run an ODE or stochastic simulation.  

Where libmoleculizer goes beyond other software packages of its type
however, is in the generation of these networks.  Biochemical networks
do not have to be expanded all at once.  This has two advantages: it
allows libmoleculizer to be able to treat networks other packages
cannot, such as unbounded species growth, and it also allows for
faster operation.  In a spatial simulation, if two species collide,
the user will not care to know every species and reaction that can
occur in the model, they will want to know whether these two species
collide.  Using libmoleculizer, users do not have to computationally
pay for the time to generate huge portions of the network they do not
care about.  The authors feel this capability as well makes
libmoleculizer a component that can be used effectively with different
simulators.

\section{Manual Overview}
This manual aims to provide introduction to the functionality and use
of the libmoleculizer library.  

This manual can be divided up into three aspects.  The first portion
of the manual is to describe what Moleculizer does; this is treated in chapters
\ref{chap:conceptualOverviewChapter} and
\ref{chap:usingLibmoleculizerChapter}.  The second objective is to
understand the MZR file format in which reaction rules are specified
for libmoleculizer; this is overviewed in chapter \ref{chap:theRulesChapter},
and more explicitly in chapter \ref{chap:mzrReference}.  Finally,
the libmoleculizer API must be learned, in order to understand how to
pass MZR files in, expand the network, and read out species and
reaction information.  This final topic is dealt with in chapters
\ref{chap:usingLibmoleculizerChapter}, \ref{chap:interfacesChapter}, and
\ref{:apiReference}.

Finally, although libmoleculizer is a relatively standard mostly-GNU
package written in Std C++, C, and a little python, and packaged up
using typical, fairly portable ways, your milage may vary.  If you
have any trouble, consult the installation instructions and
occasional troubleshooting help available in the chapter
\ref{chap:installingChapter}.

\section{History}

Read all about it in \cite{lok05}.

Libmoleculizer is the decendant of a prior program called Moleculizer,
develeped by Dr. Larry Lok at the Molecular Sciences Institute to
study the alpha mating pathway in yeast. The standard modeling
techniques at the time all involved explicitly enumerating a set of
chemical species and reactions. However, it was determined for the
alpha signal transduction pathway in yeast that there were so many
species and reactions possible, that they could never be practically
enumerated. The typical response to this problem had previously been
to simplify the system, combining species and reactions. However,
this approach was non-optimal, as it required making fundamental
assumptions about what was and was not important, prior to
investigating system behavior. 

The response was to develop a new simulator called Moleculizer that
took in a description of the basic monomeric proteins and small
molecules, as well as a set of rules that describing the basic
interactions between proteins. These rules were used within a
Gillespie simulator, which would alternate between executing Gillespie
reaction events and using the rules in order to generate enough of the
reaction network so that the immediately following reaction event
would be accurate.  

Realizing the versatility of this new rule-based approach, a new
software project was begun by Nathan Addy at the Molecular Sciences
Institute, to adapt the reaction network generating components of
Moleculizer into a generic library, called libmoleculizer, that could
be used within a variety of simulators and in a variety of contexts.

This manual corresponds to version \currentversion of that software. 