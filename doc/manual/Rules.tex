\chapter{The MZR Language for expressing rule-based models}
\label{chap:theRulesChapter}

A libmoleculizer input file, called a MZR File - short for Moleculizer
Rule File - is an xml document in a particular syntax that describes a
set of molecule types as well as a set of reaction rules
that describe interactions between those types, and groups of those types.

This chapter talks about MZR file syntax, and discusses how to express
sets of rules representing a variety of biological pathways within it.

\section{MZR File Overview}

A MZR file consists of a 'moleculizer-input' root tag that contains a
single 'model' element.  The model element contains several mandatory
subelements, which contain all of the description of the model:
'modifications', 'mols', 'allosteric-plexes', 'allosteric-omnis',
'reaction-gens', 'explicit-species', and 'explicit-reactions'. 

\begin{ExampleXML}[caption=The overall structure of a MZR file, label=overallmzrstruct]
<!-- Overall structure of a MZR file -->
<moleculizer-input>
  <model>
    <modifications>
       <!-- Information about types of modifications goes here -->
    </modifications>
    <mols>
       <!-- The different types of mols are described here. -->
    </mols>
    <allosteric-plexes>
       <!-- Allosteric complex species are described here. -->
    </allosteric-plexes>
    <allosteric-omnis>
       <!-- Allosteric sub-complexes are described here. -->
    </allosteric-omnis>
    <reaction-gens>
       <!-- The reaction rules all go in this section -->
    </reaction-gens>
    <explicit-species>
       <!-- Species can be assigned names in this section. -->
    </explicit-species>
    <explicit-reactions>
       <!-- Extra reactions can be defined amongst named species here. -->
    </explicit-reactions>
  </model>
</moleculizer-input>
\end{ExampleXML}

Generally speaking, the information in the modifications section
describes and defines each of the legal modification values, typically
different chemical groups, that can be used within the model.  

The elements in the 'mols' section correspond to molecule definitions for
the different molecules that can be used within the model.

Allosteric-plexes and allosteric-omnis both contain information that
describes how specific binding sites in a complex change their shape
relative to features of the containing complex (such as specific
binding sites being bound, specific regulatory sites being modified in
specific ways, presence of specific mols in complex, or other
conditions).  

The reaction-gens element contains each of the different reaction
rules in the model; each of these corresponds to a single biochemical
interaction between different combinations of functional and
regulatory sites.  Typically these will correspond to dimerization
reactions, decomposition reactions, and more complicated enzymatic
reactions. 
 
The explicit-species section is where complexes can be specified and
given names, names that can be used later in reactions or within
libmoleculizer to refer to those complexes.  The explicit-reactions
section containts so called explicit reactions, reactions where all
substrates, all products, and all kinetics are explicity stated.  

The remainder of this chapter deals with each of these different
sections, what kind of information goes into them, and how to write
them.  For more information, please consult the \ref{mzrReference}
chapter.

\section{Writing the 'modifications' Section}
\label{sec:modifications}
\index{modifications}

{\bf /moleculizer-input/model/modifications } 

The modifications section consists of any number (including zero) of
modification elements, each of which represents a particular
modification that may be present on a protein, e.g. 'none',
'phosphorylated', 'ubiquitinated', at one of its modification
sites. Each of the modifications that will be used and referred to in
the model must be defined here.

Each modification must have a unique name attribute and a weight-delta
subelement, representing the weight that that modification adds to the
basic molecular weight of the mol that it appears on.  

See \ref{sampleModificationsSection} for an example of what each of
these modification definitions looks like in the modification
section.  Note that a 'none' modification has been defined. Because
each modification site on each species in the model must have a
modification state at all times, this is needed to represent an
unmodified modification site.

\begin{ExampleXML}[caption=A Sample Modifications Section in a MZR file, label=sampleModificationsSection]
<modifications>
  <modification name="none">
    <weight-delta daltons="0.0" />
  </modification>
  <modification name="phosphorylated">
    <weight-delta daltons="42.0" />
  </modification>
  <modification name="doubly-phosphorylated">
    <weight-delta daltons="84.0" />
  </modification>
</modifications>
\end{ExampleXML}

\section{Defining the building blocks of species in the 'mols'
  section} 
{\bf /moleculizer-input/model/mols }

Complex species in libmoleculizer are aggregates of mols bound
together between different binding sites, along with a particular
assignment of modifications to each of the modification sites on that
complex.

In the mols section of the model, each of the mol types that can
occur in complexes within the model are specified.

Libmoleculizer supports two kinds of molecules in its networks.  The
first type are the large, structural molecules typically representing
proteins, that have some number of binding and modification sites.
These are called 'mod-mols' in libmoleculizer.  The second type are
the small molecules, called 'small-mols' that can participate in
reactions, but do not have any structure themselves.

\subsection{Defining 'mod-mol' elements}
Each mod-mol represents a single type of protein in the reaction
network.  Each type of mod-mol must have a unique name, a weight, and
any number of binding and modification sites. Mod-mols and their
structural components (binding sites, modification sites) are the
basic entities referred to by reaction rules (if you recall the
\ref{conceptualOverviewChapter} chapter, the features that reaction
rules look for are collections of specific structural components in
specific shapes).  

To define a basic mod-mol, create a mod-mol element with a name
attribute that contains the unique name of the protein.  Then give
that mod-mol a weight with a 'weight' subelement that has a daltons
attribute (this is where the weight, in daltons, goes!).  See
\ref{basicmodmolstructure} for an example.

\begin{ExampleXML}[caption=Basic mod-mol structure, label=basicmodmolstructure ]

<!-- Example of a mod-mol definition with mandatory name attribute -->
<!-- and mandatory weight subelement -->
<mod-mol name="Gpa1">
  <weight daltons="54075.9"/>
  <!-- binding-site and mod-site elements -->
</mod-mol>

\end{ExampleXML}

Next, give the mod-mol element zero or more binding-site subelements
and zero or more mod-site subelements, to give it some structure.  

\subsubsection{Defining 'binding-site' elements}
Mod-mols can have one or more binding sites, each one described by its
own binding-site subelement of the mod-mol element.  This binding site
is a specific location of iteraction where the mod-mol can complex
with other proteins.

Each binding site is associated with a list of binding site states,
called 'shapes' in libmoleculizer, an idea that encompasses binding
site conformation as well as other physical charectoristics, such as
occlusion, to binding-sites.  The use of binding site shapes,
particularly to express allosteric protein interaction behavior and
reaction kinetics is a critical property of the way libmoleculizer
works, and must be understood.  Please see
\ref{sec:allosterickinetics} in the \ref{conceptualOverviewChapter}
for more information.

Each binding site element must be given a mandatory name attribute
that must be unique among all the parent mod-mol's binding sites. Each
site must also have one or more site-shape subelements (at least one must be
specified in order to specify the default site-shape for this binding
site), representing the different conformational shapes the binding
site may possess at any given time.  Finally, a single subelement
'default-shape-ref' with a mandatory name attribute must be present,
which must take as its value the name of some site-shape defined for
the same binding site.

\begin{ExampleXML}[caption=Defining binding sites in a mod-mol, label=bindingsitedefinitions]
<mod-mol name='Gpa1'>
  <weight daltons='101.1' />
  <binding-site name="to-GXP">
    <site-shape name="default"/>
    <default-shape-ref name="default" />
  </binding-site>
  <binding-site name="to-Ste2">
    <default-shape-ref name="default" />
    <site-shape name="default" />
    <!-- The conditions under which this binding site takes the state
         occluded will be specified either in the allosteric-state
         subsection of this mod-mol definition, or in the
         allosteric-plex or allosteric-omnis element later in the
         model.  -->
    <site-shape name="occluded" />
  </binding-site>
  <binding-site name="to-Ste4">
    <default-shape-ref name="inactive" />
    <site-shape name="inactive" />
    <site-shape name="partially-active" />
    <site-shape name="fully-active" />
  </binding-site>
</mod-mol>
\end{ExampleXML}


\subsubsection{'A note on the binding-shape state constraint of
  libmoleculizer'}
As mentioned elsewhere in this manual, one of the primary modeling
metaphors used by libmoleculizer is to enforce that each binding site
in a complex species has a shape at all times and that the specific
shapes of interacting binding sites are the determiners of allosteric
kinetics.

For instance, suppose two mod-mols: Fus3 and Ste5 are defined in the
mols section of the MZR rules file.  Fus3 has one binding site
'to-Ste5' that has only one state possible: 'default'.  Ste5 has two
binding sites, 'X' and 'to-Fus3', and 'to-Fus3' can be in either a
'default' state or an 'occluded' state.  This information is specified
in Fus3 and Ste5's mod-mol element definitions.

In an appropriate place (in the allosteric-state section of the
mod-mol if the allosteric conditions are induced only by modifications
on that mod-mol, or in either the allosteric-plexes or
allosteric-omnis sections of the model, if the allosteric conditions
are induced by that mol being a part of a particular complex)
conditions under which different binding sites obtain different shapes
can be specified.  For instance, in the previous example, a rule that
states that when Ste5 is complexed with a mod-mol 'Ste11' at another
'to-Stell' site, then its 'to-Fus3' site has the shape 'occluded'
could be added to the allosteric-omnis section of the model.

Finally, in reaction rules that involving a binding site with multiple
states (typically this means different dimerization and decomposition
reactions), different reaction kinetics can be specified for
combination of relevant of binding site-states (for instance, in a dimerization
reaction, a reaction rate must be given for each pair of binding-site
shapes that can occur in the two participating binding sites).  This
is how libmoleculizer treats allosteric-kinetics.  

\subsubsection{'mod-site' elements represent modification sites on a
  mod-mol}
A mod-site element represents single modification site on the mod-mol;
i.e. a site where the protein can undergo post-translational 
modifications as a part of the reaction network.  

To give a mod-mol element a specific modification site, create a
mod-site subelement.  Each mod-site-element must have a 'name'
attribute, and a mantatory default-mod-ref subelement, which itselfhas
a mandatory 'name' attribute.  The default-mod-ref represents the
default modification value of this mod-site, and must match one of the
modifications defined in the modifications section.

\begin{ExampleXML}[caption=Defining a modification site in a mod-mol, label=modsiteexample]
<modifications>
  <modification name="none">
    <weight-delta daltons="0.0"/>
  </modification>
  <modification name="phosphorylated">
    <weight-delta daltons="42.0"/>
  </modification>
</modifications>

<mols>
  <mod-mol name="Gpa1">
    <mod-site name="phosphorylation-site-1">
      <default-mod-ref name="none"/>
    </mod-site>
    <mod-site name="phosphorylation-site-2">
      <default-mod-ref name="none"/>
    </mod-site>
  </mod-mol>
</mols>
\end{ExampleXML}

\subsubsection{'allosteric-state'}
A mod-mol may contain zero or more 'allosteric-state' elements. Each
of these elements describes a set of modification conditions under
which that mod-mol undergoes an allosteric change, such that the
shapes of its binding sites are updated in some particular ways.

For example, assume that we have defined a protein Fic2 with a binding
site 'functional-site-1' that can have a shape 'active' or 'inactive',
but is in default shape 'inactive'.  Suppose then that Fic2 also has 2
regulatory sites 'phosphorylation-site-1' and
'phosphorylation-site-2', and suppose we would like to represent that
when both these sites are phosphorylated site 'funcional-site-1'
becomes active.  This allosteric relationship would be described using
an 'allosteric-state' element.

Each allosteric state element has two mandatory elements, a mod-map
subelement and a site-shape-map subelement.  The mod-map element
consists of one or more 'mod-site-ref' elements, which desribe a
condition of having a particular modification on a particular
modification site.  When a mod-mol in a species matches all those
conditions, the changes in the site shapes described in the site-shape
map is applied to the mod-mol.

For instance, suppose we have mod-mol called 'Fic3' defined as having
a binding site 'site-1' which by default has shape 'inactive' but
which is also defined as having an active shape 'active' and a single
modification site 'phos-site' with a default modification value of
'none'.  Suppose we would like to express that 'site-1' takes the
shape 'active' when 'phos-site' has a 'phosphorylation' modification
attached to it.  We would do this by adding an allosteric-state
element to this mod-mol.  This example is shown in
\ref{allostericstateexample}. 

\begin{ExampleXML}[caption=An example of an allosteric-state element, label=allostericstateexample, showspaces=false]
<allosteric-state>
  <!-- Each mod-map describes a set of modification site -> modification
       value associations. >
  <mod-map>
    <!-- Each mod-site ref is the association of a specific mod-site,
         in this case the 'phos-site' mod-site on this mod-mol, with a
         particular modification, in this case, 'phosphorylated'. -->
    <mod-site-ref name='phos-site'>
      <mod-ref name='phosphorylated' />
    </mod-site-ref>
  </mod-map>
  <!-- When all the mod-sites have the modifications listed in the mod
       map, the site-shape-map is applied.  Each of the binding-sites
       listed in the binding-site-ref elements have the shapes applied
       in their site-shape-ref elements applied. -->
  <site-shape-map>
    <binding-site-ref name='site-1'>
      <site-shape-ref name='active' />
    </binding-site-ref>
  </site-shape-map>
</allosteric-state>
\end{ExampleXML}

\subsubsection{A More Extensive Mod-mol example.'}

To end this section, it may be helpful to define a basic protein in
its entirety using mod-mol syntax.

To do this, let us model the hypothetical protein Fic1.  In a
literature review, we see that it binds to a scafold protein where it
acts as the target of a kinase cascade.  When phosphorylated, Fic1
will dissociate from the scaffold and go and bind to a target, but
apparently using a different binding site.  It is found based on the
literature review that Fic1 does not activate the target in its off
state, and that single phosphorylation partially activates the
pathway, although at lower levels then it does in its doubly
phosphorylated state.

To represent this protein as a mod-mol, we must give it a name and
weight, a list of binding sites, and a list of modification sites.  We
must also define the various shapes that can be associated with each
binding in order to induce allosteric effects.  Here is an example of
how such a mod-mol could be defined.

\begin{ExampleXML}[caption=A complete mod-mol example, label=completemodmolexample]
<mod-mol name='Fic1'>
  <weight daltons='413.7' />
  <binding-site name='to-scaffold'>
    <site-shape name='default' />
    <site-shape name='half-inactive' />
    <site-shape name='inactive' />
    <default-shape-ref name='default' />
  </binding-site>
  <binding-site name='to-target'>
    <site-shape name='inactive' />
    <site-shape name='active' />
  </binding-site>
  <mod-site name='phos-site-1'>
    <default-mod-ref name="none" />
  </mod-site>
  <mod-site name='phos-site-2'>
    <default-mod-ref name="none" />
  </mod-site>

  <!-- This represents the allosteric state when phos-site-1 -->
  <!-- is phosphorylated and phos-site-2 is not.             -->
  <allosteric-state>
    <mod-map>
      <mod-site-ref name='phos-site-1'>
	<mod-ref name='phosphorylated' />
      </mod-site-ref>
    </mod-map>
    <site-shape-map>
      <binding-site-ref name='to-scaffold'>
	<site-shape-ref name='half-inactive' />
      </binding-site-ref>
    </site-shape-map>
  </allosteric-state>

  <!-- This represents the allosteric state when phos-site-1 -->
  <!-- is phosphorylated and phos-site-2 is not.             -->
  <allosteric-state>
    <mod-map>
      <mod-site-ref name='phos-site-2'>
	<mod-ref name='phosphorylated' />
      </mod-site-ref>
    </mod-map>
    <site-shape-map>
      <binding-site-ref name='to-scaffold'>
	<site-shape-ref name='half-inactive' />
      </binding-site-ref>
    </site-shape-map>
  </allosteric-state>

  <!-- This represents the allosteric state both phos-sites  -->
  <!-- are phosphorylated. -->
  <allosteric-state>
    <mod-map>
      <mod-site-ref name='phos-site-1'>
	<mod-ref name='phosphorylated'>
      </mod-site-ref>
      <mod-site-ref name='phos-site-2'>
	<mod-ref name='phosphorylated'>
      </mod-site-ref>
    </mod-map>
    <site-shape-map>
      <binding-site-ref name='to-scaffold'>
	<site-shape-ref name='inactive' />
      </binding-site-ref>
      <binding-site-ref name='to-target'>
	<site-shape-ref name='active' />
      </binding-site-ref>
    </site-shape-map>
  </allosteric-state>
  
</mod-mol>
\end{ExampleXML}

\subsection{'small-mol'}
A small-mol element represents a small molecule - typically an organic
molecule that participates in pathway reactions.  ADP/ATP and GDP/GTP
are two common examples.

The most defining charecteristic of the small-mol is a lack of
structural information.  Small molecules participate in chemical
reactions in toto.  They do not have binding sites, nor do they have
modification sites.  Accordingly, small-mols in libmoleculizer are
modeled as having no mod-sites, and only a single implicit
binding-site, that has the same name as the small mol itself. 

The definition for a small-mol element is a mandatory name attibute
and a mandatory weight child with a mandatory daltons attribute.
That's it!

\begin{ExampleXML}[caption=Examples of small-mol definitions, label=smallmolex]
<mols>
  <small-mol name="ADP">
    <weight daltons="427.20"/>
  </small-mol>
  <small-mol name="ATP">
    <weight daltons="507.181"/>
  </small-mol>
</mols>
\end{ExampleXML}

\section{Defining 'allosteric-plex' elements}

The allosteric-plexes element is composed of zero of more
allosteric-plex elements, each of which specify allosteric binding
shapes on binding sites in particular complex species.

An allosteric-plex is very similar to an allosteric-omni.  The
difference is that where an allosteric-omni defines a particular
subcomplex, that when found in a species confers binding-site shape
changes describbed in the allosteric-omni condition, an
allosteric-plex defines a plex species and indicates binding site
changes induced in the binding sites of that species.  

This section is little used these days, and can almost always be
subsumed into the allosteric-omni section.  There may however, be
instances where the user wishes to express exact allostery in a way
that cannot be done with an allosteric-omni.  In that case read the
sections \ref{definingAllostericOmnis}, then come back and read the
followign example.  Just remember that in an allosteric-plex element,
the plex subelement specifies a complex species, and in the
allosteric-omni element, the plex subelement specifies a sub-species
that matches and species containing it.  

\begin{ExampleXML}
<allosteric-plexs>
  <allosteric-plex>
    <plex>
      <mol-instance name="the-Ste5">
	<mol-ref name="Ste5" />
      </mol-instance>
      <mol-instance name="the-Ste11">
	<mol-ref name="Ste11" />
      </mol-instance>
      <binding>
	<mol-instance-ref name="the-Ste5">
	  <binding-site-ref name="to-Ste11" />
	</mol-instance-ref>
	<mol-instance-ref name="the-Ste11">
	  <binding-site-ref name="to-Ste5" />
	</mol-instance-ref>
      </binding>
    </plex>
    <instance-states>
      <mod-mol-instance-ref name="the-Ste5">
	<mod-map>
	  <mod-site-ref name="phos-site-1">
	    <mod-ref name="phosphorylated"
	  </mod-site-ref>
	  <mod-site-ref name="phos-site-2">
	    <mod-ref name="phosphorylated"
	  </mod-site-ref>
	</mod-map>
      </mod-mol-instance-ref>
      <mod-mol-instance-ref name="the-Ste11">
	<mod-map>
	  <mod-site-ref name="site-1">
	    <mod-ref name="none"
	  </mod-site-ref>
	</mod-map>
      </mod-mol-instance-ref>
    </instance-states>
    <allosteric-sites>
      <!-- Allosteric-sites is made of one or more mol-instance ref 
           elements, each of represents a single binding site whose 
	   shape changes. -->
      <mol-instance-ref name="the-Ste5" >
	<binding-site-ref name="to-X">
	  <site-shape-ref name="active" />
	</binding-site-ref>
      </mol-instance-ref>
      <mol-instance-ref name="the-Ste5" >
	<binding-site-ref name="to-Y">
	  <site-shape-ref name="inactive" />
	</binding-site-ref>
      </mol-instance-ref>
    </allosteric-sites>
  </allosteric-plex>
</allosteric-plexes>

\end{ExampleXML}


\section{Defining 'allosteric-omni' elements}

In the mod-mol section, we learned how we can specify allosteric
changes to a binding site shape relative to the modification state of
that mod-mol by using allosteric-state elements. This takes
case of the allosteric case where a mol has a conformational change
because of specific modifications on itself.  However, what about the
case where conformational changes because certain proteins are in some
kind of activating subcomplex?

For instance, Cdk proteins typically have multiple regulatory sites
and multiple binding sites, and the activation of a functional site
depends on multiple regulatory sites being phosphorylated, as well as
a specific binding to a cyclin protein.  For the functional
site to be activated, all three conditions - both regulatory sites being
phosphorylated and the appropriate binding site being bound to a
cyclin mol.  To describe this allosteric condition, where binding-site
confrmation is activated while in an activating complex, an
allosteric-omni must be used. 

The allosteric-omnis section describes one or more allosteric-omni
elements, distinguished subcomplexes that may have particular
modification conditions and gives to each allosteric-omnis a list of
binding sites and shapes that are given to those binding sites when
they are in the allosteric-omni.  In the case of the above example,
the allosteric-omni would be a complex that matches a Cdk mol
bound to a cyclin mol wherein the Cdk has two phosphorylated
modification sites.  The allosteric sites of that allosteric-omni
would include the condition that the functional site on the Cdk in
that omni goes from an inactive binding site shape to an active
binding site shape.  

The specification of this allosteric-omni element, has three
components: a plex element that describes the structure of the
allosteric-complex, an instance-state element, which describes
additional modification constraints on the plex, and an
allosteric-states element, which describes how the binding sites in
a matching allosteric-omni are to be updated.

\subsection{Defining the plex element}
The plex element consists of a set of mol-instances, which describe
the specific instances of mols within the allosteric-omnhi, and a set
of binding elements, which describe the bindings between the mols in
the omni.

Each mol-instance element is specified by providing a name element
that gives a unique identifier to that mol-instance in the complex, as
well as a mol-ref subelement, whose name attribute is the type of mol.

Each binding element consists of two mol-instance-ref elements, whose
name attribute is a particular mol-instance name, and a
binding-site-ref subelement, whose name is a particular binding site
on that mol.  

In \ref{omniPlexExample}, an example plex that specifies a Cdk bound
to a cylin is specified.  Sub-complexes may be of any size, including
just one mol (in which case the allosteric-omni would have the same
effect as as allosteric-state element defined in that mols
definition).

\begin{ExampleXML}[caption=An example of how to define a plex section that specifies a cyclin-bound Cdk protein, label=omniPlexExample]
<allosteric-plex>
  <plex>
    <mol-instance name="the-CDK">
      <mol-ref name="Cdk" />
    </mol-instance>
    <mol-instance name="the-cyclin">
      <mol-ref name="Cyclin" />
    </mol-instance>
    <binding>
      <mol-instance-ref name="the-CDK">
	<binding-site-ref name="to-cyclin" />
      </mol-instance-ref>
      <mol-instance-ref name="the-cyclin">
	<binding-site-ref name="to-CDK" />
      </mol-instance-ref>
    </binding>
  </plex>
  <instance-states>
    <!-- modification information goes here -->
  </instance-states>
  <allosteric-sites>
    <!-- Binding site shape updating information goes here -->
  </allosteric-sites>
</allosteric-plex>
\end{ExampleXML}

\subsection{Defining the instance-states element}
Information about modification conditions on the allosteric-plex goes
in the instance-states element.  The instance-states element consists
of one or more mod-mol-instance-ref elements, which each have a name
that must be a mol-instance name from the plex element.  Each of these
mod-mol-instance-ref elements must have a mod-map subelement, which
describes the modifications on that mod-mol.  A mod-map consists of
one or more mod-ref elements, which takes the name of a mod-site, and
a mod-ref, which takes the name of a specific modification.

To see how we would specify the instance-states element for the
allosteric-omni we are creating to specify cyclin-bound CDK, see the
next example \ref{allostericPlexInstanceStatesEx}.  Note that although
this example only provides conditions on one mol in the omniplex, by
adding more mod-mol-instance-ref elements, we could also add
conditions on the Cyclin as well.  

\begin{ExampleXML}[caption=''Giving modification conditions to the allosteric-plex, using the instance-states element'',label=allostericPlexInstanceStatesEx]
<allosteric-plex>
  <plex>
    <!-- This section describes the plex of cyclin-bound Cdk -->
  </plex>
  <instance-states>
    <mod-mol-instance-ref name="the-CDK" >
      <mod-map>
	<mod-site-ref name="phosphorylation_site_1">
	  <mod-ref name="phosphorylated" />
	</mod-site-ref>
	<mod-site-ref name="phosphorylation_site_2">
	  <mod-ref name="phosphorylated" />
	</mod-site-ref>
      </mod-map>
    </mod-mol-instance-ref>
  </instance-states>

  <allosteric-sites>
    <!-- Binding site shape updating information goes here -->
  </allosteric-sites>
</allosteric-plex>
\end{ExampleXML}

\subsection{Defining the allosteric-sites element}

In the plex and instance-states sections of the allosteric-omni, we
have defined an omni-plex, a class of sub-complex that species do or
do not possess.  In the allosteric-sites element, we specify all the
conformational binding-site-shape changes that happen to this
allosteric-omni.  All this information goes into the allosteric-sites
element.  Into this element goes one or more elements, that specify a
specific mol and binding site in the allosteric-plex and a
binding-shape that binding site should be in.  This takes the form of
one or more mol-instance-ref elements (with a name attribute of a mol
in the plex) that have a single binding-site-ref element (with a name
attribute for the name of the binding site), that has its own
site-shape-ref elements.  \ref{molInstanceRefForm} shows how each of
these binding-shape updates must be specified in an allosteric-sites
element.  \ref{allostericSitesFullExample} gives the complete
allosteric-sites section for our example.  Finally,
\ref{fullAllostericOmniExample} shows the definition of the complete
allosteric-omni, Cyclin having an active functional site when it is
doubly phosphorylated and bound with Cdk, is written.

\begin{ExampleXML}[caption=Each mol-instance-ref subelement describes a single binding site update., label=molInstanceRefForm]
<allosteric-sites>
  <!-- This mol-instance-ref says that the "functional-1" binding site of
       "the-Cdk" should be made "active" -->
  <mol-instance-ref name="the-Cdk">
    <binding-site-ref name="functional-1">
      <site-shape-ref name="active" />
    </binding-site-ref>
  </mol-instance-ref>
</allosteric-sites>
\end{ExampleXML}

\begin{ExampleXML}[caption= someexample, label=allostericSitesFullExample]
<instance-states>
  <mod-mol-instance-ref name="the-CDK" >
    <mod-map>
      <mod-site-ref name="phosphorylation_site_1">
	<mod-ref name="phosphorylated" />
      </mod-site-ref>
      <mod-site-ref name="phosphorylation_site_2">
	<mod-ref name="phosphorylated" />
      </mod-site-ref>
    </mod-map>
  </mod-mol-instance-ref>
</instance-states>
\end{ExampleXML}

\begin{ExampleXML}[caption=The full allosteric-omni specification for the Cyclin/Cdk system,label=fullAllostericOmniExample]
<allosteric-plex>
  <plex>
    <mol-instance name="the-CDK">
      <mol-ref name="Cdk" />
    </mol-instance>
    <mol-instance name="the-cyclin">
      <mol-ref name="Cyclin" />
    </mol-instance>
    <binding>
      <mol-instance-ref name="the-CDK">
	<binding-site-ref name="to-cyclin" />
      </mol-instance-ref>
      <mol-instance-ref name="the-cyclin">
	<binding-site-ref name="to-CDK" />
      </mol-instance-ref>
    </binding>
  </plex>
  <instance-states>
    <mod-mol-instance-ref name="the-CDK" >
      <mod-map>
	<mod-site-ref name="phosphorylation_site_1">
	  <mod-ref name="phosphorylated" />
	</mod-site-ref>
	<mod-site-ref name="phosphorylation_site_2">
	  <mod-ref name="phosphorylated" />
	</mod-site-ref>
      </mod-map>
    </mod-mol-instance-ref>
  </instance-states>
  <allosteric-sites>
    <mol-instance-ref name="the-Cdk">
      <binding-site-ref name="functional-1">
	<site-shape-ref name="active" />
      </binding-site-ref>
    </mol-instance-ref>
  </allosteric-sites>
</allosteric-plex>
\end{ExampleXML}
 
\section{Defining reaction rules in the 'reaction-gens' element}

The reaction-gens section is where all of the rules for extrapolating
reactions go.  Currently, there are two
reaction-generators: dimerization-gen\glossary{name={modification},
  description={A modification is a changable part of the mols.}}, which represents a
protein-protein binding interaction as well its reverse decomposition
reaction, and the omni-gen, which represents a generic reaction that
can express unary and binary enzymatic reactions and other reaction
forms.  

\section{'dimerization-gen'}
A dimerization-gen reaction generator represents a generic
dimerization/decomposition reaction between a pair of modular binding
domains belonging to two proteins.  

The first information needed by a dimerization-gen is two mol-ref
elements, each with a site-ref subelement, which each specify one of
the binding sites that participates in this dimerization reaction.
For instance, if the model has defined (in the mod-mol section) a
protein named 'Substrate' that has a binding site 'to-Enzyme', the
proper way to specify this binding site is by writing 

\begin{ExampleXML}
<mol-ref name=''Substrate''>
  <site-ref name="to-Enzyme" />
</mol-ref>
\end{ExampleXML}

Following two of these elements, two elements: default-on-rate-value
and default-off-rate value must be provided.  Each must have a value
element, which corresponds to the basic reaction rate.  Units of this
are (Hz)(l)/mol.  See /ref{simpleDimerizationExample} for a basic
example of a dimerization reaction between two proteins.

\begin{ExampleXML}
<dimerization-gen>
  <mol-ref name="Substrate">
    <site-ref name="to-Enzyme" />
  </mol-ref>
  <mol-ref name="Enzyme">
    <site-ref name="to-Substrate" />
  </mol-ref>
  <default-on-rate value="1.0e12" />
  <default-off-rate value="1.0" />
</dimerization-gen>
\end{ExampleXML}

Each allosteric form of the reaction can be listed in its own
allo-rates element.  Each allo-rates element consists of two
site-shape-ref elements, and on-rate and off-rate elements.  When the
first binding-site in the dimerization reaction has the first
binding-site shape, and the second binding-site has the second shape,
then the allo-rates on and off rates are used instead.  So in
/ref{alloRatesExample}, Substrate will bind at its 'to-Enzyme' site
with Enzyme at its 'to-Substrate' site at a rate of of 1.0e12 (Hz) (l)
/ sec, unless the 'to-Enzyme' site has the shape 'occluded', in which
case the binding rate is 0.0.

\begin{ExampleXML} 
<dimerization-gen>
  <mol-ref name="Substrate">
    <site-ref name="to-Enzyme" />
  </mol-ref>
  <mol-ref name="Enzyme">
    <site-ref name="to-Substrate" />
  </mol-ref>
  <default-on-rate value="1.0e12" />
  <default-off-rate value="1.0" />
  <allo-rates>
    <site-shape-ref name="Obstructed" />
    <site-shape-ref name="default" />
    <on-rate value = 0.0 />
    <off-rate value = 1.0e12 />
  </allo-rates>
</dimerization-gen>
 \end{ExampleXML}


\section{uni-mol-gen}
A uni-mol-gen is a reaction generator which responds to the presence
of a single enabling mod-mol.  By adding enabling modification, you
can arrange that the reaction generator only creates reactions whose
enabling substrate is in a particular modification state.  The
generated reaction can ghange the modification state of the mol by
adding modification-exchanges.  This reaction can be made
binary by adding additional product species.  Users can likewise add
additional product species in the 'additional-product-species'
section.  

\begin{ExampleXML}[caption='A simple uni-mol-gen that represents a protein with two phosphorylation sites that is ubiquitinated']
<uni-mol-gen>
  <enabling-mol name='Protein' />
  <enabling-modifications>
    <mod-site-ref name='PhosphorylationSite_1'>
      <mod-ref name='phosphorylated' />
    </mod-site-ref>
    <mod-site-ref name='PhosphorylationSite_2'>
      <mod-ref name='phosphorylated' />
    </mod-site-ref>
  </enabling-modifications>
  <modification-exchanges>
    <modification-exchange>
      <mod-site-ref name='UbiquitinationSite' />
      <installed-mod-ref name='Ubiquitinated' />
    </modification-exchange>
  </modification-exchanges>
  <rate value='1e10' />
</uni-mol-gen>  
\end{ExampleXML}

\section{'omni-gen'}

An omni-gen is a generic reaction generator that can generate, for
each complex species containing a particular omniplex, a reaction with
flexible characteristics. The generated reactions can change any one
small-mol component of the omniplex into another small-mol, if
desired. The generated reactions can change the modification at any
one modification site on any one mod-mol component of the omniplex, if
desired. The generated reactions can be binary, if desired, requiring
any explicit reactant species as a co-reactant with the species that
is recognized as containing the omniplex. The generated reactions can
produce an arbitrary explicit product species, in addition to the
transformed primary reactant, if desired. Each of these separate
activities on the part of the generated reactions is engaged simply by
including the appropriate optional elements in this reaction generator
specification. 

\begin{ExampleXML}
\end{ExampleXML}

\section{'explicit-species'}

This section is where users describe species explicitly because you
want to create molecules of the species at some point during the
simulation, because you want to track the population of the species
through the simulation, or because you want to define explict
reactions involving the species. Some things that you might think
would be explicit species, such as a monomeric protein, may not be;
rather, they are constituents of species of complexes. (The monomeric
protein is a "mol" but you can define an explicit plex-species that
contains just the one mol.)

\begin{ExampleXML}
<explicit-species>
  <plex-species name='AA_Dimer'>
    <mol-instance name='First_A'>
      <mol-ref name='A' />
    </mol-instance>
    <mol-instance name='Second_A'>
      <mol-ref name='A' />
    </mol-instance>
    <binding>
      <mol-instance-ref name='First_A'>
        <binding-site-ref name='to-A' />
      </mol-instance-ref>
      <mol-instance-ref name='Second_A'>
        <binding-site-ref name='to-A' />
      </mol-instance-ref>
    </binding>
  </plex-species>
</explicit-species>
\end{ExampleXML}

\section{'explicit-reactions'}

Most of the reactions that libmoleculizer works with are generated
automatically by reaction generators, but you can enter reactions
one-at-a-time, too. The participants in an explicit reaction have to
be explicit species, so that they have names and you therefore have a
way to refer to them in the explicit reaction's definition. 


\begin{ExampleXML}
<explicit-species>

  <plex-species name='A-singleton'>
    <mol-instance name='the-A'>
      <mol-ref name='A' />
    </mol-instance>
  </plex-species>

  <plex-species name='B-singleton'>
    <mol-instance name='the-B'>
      <mol-ref name='B' />
    </mol-instance>
  </plex-species>

  <plex-species name='AA_Dimer'>
    <mol-instance name='First_A'>
      <mol-ref name='A' />
    </mol-instance>
    <mol-instance name='Second_A'>
      <mol-ref name='A' />
    </mol-instance>
    <binding>
      <mol-instance-ref name='First_A'>
        <binding-site-ref name='to-A' />
      </mol-instance-ref>
      <mol-instance-ref name='Second_A'>
        <binding-site-ref name='to-A' />
      </mol-instance-ref>
    </binding>
  </plex-species>

  <plex-species name='AB_Dimer'>
    <mol-instance name='the-A'>
      <mol-ref name='A' />
    </mol-instance>
    <mol-instance name='the-B'>
      <mol-ref name='B' />
    </mol-instance>
    <binding>
      <mol-instance-ref name='the-A'>
        <binding-site-ref name='to-B' />
      </mol-instance-ref>
      <mol-instance-ref name='the-B'>
        <binding-site-ref name='to-A' />
      </mol-instance-ref>
    </binding>
  </plex-species>

</explicit-species>

<explicit-reactions>

  <reaction>
    <substrate-species-ref name='A-singleton' multiplicity='2' />
    <product-species-ref name='AA_Dimer' />
  </reaction>

  <reaction>
    <substrate-species-ref name='A-singleton' multiplicity='1' />
    <substrate-species-ref name='B-singleton' multiplicity='1' />
    <product-species-ref name='AB_Dimer' />
  </reaction>

</explicit-reactions>  
\end{ExampleXML}

%%% Local Variables: 
%%% mode: latex
%%% TeX-master: "user-manual"
%%% End: 
