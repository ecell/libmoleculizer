\chapter{Cookbook}

This chapter contains explicit examples of many standard model
constructs that users may find useful.

Note that in the examples in this chapter, certain information needed
in the model may not be explicitly listed.  For instance, in an
example of a dimerization reaction, the mod-mols involved may not be
explicitly defined, although they would need to be in an explicit model.

\section{Defining Explict Modifications}

Most models of signal transduction pathways will include
modifications; at the least, modifications representing
phosphorylation groups are often added.  A 'None' or 'Null'
modification almost needed as well to represent a modification that
doesn't exist: an 'un-phosphorylated' or 'un-ubiquitinated' or
generally 'un-modified' state.

\ref{'exampleSimpleModSection'} gives an example of a simple
modification section that defines one phosphorylation section.

\begin{lstlisting}[caption='Example of a simple modification section', label='exampleSimpleModSection']
<modifications>
  <modification name='None'>
    <weight-delta daltons='0.0' />
  </modification>
  <modification name='phosphorylated'>
    <weight-delta daltons='42.0'/>
  </modification>
</modifications>
\end{lstlisting}

\section{Defining Abstract Modifications}

It is sometimes adventageous to define modifications that do not
explicitly correspond to physical modifications. For instance, when
modeling a doubly-phosphorylatable kinase target, it may be desirable
to model the target as having only one modification site that can be
in a 'singly-phosphorylated' or 'doubly-phosphorylated' state.  This
can simplify reaction generation at the expense of some physical
explicitness.  

Using this simplified and abstractified system, only two reaction
generators must be written, which take the target from 'None' to
'phosphorylated', and another from 'phosphorylated' to
'doubly-phosphorylated'.  

\begin{lstlisting}
<modifications>
  <modification name='None'>
    <weight-delta daltons='0.0' />
  </modification>
  <modification name='phosphorylated'>
    <weight-delta daltons='42.0'/>
  </modification>
  <modification name='doubly-phosphorylated'>
    <weight-delta daltons='84.0'/>
  </modification>
</modifications>
\end{1stlisting}

\section{Defining a binding-site on a mod-mol}
\section{Defining an allosteric form of a binding-site}
\section{Defining a site that has different behavior based on the
  state of the containing protein}
\section{Defining a binding site that has different behavior based on
  more complicated factors}
\section{Defining a simple dimerization reaction}

The most general kind of reaction generator in libmoleculizer is the
dimerization reaction, which defines a binding reaction between two
proteins.  

In its simplest form, it can be written as follows.  

\section{Defining a dimerization reaction with allosteric kinetics}
\section{Defining a modification state decomposition}
A user may wish to express that a particular modified protein will
decompose into an unmodified protein at some particular rate.

This can be done using a uni-mol-gen.

\begin{lstlisting}
<uni-mol-gen>
  <enabling-mol name='Protein' />
  <enabling-modifications>
    <mod-site-ref name='PhosphorylationSite'>
      <mod-ref name='phosphorylated' />
    </mod-site-ref>
  </enabling-modifications>
  <modification-exchanges>
    <modification-exchange>
      <mod-site-ref name='PhosphorylationSite' />
      <installed-mod-ref name='None' />
    </modification-exchange>
  </modification-exchanges>
  <rate value='1e5' />
</uni-mol-gen>
\end{lstlisting}

\section{Defining a Complicated Reaction}