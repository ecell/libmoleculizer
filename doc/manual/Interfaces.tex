\chapter{Libmoleculizer Interfaces}
Libmoleculizer has multiple wrappings that allow users of several
languages -- c++, c, or Python -- to use it in their own programs.
This chapter describes each of them, as well as their intended use,
and is followed by a section that discusses extending this procedure
to new languages.  

\subsection{C++ Interface}
This is the standard interface for libmoleculizer, as well as the
original coding language for the code.  Consequently, the c++
interface is the most powerful as well as the most detailed.

The c++ interface is composed of two portions. The
ReactionNetworkDescription portion records all the species and
reactions within the model and provides an interface to 
locating species as well as locating reactions between them.  The
Moleculizer portion of the interface is charged with managing the reaction
network expansion settings, as well as loading sets of rules and
species to be generated.  Together, these two make up the
libmoleculizer interface.  These two portions of the code are
given an overview below.  For a full discussion of the API, either
consult the reference portions of this documentation, or browse the
source code. 

\subsubsection{The Moleculizer Interface}
This interface (found in file src/mzr/moleculizer.hh) controls loading
and unloading rules files, and species descriptions, as well as
controlling parameters of how the species reaction network is to be
generated independantly of the rules file (setting things like type of
extrapolation, what things are being extrapolated, and other key
parameters.)

Creating a Moleculizer object is easy, as moleculizer only has one
constructor: moleculizer::moleculizer().  An individual wishing the
libmoleculizer library to be used in his/her work then would begin by
creating a moleculizer object. 

mzr::moleculizer m; // All the relevant code is contained in the
                    // namespece mzr.

To begin a libmoleculizer session, a rules file must be loaded.
However, before this occurs, the session must be configured using the
moleculizer interface.

Libmoleculizer comes default with two basic schemes for reaction and
species generation -- a spatial version in which radii,  diffusion
coefficients, and masses of the newly created species are generated, and the
intrinsic reaction rate is generated for newly genrated reactions, and
a non-spatial mode in which mass is generated for the species, and the
reaction rate (equivelent to the reaction rates in either differential
equation systems or stochastic, Gillespie-type reaction systems) is
generated for new reactions.  The moleculizer interface facilitates
switching beetween these options.

For the user of libmoleculizer, there are two ways that the
operational mode (spatial or non-spatial) can be set, using the
interface or set in the rules.  

Setting this up using the api, is quite straightforward.  If spatial
network is desired, the member function
``enableSpatialReactionNetworkGeneration()'' must be called prior to
rules being loaded.  What does this mean?  It means operationally,
that libmoleculizer will generate masses, radii, and diffusion
coefficients for each species that is constructed.  It also means
that, in order to do this, the user must provide diffusion
coefficients for each of the different mols in the model, as well as
each of the named explicit-plex-species the user defines  With this
mode of operation, the functions getKDForSpecies and
getRadiusForSpecies can be called on any species pointer owned by the
ReactionNetowrkDescription..

If non-spatial network generation is desired, the moleculizer member
function ``enableNonspatialReactionNetworkGeneration(bool)'' may be
called.  This mode will only extrapolate masses of species, and the
reaction rate of generated reactions.  The boolean value here
determines whether the extrapolation of the reaction rate will use
mass-extrapolation in calculating the reaction rate or not.  

When the value passed in is false, overall reaction rates are
calculated as follows.  Each reaction rule created by the user is
provided with a basic on-rate and off-rate, which may be adjusted by
specifying allostery (particular structural properties of the complex
the mol exists within) that modifies these on and off rates.  If a
particular pair of species is found to react, the most specific
reaction rates associated with this rule (the basic rates are used
unless an allosteric 'special case' applies to the reaction, in which
case the allosteric rates are used) are used as the overall reaction
rate, without modification.

When the value passed in is true, mass-based reaction rate
modification is used.  The basic on and off rates are obtained as
described in the previous paragraph.  However, the final
reaction-specific on and off rates are obtained by multiplying the
basic rates by a mass-related correction factor.  [[[Show factor and
discuss moleculizer paper.]]]

Once configuration parameters have been set up, a model can be
loaded.  This can be accomplished using one of three functions:
``attachFileName( const std::string& aFileName )'', 
``attachString( const std::string& documentAsString )'',
or ``attachDocument( xmlpp::Document* pDoc )''.  Each of these takes a
model, in one form or another, loads it, parses it, and loads it,
preparing Moleculizer for species and reaction generation.  

A libmoleculizer model is an xml description of basic molecules (mols) and
reaction rules defined in terms of molecules, that usually describe
either protein-protein interactions or enzymatic reactions on
proteins.  A description of the full syntax of these files can be
found in the chapter 'Rules'.  An xml specification can be either
passed to libmoleculizer as a string containing file contents, a path
to a model file, or can be passed in as an xmlpp::Document*
(corresponding to an xml document parsed as a dom object using
libxml++, the xml library used by libmoleculizer).

Use cases:
1.  Model file is contained in a file.
If filePath is a string containing either an absolute or relative path
to the file, then the function moleculizer::attachFileName can be
called as follows.  

// Create a new reaction-network generator.
mzr::moleculizer m;

// Setup extrapolation here. 
m.enableSpatialReactionNetworkGeneration(); // For instance.

// Load the file into the moleculizer object.
m.attachFileName( filePath );

2..Model is contained in a string.
If modelContents is a string containing a model file, then the
function moleculizer::attachString can be used to load this model file
as follows

// Create a new reaction-network generator.
mzr::moleculizer m; 

// Setup extrapolation here.
m.enableNonspatialReactionNetworkGeneration( true ); // For instance.

// Load the string holding the model file into the moleculizer object.
m.attachString( modelContents );

3.  Model has already been parsed and is contained in an
xmlpp::Document* object called ptrModelDoc.

// Create a new reaction-network generator.
mzr::moleculizer m;

// Setup extrapolation here.
m.enableNonspatialReactionNetworkGeneration( false );  // For
instance.

// Load the Document* into moleculizer.
m.attachDocument( ptrModelDocument );

Any of these will load the necessary information into a model
instance. Note however, that information can only be added into
moleculizer once per instance.  Should the model have to be updated,
the moleculizer object must be destroyed and recreated and loaded with
the new model.  This is not optimal, and may change in upcoming
versions.  Users can determine whether a model is loaded by calling
the 'bool moleculizer::getModelHasBeenLoaded() const' function.  

The moleculizer interface also contains one more critical function,
the function 
'const mzrSpecies* getSpeciesWithName( const std::string speciesName)'
function.  A user can provide a species key (interchangably referred
to as a species name) to molculizer, and it will return a species
pointer to that species.  The primary feature of this function is that
it will create the species if it does not already exist.  However, if
it does exist, this function will merely return a pointer to that
species.  If the species does not refer to a legal species, per the
rules that have been loaded in, it will throw a mzr::IllegalNameXcpt.  

\subsubsection{The ReactionNetworkDescription interface.}
The ReactionNetworkDescription interface
(fnd::ReactionNetworkDescription<mzr::mzrSpecies, mzr::mzrReaction> in
code, found in src/fnd/reactionNetworkDescription.hh) manages questions
about species and reactions that have been created within a session of
libmoleculizer.  The basic, relevant functions to using this interface
are listed below.  Full a full description of this interface, please
consult class documentation provided elsewhere in this documentation.

Generally speaking, the ReactionNetworkDescription interaface can do
three things:

1.
Find species with certain names, using the findSpecies function.  This
function takes the name of a species and either returns a constant
pointer to that species, or throws a fnd::NoSuchSpeciesXcpt
exception.  Note that an exception may be thrown even if the species
name is a legal one for this model, so long as that species has not
been constructed yet.  Typically this is appropriate.  For instance,
any species formed as a procuct of a generated reaction is guaranteed
to have been created such that it will be returned by findSpecies.
That said, there may be special cases, such as working with data from
a previous run of the same model, where legal species names must be
used to create species even though the current expansion of the model
has not expanded that portion yet.  In this case,
mzr::moleculizer::getSpeciesWithName, as described in the previous
section, should be used.  

2.  Find reactions with one or more substrates.
This is accomplished using the findReactionWithSubstrates functions.
If unary reactions are desired (what decompositions can a particular
species undergo), use the function 'bool
findReactionWithSubstrates(SpeciesCPtr A,
std::vector<ReactionTypeCPtr>& reactionVector)' function.  To use
this, a species pointer that has been gotten using either the
findSpecies or getSpeciesWithName functions is passed, along with an
empty vector containing ReactionTypeCPtrs.  The function places a
unique pointer to every unary reaction involving that species as a
substrate, returning true if any exist.  

For instance:
m = mzr::molecuculizer();
m.attachFile( ``Simpledemo-rules.xml''); 

std::vector<const mzr::mzrReaction*> speciesDecompositionRxns;;

// This function also converts user names, described in the rules, to
// species pointers.
const mzr::mzrSpecies* species = m.getSpeciesWithName(``A-B-dimer'');

// When this function returns, speciesDecompositionRxns will contain
// one element, corresponding to the ``AB -> A + B'' reaction.  
m.findReactionWithSubstrates( species, speciesDecompositionRxns);

For reactions involving two substrates (libmoleculizer does not
support reactions involving three or greater substrates at this time),
the same idea is followed, except involving two species.  

Example:
m = mzr::moleculizer();
m.attachFile(``Simpledemo-rules.xml'');
std::vector<const mzr::mzrReaction*> binaryRxns;;

const mzr::mzrSpecies* species1 = m.findSpecies(``___1A______'');
const mzr::mzrSpecies* species2 = m.findSpecies(``___1B______'');

// After this function call, binaryRxns will contain one element
// corresponding to the reaction ``A + B -> AB'').
m.findReactionWithSubstrates(species1, species2, binaryRxns);


ReactionNetworkDescriptionInterface also provides general information
on the state of the generated network through several functions:

unsigned int getTotalNumberSpecies() const, 
unsigned int getTotalNumberReactions() const,
bool checkSpeciesIsKnow( const std::string& speciesName) const.

These functions all provide information about the state of the
generated network.  The first two tell how large it is, and the third
helps determine what is and is not in it.  





\subsection{C Interface}
\subsection{Python Interface}



