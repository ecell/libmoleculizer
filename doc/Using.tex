\chapter{*Using Libmoleculizer}
\label{chap:usingLibmoleculizerChapter}

To use libmoleculizer, a libmoleculizer object must be created in
code, a set of rules that describes a biochemical network must be
loaded into it, and then various functions in the libmoleculizer
interface must be called in order to either expand the reaction
network in a controlled fashion or to read information about the
portion of the reaction network that has already been generated.  

This chapter gives an overview to the various kinds of tasks and
capabilities of libmoleculizer.  Although it is not an exhaustive list
of every function in the libmoleculizer interface -- for this, please
consult chapter \ref{chap:apiReference} -- we hope it will provide a
kind of bird's eye view to how to use libmoleculizer, providing ideas
and examples of different patterns.  As extra-credit reading, you may
also wish to check out the demo programs that use libmoleculize.
These can be found in the demos/ directory of the code, and can be
built by passing the configure script an --enable-demos flag.  Please
see the section ``Enabling the demos'', in chapter
\ref{chap:installingChapter} for more details. 

Please note that this chapter only deals with using libmoleculizer in
the C++ interface.  

\section{Expanding Reaction Networks}



